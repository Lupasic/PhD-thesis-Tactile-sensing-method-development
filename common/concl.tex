%% Согласно ГОСТ Р 7.0.11-2011:
%% 5.3.3 В заключении диссертации излагают итоги выполненного исследования, рекомендации, перспективы дальнейшей разработки темы.
%% 9.2.3 В заключении автореферата диссертации излагают итоги данного исследования, рекомендации и перспективы дальнейшей разработки темы.

Основной  научный  результат  диссертации --- методы построения определения геометрических и физико-механических свойств опорной поверхности на базе многоногого шагающего аппарата с тактильным очувствлением без использования оптических сенсоров.

Данное решение подходит для первичного исследования замкнутых труднодоступных пространств, где отсутствует освещение, обилие грязи, пыли, а так же водных препятствий. Алгоритмы и концепты навигации данной системы могут быть использованы как резервная система навигации для других робототехнических систем, когда главная система, которая является более точной, из-за природы использованных датчиков, вышла из строя.


При  проведении  исследований  и  разработок  в  диссертационной  работе  получены следующие результаты.
\begin{enumerate}
  \item Был проведен обзор и анализ робототехнических систем и условия их применения. Обобщая, была проведена классификация машин, использующих ноги в качестве движителя. Наиболее полно были рассмотрены машины с циклическими движителями. В литературный обзор вошли роботы, которые могут быть использованы для исследования пещер. Была предложена их классификация.

  Более того, для понимания условий применений разрабатываемой робототехнической системы, было описаны параметры исследуемых пещер и их особенности.

  Для разработки системы, важной частью которой является сенсорные устройства, был проведен глубокий их обзор и классификация. Так же был проведен литературный обзор алгоритмической части работы с сенсорами, к примеру обзор алгоритмов по триангуляцию.

  Выводом обзора является описание разработанной системы.
  \item Разработан метод оптимизации конструкции многоногих шагающих роботов с цикловыми движителями с одной степенью свободы по критериям проходимости (длина робота), детализации (количества ног), пройденного пути.

  Данный метод основан на применении генетического алгоритма OpenAI-ES, где были разработаны и реализованы операции скрещивания и мутации. Была разработана математическая модель робота, которая была реализована в GazeboSim. 
  
  Для генерации семейства роботов было предложено геометрическое представление объекта. Так же пришлось разработать способ для генерирования местности, которую будет проходить экземпляр робота.

  Помимо оптимизации конструкции по предложенным выше критериям, был разработан метод оптимизации конструкции робота для прохождения узких участков. Это важный концепт, так как по обзору пещер стало ясно, что пещеры имеют очень большую девиацию в ширину.
  \item Изучив существующие тактильные сенсоры было решено разработать и исследовать преобразователь силы на основе Velostat.

  В течение разработки сенсора были найдена особенность, что при одинаковой силе нажатия на сенсор, возникают различные результаты, в зависимости от места нажатия и площади нажатия. Для исследования данного феномена был разработан автоматизированный экспериментальный стенд. 
  
  По результатам поставленных экспериментов, характеристики преобразователя удовлетворяют требованиям к системе тактильного восприятия шагающего робота, когда ожидаемый размер площади контакта превышает 25 процентов площади преобразователя.
  \item Был разработан метод определения геометрических свойств поверхности с помощью ощупывания. Он основан на алгоритме вогнутой Триангуляции Делоне с использованием альфа формы. Для первичной проверки гипотез была разработана сцена в симуляторе CoppeliaSim.

  Так же результат интеллектуальной деятельности проверялся на натурном эксперименте и была получена точность в 8 см, что является приемлемой для данной задачи.
  \item Для получение максимально полной информации о проходимой поверхности, необходимо знать её физико-механические свойства. Это было реализовано с помощью алгоритмов машинного обучения, а все данные были получены из натурных экспериментов. Как результат, стало возможно определять процентное соотношение твердых, упругих и пластинчатых свойств пройденной поверхности.
\end{enumerate}