%% Согласно ГОСТ Р 7.0.11-2011:
%% 5.3.3 В заключении диссертации излагают итоги выполненного исследования, рекомендации, перспективы дальнейшей разработки темы.
%% 9.2.3 В заключении автореферата диссертации излагают итоги данного исследования, рекомендации и перспективы дальнейшей разработки темы.

Основной  научный  результат  диссертации  заключается  в  решении  актуальной 
научной  задачи,  имеющей  важное  практическое  значение: разработка и исследование робототехнической системы построения карты местности и определения геометрических и физических свойств опорной поверхности на базе многоногого шагающего аппарата с тактильным очувствлением без использования оптических сенсоров.

Данное решение отлично подходит для первичного исследования замкнутых труднодоступных пространств, где отсутствует освещение, обилие грязи, пыли, а так же водных препятствий. Алгоритмы и концепты навигации данной системы могут быть использованы как резервная система навигации для других робототехнических систем, когда более точная --- оптическая вышла из строя.


Предложенное решение очень хорошо подходит для изначального изучения закрытых недоступных мест, где отсутствует освещение, много грязи, пыли, и водных преград. Методы и концепты навигации предоставленной системы имеют все шансы быть применены как запасная система навигации для иных робототехнических систем, когда больше точная --- оптическая вышла из строя.

При  проведении  исследований  и  разработок  в  диссертационной  работе  получены 
следующие результаты.
\begin{enumerate}
  \item Был проведен обзор и анализ робототехнических систем и условия их применения. То есть была проведена классификация машин, использующих ноги в качестве дивжителя, упор был сделан на машины с циклическими движетилями. Рассмотрены роботы, которые могут быть использованы для исследования пещер. Была предложена их классификация.

  Более того, для понимания условий применений разрабатываемой робототехнической системы, было описаны параметры исследуемых пещер и их особенности.

  Для разработки системы, важной частью которой является сенсорные устройства, был проведен глубокий их обзор и классификая. Так же был проведен литературный обзор алгоритмической части работы с сенсорами, к примеру обзор алгоритмов по триангуляциию.

  Выводом обзора является описание применимости разработанной системы.
  \item Разработан метод оптимизации конструкции многоногих шагающих роботов с цикловыми движителями с одной степенью свободы по критериям проходимости (длина робота), детализации (количества ног), пройденного пути.

  Данный метод основан на применении генетического алгоритма OpenAI-ES, с самописными реализацией скрещивания и мутации. Была разработана математическая модель робота, которая была реализована в GazeboSim. 
  
  Для генерации семейства роботов было предложено геометрическое представление объекта. Так же пришлось разработать способ для генерирования местонсти, которую будет проходить экземляр робота.

  Помимо оптимизации конструкции по предложенным выше критериям, была разработан метод оптимизации конструкции робота для прохождения узких участков. Это важный концепт, так как по обзору пещер стало ясно, что пещеры имеют очень большую девиацию в ширину.
  \item Изучив существующие тактильные сенсоры было решено разработать и исследовать преобразователь силы на основе Velostat. Для этого пришлось физические создать преобразователь, адаптировать его под конкретное применение.

  В течение разработки сенсора были найдена особенность, что при одинаковой силе нажатия на сенсор, возникают различные результаты, в зависимости от места нажатия и площади нажатия. Для исследования данного феномена был разработан автоматизированный экспериментальный стенд. 
  
  По результатам поставленных экспериментов, характеристики преобразователя удовлетворяют требованиям к системе тактильного восприятия шагающего робота, когда ожидаемый размер площади контакта превышает 25 процентов площади преобразователя.
  \item Было разработан метод картографирования с помощью ощупывания поверхности. Для первичной проверки гипотез была разработана сцена в симуляторе CoppeliaSim. После череды экспериментов, было решено использовать и реализовать алгоритмы вогнутой Триангуляции Делоне с использованием альфа формы.

  Так же результат интеллектуальной деятельности проверялся на в натурном эксперименте и была получена точность в 8 см, что является приемлимой для данной задачи.
  \item Для получение максимально полной информации о проходимой поверхности, необходимо знать еще и тип поверхности, по которой проходит робот. Это было реализовано с помощью методов машинного обучения, а все данные были получены из натурных экспериментов. Как результат, точность определения примерно в 85 процентов (3 типа поверхности, земля, резина и камень), позволяет довольно эффективно использовать данные знания для реализации алгоритмов управления.
\end{enumerate}

Подведя итог, данная работа имеет как практическую, так и теоретическую составляющую. Предложенные методы апробировались на роботе.