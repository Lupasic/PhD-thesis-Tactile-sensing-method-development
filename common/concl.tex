%% Согласно ГОСТ Р 7.0.11-2011:
%% 5.3.3 В заключении диссертации излагают итоги выполненного исследования, рекомендации, перспективы дальнейшей разработки темы.
%% 9.2.3 В заключении автореферата диссертации излагают итоги данного исследования, рекомендации и перспективы дальнейшей разработки темы.

Основной  научный  результат  диссертации --- метод построения карты местности с определением геометрических и физико-механических свойств опорной поверхности с помощью многоногого шагающего аппарата с тактильным очувствлением без использования оптических сенсоров.

Полученное решение подходит для первичного исследования замкнутых труднодоступных пространств, где отсутствует освещение, имеется обилие грязи, пыли, а так же водных препятствий. Алгоритмы и концепты навигации данной системы могут быть использованы как резервная система навигации для других робототехнических систем, когда главная система, которая является более точной, из-за природы использованных датчиков, вышла из строя.


При  проведении  исследований  и  разработок  в  диссертационной  работе  получены следующие результаты.
\begin{enumerate}
  \item Был проведен обзор и анализ робототехнических систем и условия их применения. Проведена классификация машин, использующих ноги в качестве движителя. Наиболее полно были рассмотрены машины с циклическими движителями. В литературный обзор вошли роботы, которые могут быть использованы для исследования пещер. Была предложена их классификация.

  Для понимания условий применений разрабатываемой робототехнической системы, были описаны параметры исследуемых пещер и их особенности.

  Для разработки системы, важной частью которой является сенсорные устройства, был проведен глубокий их обзор и классификация. Так же был проведен литературный обзор алгоритмической части работы с сенсорами, к примеру обзор алгоритмов по триангуляцию.

  По результатам анализа была сформирована концепция разрабатываемой системы.
  \item Разработан метод оптимизации конструкции многоногих шагающих роботов с цикловыми движителями с одной степенью свободы по критериям проходимости (длина робота), детализации (количества ног), пройденного пути.

  Данный метод основан на применении генетического алгоритма OpenAI-ES, где были разработаны и реализованы операции скрещивания и мутации. Была разработана математическая модель робота, которая была реализована в GazeboSim. 
  
  Для генерации семейства роботов было предложено геометрическое представление объекта. Так же разработан способ для генерирования местности, которую будет проходить экземпляр робота.

  Помимо оптимизации конструкции по предложенным выше критериям, был разработан метод оптимизации конструкции робота для прохождения узких участков.
  \item Разработан и исследован преобразователь силы на основе Velostat.

  ВОдной из особенностей такого сенсора является то, что при одинаковой силе нажатия на сенсор, возникают различные результаты, в зависимости от места нажатия и площади нажатия. Для исследования данного феномена был разработан автоматизированный экспериментальный стенд. 
  
  По результатам поставленных экспериментов показано, что характеристики преобразователя удовлетворяют требованиям к системе тактильного восприятия шагающего робота, когда ожидаемый размер площади контакта превышает 25 процентов площади преобразователя.
  \item Разработан метод определения геометрических свойств поверхности с помощью ощупывания. Метод основан на алгоритме вогнутой триангуляции Делоне с использованием альфа формы. Для первичной проверки гипотез была также разработана сцена в симуляторе CoppeliaSim.

 Метод был проверен в натурном эксперименте на разработанной модели робота. По результатам экспериментов погрешность определения формы поверхности не превышала 8 см, что является приемлемым для практического применения.
  \item Разработан метод определения физико-механических свойств опорной поверхности с помощью очувствления шагающего робота. Метод реализован на базе искусственной нейронной сети, и позволяет классифицировать различные типы поверхностей. Обучение нейронной сети проводилось на экспериментальных данных для трёх типов поверхности: с преобладанием твёрдых, упругих и пластичных свойств.
\end{enumerate}