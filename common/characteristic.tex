
{\actuality} Жизнь многих народов Земли неразрывно связана с пещерами. В них можно найти множество полезных ресурсов, такие как металлы, драгоценные камни редкие разновидности мхов.

Но изучение пещер всегда сопровождается большими опасностями и
трудностями. К примеру таковыми являются сифоны (рис 1), сталактиты, обилие скользких
грунтов. В пещерах недостаток света и тепла, часто влажно.

Одно из преимуществ роботов --- они могут работать в опасных средах без нахождения рядом человека. Таким образом использование роботов в пещерах нивелирует все опасности для человека.

Для полноценного функционирования робота в пещере необходимы сенсоры. Классическими внешними сенсорами являются камера и лидар.

Но так как пещеры по природе грязные, пыльные с обилием воды и с
недостатком света, то классические сенсоры могут легко выйти из строя. К примеру грязь может закрыть обзор камере. Водяной пар или водная гладь будет отражаться от лазера лидара и искажать полученные данные.

Эту проблему возможно частично решить с помощью другого набора сенсоров. С помощью внутренних сенсоров, на которые меньше влияют внешние факторы. Такими сенсорами являются Инерциальное Измерительное устройство (IMU), датчики силы и маяки.

Таким образом необходимо разработать методику получения полезной информации об окружающей среде с помощью данного набора датчиков. Так как основные данные получаются с помощью касаний, это называется тактильным очувствлением.

Подтверждением актуальности проблемы могут являться DARPA Subterranean Challenge, отчет, который опубликовала команда Яндекс Беспилотники, а так же выделенные средства фондов НТИ и РФФИ на решение этой проблемы.

% \ifsynopsis
% Этот абзац появляется только в~автореферате.
% Для формирования блоков, которые будут обрабатываться только в~автореферате,
% заведена проверка условия \verb!\!\verb!ifsynopsis!.
% Значение условия задаётся в~основном файле документа (\verb!synopsis.tex! для
% автореферата).
% \else
% Этот абзац появляется только в~диссертации.
% Через проверку условия \verb!\!\verb!ifsynopsis!, задаваемого в~основном файле
% документа (\verb!dissertation.tex! для диссертации), можно сделать новую
% команду, обеспечивающую появление цитаты в~диссертации, но~не~в~автореферате.
% \fi

{\aim} данной работы является разработка метода тактильного очувствления мобильного шагающего робота в закрытых пространствах естественного или искусственного происхождения, в которых невозможно получение данных со спутниковой навигации, затруднено применение оптических сенсоров.

Для~достижения поставленной цели необходимо было решить следующие {\tasks}:
\begin{enumerate}[beginpenalty=10000] % https://tex.stackexchange.com/a/476052/104425
  \item Исследовать, разработать, вычислить и~т.\:д. и~т.\:п.
  \item Исследовать, разработать, вычислить и~т.\:д. и~т.\:п.
  \item Исследовать, разработать, вычислить и~т.\:д. и~т.\:п.
  \item Исследовать, разработать, вычислить и~т.\:д. и~т.\:п.
\end{enumerate}

{\researchobj}
Для разведки местности под землей в труднодоступных местах с малой видимостью разработан прототип многоногого шагающего робота СтриРус (СТАТЬИ МОИ). Ниже (Рис. 1) представлена САПР модель сборки робота.

Рис. 1 Многоногий шагающий робот СтриРус, модель в САПР

Данный робот состоит из двух сегментов с одной активной степенью свободы. Робот обладает 10 независимыми лапками, 6 лап в первом сегменте и 4 во втором.

Особенность конструкции робота в том, что возможно изменять угол между лапкой и корпусом робота. Данное конструктивное изменение позволило сделать перемещение робота всенаправленным, то есть робот может двигаться во все стороны без смены ориентации корпуса робота.


{\methods} За базис были взяты методологии из теории робототехнических систем, теоретической механики, механизмов и машин, теории оптимизации.

Для экспериментального исследования применялось численное и стендовое моделирования.

{\reliability} Правдивость результатов обеспечивается согласованностью с опубликованными результатами научных исследований других авторов, подтверждаются результатами компьютерного моделирования, натурными испытаниями. Результаты диссертационного исследования докладывались и обсуждались на российских и международных научных конференциях, и получили положительный отзыв научной общественности.


{\novelty}
\textbf{Доказано}, \textbf{показано}, \textbf{предложено}, \textbf{установлено}, \textbf{сделан вывод}


{\defpositions}
\begin{enumerate}[beginpenalty=10000] % https://tex.stackexchange.com/a/476052/104425
  \item Результаты оптимизации конструкции
  \item Заключение статьи с Кириллом
  \item Метод (продумать как подать)
  \item Четвертое положение
\end{enumerate}


{\influence} Реализация полученных результатов в виде продукта позволит получить надежную резервную систему навигации для робота, у которого есть датчики силы, IMU и возможность устанавливать маяки по мере продвижения.

Если у робота есть только датчики силы, будет возможно получать информацию о типе пройденной поверхности, а так же строить карту поверхности под толщей воды, там где лидар и камера не смогут выдать адекватный результат.


{\probation}
Основные результаты работы докладывались~на:
перечисление основных конференций, симпозиумов и~т.\:п.

{\contribution} Все научные результаты диссертации, выдвигаемые для защиты, получены автором лично.

% Вставка кто сколько опубликовался
\input{common/fancybibcalc.tex}

{\struct}
В введении рассказывается об актуальности проблемы, в чем научная новизна и цель проекта.

Во первой главе показан обзор существующих решений.

Вторая глава покрывает разработку объекта исследования, а именно решение задачи топологического синтеза и инженерную разработку прототипа.

Третья глава посвящена разработке и исследованию самодельного преобразователя силы на основе Velostat.

Четвертая глава раскрывает детали создания алгоритма построения карты с помощью тактильного очувствления, решение проблемы локализации с помощью датчиков датчиков силы, IMU и маяков. Решатся проблема определения типа поверхности.