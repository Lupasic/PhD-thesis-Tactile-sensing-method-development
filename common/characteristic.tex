
{\actuality} Жизнь многих народов Земли неразрывно связана с пещерами. В них можно найти множество полезных ресурсов, такие как металлы, драгоценные камни редкие разновидности мхов.

Но изучение пещер всегда сопровождается большими опасностями и
трудностями. К примеру таковыми являются сифоны (рис 1), сталактиты, обилие скользких
грунтов. В пещерах недостаток света и тепла, часто влажно.

Одно из преимуществ роботов --- они могут работать в опасных средах без нахождения рядом человека. Таким образом использование роботов в пещерах нивелирует все опасности для человека.

Для полноценного функционирования робота в пещере необходимы сенсоры. Классическими внешними сенсорами являются камера и лидар.

Но так как пещеры по природе грязные, пыльные с обилием воды и с
недостатком света, то классические сенсоры могут легко выйти из строя. К примеру грязь может закрыть обзор камере. Водяной пар или водная гладь будет отражаться от лазера лидара и искажать полученные данные.

Эту проблему возможно частично решить с помощью другого набора сенсоров. С помощью внутренних сенсоров, на которые меньше влияют внешние факторы. Такими сенсорами являются Инерциальное Измерительное устройство (IMU), датчики силы и маяки.

Таким образом необходимо разработать методику получения полезной информации об окружающей среде с помощью данного набора датчиков. Так как основные данные получаются с помощью касаний, это называется тактильным очувствлением.

Подтверждением актуальности проблемы могут являться DARPA Subterranean Challenge, отчет, который опубликовала команда Яндекс Беспилотники, а так же выделенные средства фондов НТИ и РФФИ на решение этой проблемы.

% \ifsynopsis
% Этот абзац появляется только в~автореферате.
% Для формирования блоков, которые будут обрабатываться только в~автореферате,
% заведена проверка условия \verb!\!\verb!ifsynopsis!.
% Значение условия задаётся в~основном файле документа (\verb!synopsis.tex! для
% автореферата).
% \else
% Этот абзац появляется только в~диссертации.
% Через проверку условия \verb!\!\verb!ifsynopsis!, задаваемого в~основном файле
% документа (\verb!dissertation.tex! для диссертации), можно сделать новую
% команду, обеспечивающую появление цитаты в~диссертации, но~не~в~автореферате.
% \fi

{\aim} данной работы является разработка метода тактильного очувствления мобильного шагающего робота в закрытых пространствах естественного или искусственного происхождения, в которых невозможно получение данных со спутниковой навигации, затруднено применение оптических сенсоров.

Для~достижения поставленной цели необходимо было решить следующие {\tasks}:
\begin{enumerate}[beginpenalty=10000] % https://tex.stackexchange.com/a/476052/104425
  \item Исследовать, разработать, вычислить и~т.\:д. и~т.\:п.
  \item Исследовать, разработать, вычислить и~т.\:д. и~т.\:п.
  \item Исследовать, разработать, вычислить и~т.\:д. и~т.\:п.
  \item Исследовать, разработать, вычислить и~т.\:д. и~т.\:п.
\end{enumerate}

{\researchobj}
Для разведки местности под землей в труднодоступных местах с малой видимостью разработан прототип многоногого шагающего робота СтриРус (СТАТЬИ МОИ). Ниже (Рис. 1) представлена САПР модель сборки робота.

Рис. 1 Многоногий шагающий робот СтриРус, модель в САПР

Данный робот состоит из двух сегментов с одной активной степенью свободы. Робот обладает 10 независимыми лапками, 6 лап в первом сегменте и 4 во втором.

Особенность конструкции робота в том, что возможно изменять угол между лапкой и корпусом робота. Данное конструктивное изменение позволило сделать перемещение робота всенаправленным, то есть робот может двигаться во все стороны без смены ориентации корпуса робота.


{\methods} За базис были взяты методологии из теории робототехнических систем, теоретической механики, механизмов и машин, теории оптимизации.

Для экспериментального исследования применялось численное и стендовое моделирования.

{\reliability} Правдивость результатов обеспечивается согласованностью с опубликованными результатами научных исследований других авторов, подтверждаются результатами компьютерного моделирования, натурными испытаниями. Результаты диссертационного исследования докладывались и обсуждались на российских и международных научных конференциях, и получили положительный отзыв научной общественности.


{\novelty}
\textbf{Доказано}, \textbf{показано}, \textbf{предложено}, \textbf{установлено}, \textbf{сделан вывод}


{\defpositions}
\begin{enumerate}[beginpenalty=10000] % https://tex.stackexchange.com/a/476052/104425
  \item Результаты оптимизации конструкции
  \item Заключение статьи с Кириллом
  \item Метод (продумать как подать)
  \item Четвертое положение
\end{enumerate}


{\influence} Реализация полученных результатов в виде продукта позволит получить надежную резервную систему навигации для робота, у которого есть датчики силы, IMU и возможность устанавливать маяки по мере продвижения.

Если у робота есть только датчики силы, будет возможно получать информацию о типе пройденной поверхности, а так же строить карту поверхности под толщей воды, там где лидар и камера не смогут выдать адекватный результат.


{\probation}
Основные результаты работы докладывались~на:
перечисление основных конференций, симпозиумов и~т.\:п.

{\contribution} Все научные результаты диссертации, выдвигаемые для защиты, получены автором лично.

% Вставка кто сколько опубликовался
\ifnumequal{\value{bibliosel}}{0}
{%%% Встроенная реализация с загрузкой файла через движок bibtex8. (При желании, внутри можно использовать обычные ссылки, наподобие `\cite{vakbib1,vakbib2}`).
    {\publications} Основные результаты по теме диссертации изложены
    в~XX~печатных изданиях,
    X из которых изданы в журналах, рекомендованных ВАК,
    X "--- в тезисах докладов.
}%
{%%% Реализация пакетом biblatex через движок biber
    \begin{refsection}[bl-author, bl-registered]
        % Это refsection=1.
        % Процитированные здесь работы:
        %  * подсчитываются, для автоматического составления фразы "Основные результаты ..."
        %  * попадают в авторскую библиографию, при usefootcite==0 и стиле `\insertbiblioauthor` или `\insertbiblioauthorgrouped`
        %  * нумеруются там в зависимости от порядка команд `\printbibliography` в этом разделе.
        %  * при использовании `\insertbiblioauthorgrouped`, порядок команд `\printbibliography` в нём должен быть тем же (см. biblio/biblatex.tex)
        %
        % Невидимый библиографический список для подсчёта количества публикаций:
        \printbibliography[heading=nobibheading, section=1, env=countauthorvak,          keyword=biblioauthorvak]%
        \printbibliography[heading=nobibheading, section=1, env=countauthorwos,          keyword=biblioauthorwos]%
        \printbibliography[heading=nobibheading, section=1, env=countauthorscopus,       keyword=biblioauthorscopus]%
        \printbibliography[heading=nobibheading, section=1, env=countauthorconf,         keyword=biblioauthorconf]%
        \printbibliography[heading=nobibheading, section=1, env=countauthorother,        keyword=biblioauthorother]%
        \printbibliography[heading=nobibheading, section=1, env=countregistered,         keyword=biblioregistered]%
        \printbibliography[heading=nobibheading, section=1, env=countauthorpatent,       keyword=biblioauthorpatent]%
        \printbibliography[heading=nobibheading, section=1, env=countauthorprogram,      keyword=biblioauthorprogram]%
        \printbibliography[heading=nobibheading, section=1, env=countauthor,             keyword=biblioauthor]%
        \printbibliography[heading=nobibheading, section=1, env=countauthorvakscopuswos, filter=vakscopuswos]%
        \printbibliography[heading=nobibheading, section=1, env=countauthorscopuswos,    filter=scopuswos]%
        %
        \nocite{*}%
        %
        {\publications} Основные результаты по теме диссертации изложены в~\arabic{citeauthor}~печатных изданиях,
        \arabic{citeauthorvak} из которых изданы в журналах, рекомендованных ВАК\sloppy%
        \ifnum \value{citeauthorscopuswos}>0%
            , \arabic{citeauthorscopuswos} "--- в~периодических научных журналах, индексируемых Web of~Science и Scopus\sloppy%
        \fi%
        \ifnum \value{citeauthorconf}>0%
            , \arabic{citeauthorconf} "--- в~тезисах докладов.
        \else%
            .
        \fi%
        \ifnum \value{citeregistered}=1%
            \ifnum \value{citeauthorpatent}=1%
                Зарегистрирован \arabic{citeauthorpatent} патент.
            \fi%
            \ifnum \value{citeauthorprogram}=1%
                Зарегистрирована \arabic{citeauthorprogram} программа для ЭВМ.
            \fi%
        \fi%
        \ifnum \value{citeregistered}>1%
            Зарегистрированы\ %
            \ifnum \value{citeauthorpatent}>0%
            \formbytotal{citeauthorpatent}{патент}{}{а}{}\sloppy%
            \ifnum \value{citeauthorprogram}=0 . \else \ и~\fi%
            \fi%
            \ifnum \value{citeauthorprogram}>0%
            \formbytotal{citeauthorprogram}{программ}{а}{ы}{} для ЭВМ.
            \fi%
        \fi%
        % К публикациям, в которых излагаются основные научные результаты диссертации на соискание учёной
        % степени, в рецензируемых изданиях приравниваются патенты на изобретения, патенты (свидетельства) на
        % полезную модель, патенты на промышленный образец, патенты на селекционные достижения, свидетельства
        % на программу для электронных вычислительных машин, базу данных, топологию интегральных микросхем,
        % зарегистрированные в установленном порядке.(в ред. Постановления Правительства РФ от 21.04.2016 N 335)
    \end{refsection}%
    \begin{refsection}[bl-author, bl-registered]
        % Это refsection=2.
        % Процитированные здесь работы:
        %  * попадают в авторскую библиографию, при usefootcite==0 и стиле `\insertbiblioauthorimportant`.
        %  * ни на что не влияют в противном случае
        \nocite{vakbib2}%vak
        \nocite{patbib1}%patent
        \nocite{progbib1}%program
        \nocite{bib1}%other
        \nocite{confbib1}%conf
    \end{refsection}%
        %
        % Всё, что вне этих двух refsection, это refsection=0,
        %  * для диссертации - это нормальные ссылки, попадающие в обычную библиографию
        %  * для автореферата:
        %     * при usefootcite==0, ссылка корректно сработает только для источника из `external.bib`. Для своих работ --- напечатает "[0]" (и даже Warning не вылезет).
        %     * при usefootcite==1, ссылка сработает нормально. В авторской библиографии будут только процитированные в refsection=0 работы.
}
% При использовании пакета \verb!biblatex! будут подсчитаны все работы, добавленные
% в файл \verb!biblio/author.bib!. Для правильного подсчёта работ в~различных
% системах цитирования требуется использовать поля:
% \begin{itemize}
%         \item \texttt{authorvak} если публикация индексирована ВАК,
%         \item \texttt{authorscopus} если публикация индексирована Scopus,
%         \item \texttt{authorwos} если публикация индексирована Web of Science,
%         \item \texttt{authorconf} для докладов конференций,
%         \item \texttt{authorpatent} для патентов,
%         \item \texttt{authorprogram} для зарегистрированных программ для ЭВМ,
%         \item \texttt{authorother} для других публикаций.
% \end{itemize}
% Для подсчёта используются счётчики:
% \begin{itemize}
%         \item \texttt{citeauthorvak} для работ, индексируемых ВАК,
%         \item \texttt{citeauthorscopus} для работ, индексируемых Scopus,
%         \item \texttt{citeauthorwos} для работ, индексируемых Web of Science,
%         \item \texttt{citeauthorvakscopuswos} для работ, индексируемых одной из трёх баз,
%         \item \texttt{citeauthorscopuswos} для работ, индексируемых Scopus или Web of~Science,
%         \item \texttt{citeauthorconf} для докладов на конференциях,
%         \item \texttt{citeauthorother} для остальных работ,
%         \item \texttt{citeauthorpatent} для патентов,
%         \item \texttt{citeauthorprogram} для зарегистрированных программ для ЭВМ,
%         \item \texttt{citeauthor} для суммарного количества работ.
% \end{itemize}
% % Счётчик \texttt{citeexternal} используется для подсчёта процитированных публикаций;
% % \texttt{citeregistered} "--- для подсчёта суммарного количества патентов и программ для ЭВМ.

% Для добавления в список публикаций автора работ, которые не были процитированы в
% автореферате, требуется их~перечислить с использованием команды \verb!\nocite! в
% \verb!Synopsis/content.tex!.


{\struct}
В введении рассказывается об актуальности проблемы, в чем научная новизна и цель проекта.

Во первой главе показан обзор существующих решений.

Вторая глава покрывает разработку объекта исследования, а именно решение задачи топологического синтеза и инженерную разработку прототипа.

Третья глава посвящена разработке и исследованию самодельного преобразователя силы на основе Velostat.

Четвертая глава раскрывает детали создания алгоритма построения карты с помощью тактильного очувствления, решение проблемы локализации с помощью датчиков датчиков силы, IMU и маяков. Решатся проблема определения типа поверхности.