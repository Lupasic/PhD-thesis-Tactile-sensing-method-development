{\actuality} Движение по пещере часто происходит по опасными и труднопроходимыми участкам. Наиболее опасными являются сифоны \pic{fig:surface_types/syphon}, сталактиты, сталагмиты, обилие скользких грунтов \pic{fig:surface_types/ice, fig:surface_types/moss, fig:surface_types/clay}. В пещерах недостаток света, часто влажно. Встречаются участки, покрытые водой \pic{fig:surface_types/splash} и растительностью \pic{fig:surface_types/moss}.

\begin{figure}[H]
  \begin{subfigure}[b]{0.3\textwidth}
      \centering\includegraphics[height=2.8cm,width=1\textwidth,keepaspectratio]{surface_types/salt.jpg}\\
      \caption{Соляные отложения}
      \label{fig:surface_types/salt}
  \end{subfigure}
  \hfill
  \begin{subfigure}[b]{0.3\textwidth}
      \centering\includegraphics[height=2.8cm,width=1\textwidth,keepaspectratio]{surface_types/siphon.png}\\
      \caption{Сифон}
      \label{fig:surface_types/syphon}
  \end{subfigure}
  \hfill
  \begin{subfigure}[b]{0.3\textwidth}
      \centering\includegraphics[height=2.8cm,width=1\textwidth,keepaspectratio]{surface_types/ice.png}\\
      \caption{Ледяная пещера}
      \label{fig:surface_types/ice}
  \end{subfigure}

  \begin{subfigure}[b]{0.3\textwidth}
      \centering
      \begin{tikzpicture}
          % Include the image in a node
          \node [above right, inner sep=0] (image) at (0,0)
          {\centering\includegraphics[height=2.8cm,width=1\textwidth,keepaspectratio]{surface_types/clay.jpg}};
          % Create scope with normalized axes
          \begin{scope}[
                  x={($ 0.1*(image.south east)$)},
                  y={($ 0.1*(image.north west)$)}]
              % Grid and axes' labels
              % \draw[lightgray,step=1] (image.south west) grid (image.north east);
              % \foreach \x in {0,1,...,10} { \node [below] at (\x,0) {\x}; }
              % \foreach \y in {0,1,...,10} { \node [left] at (0,\y) {\y};}
              % Labels
              \draw[stealth-, very thick,green] (6,8) -- ++(1,1)
              node[rounded corners=3pt,right,black,fill=white]{\tiny Человек};
          \end{scope}
      \end{tikzpicture}
      \caption{Глина}
      \label{fig:surface_types/clay}
  \end{subfigure}
  \hfill
  \begin{subfigure}[b]{0.3\textwidth}
      \centering
      \begin{tikzpicture}
          % Include the image in a node
          \node [above right, inner sep=0] (image) at (0,0)
          {\centering\includegraphics[height=2.8cm,width=1\textwidth,keepaspectratio]{surface_types/splash.png}};
          % Create scope with normalized axes
          \begin{scope}[
                  x={($ 0.1*(image.south east)$)},
                  y={($ 0.1*(image.north west)$)}]
              % Grid and axes' labels
              % \draw[lightgray,step=1] (image.south west) grid (image.north east);
              % \foreach \x in {0,1,...,10} { \node [below] at (\x,0) {\x}; }
              % \foreach \y in {0,1,...,10} { \node [left] at (0,\y) {\y};}

              % Labels
              \draw[stealth-, very thick,green] (5,2) -- ++(-2,+1)
              node[rounded corners=3pt,left,black,fill=white]{\tiny Лужа};
          \end{scope}
      \end{tikzpicture}
      \caption{Пещера, заполненная водой по~колено}
      \label{fig:surface_types/splash}
  \end{subfigure}
  \hfill
  \begin{subfigure}[b]{0.3\textwidth}
      \centering\includegraphics[height=2.8cm,width=1\textwidth,keepaspectratio]{surface_types/moss.jpg}\\
      \caption{Мох}
      \label{fig:surface_types/moss}
  \end{subfigure}
  \caption{Препятствия, встречающиеся в пещерах}\label{fig:obstacles}
\end{figure}


Эти препятствия могут встретиться человеком при исследовании или инспекции пещеры. Одно из преимуществ роботов --- они могут работать в опасных средах без нахождения рядом человека. Таким образом использование роботов в пещерах нивелирует все опасности для человека.

Существуют различные типы движителей роботов. С препятствиями представленными выше лучше всего справляются многоногие шагающие роботы. Такие роботы могут проходить по сыпучим грунтам, каменистым грядам и преодолевать небольшие водные преграды.

Для полноценного функционирования в пещере необходимы сенсоры. Внешними сенсорами являются камера и лидар.

Характерные для пещеры условия могут вывевсти из строя сенсоры. К примеру грязь \pic{fig:surface_types/clay} может закрыть обзор камере или лидару. Или водная гладь \pic{fig:surface_types/splash} будет отражать лучи лазера лидара и искажать данные \pic{fig:unsolvable_case}.

      \begin{figure}[ht!]
      \begin{subfigure}[t]{0.3\textwidth}
        \centering\includegraphics[height=3cm,width=1\textwidth,keepaspectratio]{terrain_wo_water.png}
        \caption{Территория без воды}
    \end{subfigure}
          \begin{subfigure}[t]{0.35\textwidth}
              \centering
              \begin{tikzpicture}
                  % Include the image in a node
                  \node [above right, inner sep=0] (image) at (0,0)
                  {\centering\includegraphics[height=3.5cm,width=1\textwidth,keepaspectratio]{terrain_w_water1.png}};
                  % Create scope with normalized axes
                  \begin{scope}[
                          x={($ 0.1*(image.south east)$)},
                          y={($ 0.1*(image.north west)$)}]
                      % Grid and axes' labels
                      \draw[stealth-, very thick,green] (6,8) -- ++(2,1)
                      node[rounded corners=3pt,right,black,fill=white]{\tiny Вода};

                      \draw[stealth-, very thick,green] (0.5,5.5) -- (3,2);
                      \draw[stealth-, very thick,green] (2.5,4.2) -- (3,2);
                      \draw[stealth-, very thick,green] (4.5,4) -- (3,2)
                      node[rounded corners=3pt,below,black,fill=white]{\tiny Данные с лидара};
                  \end{scope}
              \end{tikzpicture}
              \caption{Территория с водой}
              \label{fig:terrain_w_water1.png}
          \end{subfigure}
          \begin{subfigure}[t]{0.3\textwidth}
              \centering\includegraphics[height=3cm,width=1\textwidth,keepaspectratio]{terrain_w_water_camera.png}
              \caption{Изображение с камеры}
          \end{subfigure}
          \caption{Пример ситуации, где навигация, основанная на камере или лидаре построит неправильную карту}
          \label{fig:unsolvable_case}
      \end{figure}
  % \legend{Подрисуночный текст, описывающий обозначения, например. Согласно
  %     ГОСТ 2.105, пункт 4.3.1, располагается перед наименованием рисунка.}

% \ifsynopsis
% Этот абзац появляется только в~автореферате.
% Для формирования блоков, которые будут обрабатываться только в~автореферате,
% заведена проверка условия \verb!\!\verb!ifsynopsis!.
% Значение условия задаётся в~основном файле документа (\verb!synopsis.tex! для
% автореферата).
% \else
% Этот абзац появляется только в~диссертации.
% Через проверку условия \verb!\!\verb!ifsynopsis!, задаваемого в~основном файле
% документа (\verb!dissertation.tex! для диссертации), можно сделать новую
% команду, обеспечивающую появление цитаты в~диссертации, но~не~в~автореферате.
% \fi

{\aim} работы является разработка и исследование робототехнической системы построения карты местности и определения геометрических и физических свойств опорной поверхности на базе многоногого шагающего аппарата с тактильным очувствлением без использования оптических сенсоров.

Данное решение отлично подходит для первичного исследования замкнутых труднодоступных пространств, где отсутствует освещение, присутствует обилие грязи, пыли, а так же водных препятствий. Алгоритмы и концепты навигации данной системы могут быть использованы как резервная система навигации для других робототехнических систем, когда более точная --- оптическая вышла из строя.

Для~достижения поставленной цели решаются следующие {\tasks}:
\begin{enumerate}[beginpenalty=10000] % https://tex.stackexchange.com/a/476052/104425
    \item разрабатока метода оптимизации конструкции многоногих шагающих роботов с цикловыми движителями с одной степенью свободы по критериям проходимости (длина робота), детализации (количества ног), пройденного пути;
    \item создание метода исследования датчика силы, когда площадь контакта нажатия на сенсор меньше чувствительной области самого сенсора;
  \item  проектирование метода построения карты местности и определения поверхности с помощью тактильного очувствления;
  \item реализация алгоритма, позволяющего определять геометрические и физические свойства опорной поверхности.
\end{enumerate}

{\researchobj}
Объектом исследования является класс многоногих шагающих роботов с цельным или сочленённым корпусом, и цикловыми движителями с одной степенью свободы, управляемые зависимо или независимо друг от друга.

\begin{figure}[H]
  \centering\includegraphics[height=7cm,width=1\textwidth,keepaspectratio]{strirus_2.jpg}
  \caption{Прототип, на котором было сделано большинство экспериментов}
  \label{fig:strirus_2.jpgg}
\end{figure}

Основная часть экспериментальных исследований проведена с прототипом \pic{fig:strirus_2.jpgg}, корпус которого состоит из двух сегментов с одной активной степенью свободы. Робот обладает 12 независимыми педипуляторами, 6 ног в первом сегменте и 6 во втором.

Особенность конструкции робота в том, что возможно изменять угол между ногой и корпусом робота. Данное конструктивное изменение позволило сделать перемещение робота всенаправленным, то есть робот может двигаться во все стороны без смены ориентации корпуса робота.


{\methods} За основу были взяты методологии из теории по разработке робототехнических систем, теоретической механики, механизмов и машин, теории оптимизации.

Для экспериментального исследования применялось численное и стендовое моделирования.

{\reliability} Правдивость результатов обеспечивается согласованностью с опубликованными результатами научных исследований других авторов, подтверждаются результатами компьютерного моделирования, натурными испытаниями. Результаты диссертационного исследования докладывались и обсуждались на российских и международных научных конференциях, и получили положительный отзыв научной общественности.


{\novelty} Сформулирована и решена задача построения карты местности с помощью тактильного очувствления шагающего робота с цикловыми движителями и датчиками силы, установленными на опорных поверхностях движителей.
Разработан метод оптимизации конструкции многоногого шагающего робота с цикловыми движителями. 
Представлен метод автоматизированного исследования датчика силы.


\textbf{Доказана} возможность построения карты местности и определения типа поверхности с помощью тактильного очувствления как в робототехническом симуляторе, так с помощью натурного эксперимента.

\textbf{Показано}, что оптимальное количество ног для циклового движителя с одной степенью свободы в ноге находится в диапазоне от 8 до 14 ног. 

\textbf{Предложено} использовать преобразователь силы на основе полимерного материала Velostat. \textbf{Установлено}, что данный преобразователь можно использовать для изначальной задачи, то есть при площади контакта с поверхностью большей, чем 25\% площади сенсора. 

\textbf{Сделан вывод} об эффективности предложенных методик, на основе результатов натурных испытаний.

{\defpositions}
\begin{enumerate}[beginpenalty=10000] % https://tex.stackexchange.com/a/476052/104425
  \item метод оптимизации конструкции многоногих шагающих роботов с цикловыми движителями с одной степенью свободы по критериям проходимости (длина робота), детализации (количества ног), пройденного пути;
  \item метод исследования датчика силы, когда площадь соприкосновения меньше площади сенсора;
  \item алгоритм, позволяющий определять тип поверхности;
  \item метод построения карты местности с помощью датчиков силы, установленных на ногах робота.
\end{enumerate}


{\influence} Реализация полученных результатов в виде продукта (реализованное на промышленном оборудовании со всеми практиками, связанными с промышленной разработкой) позволит получать информацию о типе пройденной поверхности, а так же строить карту поверхности под небольшим слоем воды (лужа), там где лидар и камера не смогут выдать адекватный результат.


{\probation}
Основные результаты работы докладывались~на:
\begin{itemize}
  \item ICINCO 2017 --- 14th International Conference on Informatics in Control, Automation and Robotics (Madrid, Spain, 26-28 july 2017);
  \item IEEE International Conference on Robotics and Biomimetics, ROBIO 2017 (Macau, China, 5-8 december 2017);
  \item  международной  научно-практической  конференции  «Прогресс  транспортных 
  средств и систем» (г. Волгоград, 9-11 октября 2018 г.);
  \item 23rd IEEE FRUCT Conference (Bologna, Italy, 13-16 november 2018).
  \item XXXI международной конференции молодых ученых и студентов МИКМУС-2019 
  (г. Москва, 4-6 декабря 2019 г.);
  \item Международная конференция <<Зимняя Школа Робототехники в Сириусе --- 2022>> (г. Адлер, Россия, 25 января - 6 февраля 2022)
\end{itemize}

{\contribution} Все научные результаты диссертации, выдвигаемые для защиты, получены автором лично.

{%%% Реализация пакетом biblatex через движок biber
\begin{refsection}[bl-author, bl-registered]
    % Это refsection=1.
    % Процитированные здесь работы:
    %  * подсчитываются, для автоматического составления фразы "Основные результаты ..."
    %  * попадают в авторскую библиографию, при usefootcite==0 и стиле `\insertbiblioauthor` или `\insertbiblioauthorgrouped`
    %  * нумеруются там в зависимости от порядка команд `\printbibliography` в этом разделе.
    %  * при использовании `\insertbiblioauthorgrouped`, порядок команд `\printbibliography` в нём должен быть тем же (см. biblio/biblatex.tex)
    %
    % Невидимый библиографический список для подсчёта количества публикаций:

    \nocite{*}

    \printbibliography[heading=nobibheading, section=1, env=countauthorvak,          keyword=biblioauthorvak]%
    \printbibliography[heading=nobibheading, section=1, env=countauthorwos,          keyword=biblioauthorwos]%
    \printbibliography[heading=nobibheading, section=1, env=countauthorscopus,       keyword=biblioauthorscopus]%
    \printbibliography[heading=nobibheading, section=1, env=countauthorconf,         keyword=biblioauthorconf]%
    \printbibliography[heading=nobibheading, section=1, env=countauthorother,        keyword=biblioauthorother]%
    \printbibliography[heading=nobibheading, section=1, env=countregistered,         keyword=biblioregistered]%
    \printbibliography[heading=nobibheading, section=1, env=countauthorpatent,       keyword=biblioauthorpatent]%
    \printbibliography[heading=nobibheading, section=1, env=countauthorprogram,      keyword=biblioauthorprogram]%
    \printbibliography[heading=nobibheading, section=1, env=countauthor,             keyword=biblioauthor]%
    \printbibliography[heading=nobibheading, section=1, env=countauthorvakscopuswos, filter=vakscopuswos]%
    \printbibliography[heading=nobibheading, section=1, env=countauthorscopuswos,    filter=scopuswos]%
    %
    %
    %
    {\publications} Основные результаты по теме диссертации изложены в~\arabic{citeauthor}~печатных изданиях,
    \arabic{citeauthorvak} из которых изданы в журналах, рекомендованных ВАК\sloppy%
    \ifnum \value{citeauthorscopuswos}>0%
        , \arabic{citeauthorscopuswos} "--- в~периодических научных журналах, индексируемых Web of~Science и Scopus\sloppy%
    \fi%
    \ifnum \value{citeauthorconf}>0%
        , \arabic{citeauthorconf} "--- в~тезисах докладов.
    \else%
        .
    \fi%
    \ifnum \value{citeregistered}=1%
        \ifnum \value{citeauthorpatent}=1%
            Зарегистрирован \arabic{citeauthorpatent} патент.
        \fi%
        \ifnum \value{citeauthorprogram}=1%
            Зарегистрирована \arabic{citeauthorprogram} программа для ЭВМ.
        \fi%
    \fi%
    \ifnum \value{citeregistered}>1%
        Зарегистрированы\ %
        \ifnum \value{citeauthorpatent}>0%
        \formbytotal{citeauthorpatent}{патент}{}{а}{}\sloppy%
        \ifnum \value{citeauthorprogram}=0 . \else \ и~\fi%
        \fi%
        \ifnum \value{citeauthorprogram}>0%
        \formbytotal{citeauthorprogram}{программ}{а}{ы}{} для ЭВМ.
        \fi%
    \fi%
    % К публикациям, в которых излагаются основные научные результаты диссертации на соискание учёной
    % степени, в рецензируемых изданиях приравниваются патенты на изобретения, патенты (свидетельства) на
    % полезную модель, патенты на промышленный образец, патенты на селекционные достижения, свидетельства
    % на программу для электронных вычислительных машин, базу данных, топологию интегральных микросхем,
    % зарегистрированные в установленном порядке.(в ред. Постановления Правительства РФ от 21.04.2016 N 335)
\end{refsection}%
% \begin{refsection}[bl-author, bl-registered]
%     % Это refsection=2.
%     % Процитированные здесь работы:
%     %  * попадают в авторскую библиографию, при usefootcite==0 и стиле `\insertbiblioauthorimportant`.
%     %  * ни на что не влияют в противном случае
%     % \nocite{vakbib2}%vak
%     % \nocite{patbib1}%patent
%     % \nocite{progbib1}%program
%     % \nocite{bib1}%other
%     % \nocite{confbib1}%conf
% \end{refsection}%
    %
    % Всё, что вне этих двух refsection, это refsection=0,
    %  * для диссертации - это нормальные ссылки, попадающие в обычную библиографию
    %  * для автореферата:
    %     * при usefootcite==0, ссылка корректно сработает только для источника из `external.bib`. Для своих работ --- напечатает "[0]" (и даже Warning не вылезет).
    %     * при usefootcite==1, ссылка сработает нормально. В авторской библиографии будут только процитированные в refsection=0 работы.
}

Диссертационная работа была выполнена при поддержке грантов:
\begin{itemize}
    \item НТИ по поддержке Центра <<Технологий Компонентов Робототехники и Мехатроники>> на базе Университета Иннополис по теме <<Разработка роботизированных платформ для автономной подземной и наземной инспекции местности в условиях трудной проходимости и плохой видимости>>. 
    \item РФФИ № 20-38-90265 по теме <<Разработка метода очувствления мобильного шагающего робота, перемещающегося в закрытом пространстве естественного происхождения>>.
\end{itemize}

% Объем и структура находятся в файле introduction