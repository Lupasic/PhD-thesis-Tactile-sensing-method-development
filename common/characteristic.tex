
{\actuality} Жизнь многих народов Земли неразрывно связана с пещерами. В них можно найти множество полезных ресурсов, такие как металлы, драгоценные камни редкие разновидности мхов.

Но изучение пещер всегда сопровождается большими опасностями и
трудностями. К примеру таковыми являются сифоны \pic{fig:siphon}, сталактиты, обилие скользких
грунтов \pic{fig:ice}. В пещерах недостаток света и тепла, часто влажно.

Одно из преимуществ роботов --- они могут работать в опасных средах без нахождения рядом человека. Таким образом использование роботов в пещерах нивелирует все опасности для человека.

Для полноценного функционирования робота в пещере необходимы сенсоры. Классическими внешними сенсорами являются камера и лидар.

В пещерах с обилием минеральных и органических загрязнений, такие как помет летучих мышей, глина затруднительно использовать только классические способы очувствления (лидары, камеры), более того, недостаток света нивелирует преимущество классических камер из-за невозможности использования алгоритмов, где требуется большое количество фич. Типичная проблема --- грязь \pic{fig:clay} может закрыть обзор камере. Или водяной пар и водная гладь \pic{fig:splash} будут отражаться от лазера лидара и искажать полученные данные.

\begin{figure}[ht]
  \centerfloat{
      \hfill
      \subcaptionbox[List-of-Figures entry]{Сифон\label{fig:siphon}}{%
          \includegraphics[width=0.24\linewidth]{surface_types/siphon.png}}
      \hfill
      \subcaptionbox{Пещера, заполненная водой по~колено\label{fig:splash}}{%
      \includegraphics[width=0.24\linewidth]{surface_types/splash.png}}
      \hfill
      \subcaptionbox{Лед в пещерах\label{fig:ice}}{%
      \includegraphics[width=0.24\linewidth]{surface_types/ice.png}}
      \hfill
      \subcaptionbox{Глина\label{fig:clay}}{%
          \includegraphics[width=0.24\linewidth]{surface_types/clay.jpg}}
      \hfill
  }
  % \legend{Подрисуночный текст, описывающий обозначения, например. Согласно
  %     ГОСТ 2.105, пункт 4.3.1, располагается перед наименованием рисунка.}
  \caption[Этот текст попадает в названия рисунков в списке рисунков]{Препятствия, встречающиеся в пещерах}\label{fig:obstacles}
\end{figure}
\vspace{-0.5cm}
Эту проблему возможно частично решить с помощью другого набора сенсоров. С помощью внутренних сенсоров, на которые меньше влияют внешние факторы. Такими сенсорами являются Инерциальное Измерительное устройство (IMU), датчики силы и маяки.

Таким образом необходимо разработать методику получения полезной информации об окружающей среде с помощью данного набора датчиков. Так как основные данные получаются с помощью касаний, это называется тактильным очувствлением.

Подтверждением актуальности проблемы могут являться DARPA Subterranean Challenge, отчет, который опубликовала команда Яндекс Беспилотники, а так же выделенные средства фондов НТИ и РФФИ на решение этой проблемы.

% \ifsynopsis
% Этот абзац появляется только в~автореферате.
% Для формирования блоков, которые будут обрабатываться только в~автореферате,
% заведена проверка условия \verb!\!\verb!ifsynopsis!.
% Значение условия задаётся в~основном файле документа (\verb!synopsis.tex! для
% автореферата).
% \else
% Этот абзац появляется только в~диссертации.
% Через проверку условия \verb!\!\verb!ifsynopsis!, задаваемого в~основном файле
% документа (\verb!dissertation.tex! для диссертации), можно сделать новую
% команду, обеспечивающую появление цитаты в~диссертации, но~не~в~автореферате.
% \fi

{\aim} данной работы является разработка метода тактильного очувствления мобильного шагающего робота в закрытых пространствах естественного или искусственного происхождения, в которых невозможно получение данных со спутниковой навигации, затруднено применение оптических сенсоров.

Это необходимо для проблемы получения достоверных данных о поверхности, когда робот передвигается по лужам и классические внешние сенсоры выдают некорректные данные.

Для~достижения поставленной цели необходимо было решить следующие {\tasks}:
\begin{enumerate}[beginpenalty=10000] % https://tex.stackexchange.com/a/476052/104425
  \item спроектировать объект исследования --- шагающего многоногого робота: подобрать количество ног, их форму. Обосновать конструкцию тела и количество степеней свободы;
  \item подобрать сенсоры для решения поставленной задачи;
  \item разработать методику построения карты местности и определения типа поверхности с помощью тактильного очувствления;
  \item решить проблему локализации на основе тактильного очувствления.
\end{enumerate}

{\researchobj}
Для разведки местности под землей в труднодоступных местах с малой видимостью разработан прототип многоногого шагающего робота СтриРус \pic{fig:strirus_4}

Данный робот состоит из двух сегментов с одной активной степенью свободы. Робот обладает 10 независимыми лапками, 6 лап в первом сегменте и 4 во втором.

Особенность конструкции робота в том, что возможно изменять угол между лапкой и корпусом робота. Данное конструктивное изменение позволило сделать перемещение робота всенаправленным, то есть робот может двигаться во все стороны без смены ориентации корпуса робота.


{\methods} За базис были взяты методологии из теории робототехнических систем, теоретической механики, механизмов и машин, теории оптимизации.

Для экспериментального исследования применялось численное и стендовое моделирования.

{\reliability} Правдивость результатов обеспечивается согласованностью с опубликованными результатами научных исследований других авторов, подтверждаются результатами компьютерного моделирования, натурными испытаниями. Результаты диссертационного исследования докладывались и обсуждались на российских и международных научных конференциях, и получили положительный отзыв научной общественности.


{\novelty} состоит в формулировании проблемы построения карты с помощью тактильного очувствления, а так же ее комплексного решения. В это решение входит разработка метода оптимизации конструкции робота, методика исследования датчика силы, а так же методика построения карты местности с помощью датчиков силы, установленных на ногах робота.

\textbf{Доказана} возможность построения карты местности и определения типа поверхности с помощью тактильного очувствления как в робототехническом симуляторе, так с помощью натурного эксперимента.

\textbf{Показано}, что оптимальное количество ног для циклового движителя с одной степенью свободы в ноге находится в диапазоне от 8 до 14 ног. 

\textbf{Предложено} использовать преобразователь силы на основе полимерного материала Velostat. \textbf{Установлено}, что данный преобразователь можно рассматривать как единое тело, при площади нажатия больше 50\% площади сенсора. 

\textbf{Сделан вывод} об эффективности предложенных методик, на основе результатов натурных испытаний.

{\defpositions}
\begin{enumerate}[beginpenalty=10000] % https://tex.stackexchange.com/a/476052/104425
  \item метод оптимизации конструкции многоногих роботов;
  \item разработанная методика исследования датчика силы, когда площадь нажатия на сенсор меньше самого сенсора;
  \item реализация программно-алгоритмического обеспечения (ПАО), позволяющего определять тип поверхности;
  \item методика построения карты местности с помощью датчиков силы, установленных на ногах робота.
\end{enumerate}


{\influence} Реализация полученных результатов в виде продукта позволит получить надежную резервную систему навигации для робота, у которого есть датчики силы, IMU и возможность устанавливать маяки по мере продвижения.

Если у робота есть только датчики силы, будет возможно получать информацию о типе пройденной поверхности, а так же строить карту поверхности под толщей воды, там где лидар и камера не смогут выдать адекватный результат.


{\probation}
Основные результаты работы докладывались~на:
\begin{itemize}
  \item ICINCO 2017 --- 14th International Conference on Informatics in Control, Automation and Robotics (Madrid, Spain, 26-28 july 2017);
  \item IEEE International Conference on Robotics and Biomimetics, ROBIO 2017 (Macau, China, 5-8 december 2017);
  \item  международной  научно-практической  конференции  «Прогресс  транспортных 
  средств и систем» (г. Волгоград, 9-11 октября 2018 г.);
  \item 23rd IEEE FRUCT Conference (Bologna, Italy, 13-16 november 2018).
  \item XXXI международной конференции молодых ученых и студентов МИКМУС-2019 
  (г. Москва, 4-6 декабря 2019 г.);
  \item Международная конференция <<Зимняя Школа Робототехники в Сириусе --- 2022>> (г. Адлер, Россия, 25 января - 6 февраля 2022)
\end{itemize}

{\contribution} Все научные результаты диссертации, выдвигаемые для защиты, получены автором лично.

% Вставка кто сколько опубликовался
\input{common/fancybibcalc.tex}

Диссертационная работа была выполнена при поддержке грантов:
\begin{itemize}
    \item НТИ по поддержке Центра <<Технологий Компонентов Робототехники и Мехатроники>> на базе Университета Иннополис по теме <<Разработка роботизированных платформ для автономной подземной и наземной инспекции местности в условиях трудной проходимости и плохой видимости>>. 
    \item РФФИ № 20-38-90265 по теме <<Разработка метода очувствления мобильного шагающего робота, перемещающегося в закрытом пространстве естественного происхождения>>.
\end{itemize}

{\struct}
В введении рассказывается об актуальности проблемы, в чем научная новизна и цель проекта.

Во первой главе показан обзор существующих решений.

Вторая глава покрывает разработку объекта исследования, а именно решение задачи топологического синтеза и инженерную разработку прототипа.

Третья глава посвящена разработке и исследованию самодельного преобразователя силы на основе Velostat.

Четвертая глава раскрывает детали создания алгоритма построения карты с помощью тактильного очувствления, решение проблемы локализации с помощью датчиков датчиков силы, IMU и маяков. Решатся проблема определения типа поверхности.