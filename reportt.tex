%&preformat-synopsis
\RequirePackage[l2tabu,orthodox]{nag} % Раскомментировав, можно в логе получать рекомендации относительно правильного использования пакетов и предупреждения об устаревших и нерекомендуемых пакетах

% Откомментируйте, чтобы отключить генерацию закладок в pdf
% \PassOptionsToPackage{bookmarks=false}{hyperref}
\documentclass[a4paper,14pt,oneside,openany,article]{memoir} %,draft

\input{common/setup}          % общие настройки шаблона
\input{common/packages}       % Пакеты общие для диссертации и автореферата
\synopsistrue                 % Этот документ --- автореферат
\input{Synopsis/synpackages}  % Пакеты для автореферата
%%% Микротипографика %%%
%\ifnumequal{\value{draft}}{0}{% Только если у нас режим чистовика
%    \usepackage[final]{microtype}[2016/05/14] % улучшает представление букв и слов в строках, может помочь при наличии отдельно висящих слов
%}{}

% \usepackage[skip=2pt]{caption}
% % will apply to all subcaptions
% \usepackage[skip=2pt]{subcaption}

% \setlength{\abovecaptionskip}{1pt} 
% \setlength{\belowcaptionskip}{1pt}  % Пакеты для специфических пользовательских задач

\input{common/newnames}       % Новые переменные, которые могут использоваться во всём проекте
\input{Synopsis/setup}        % Упрощённые настройки шаблона

%%% Основные сведения %%%
\newcommand{\thesisAuthorLastName}{\fixme{Буличев}}
\newcommand{\thesisAuthorOtherNames}{\fixme{Олег Викторович}}
\newcommand{\thesisAuthorInitials}{\fixme{О.\,В.}}
\newcommand{\thesisAuthor}             % Диссертация, ФИО автора
{%
    \texorpdfstring{% \texorpdfstring takes two arguments and uses the first for (La)TeX and the second for pdf
        \thesisAuthorLastName~\thesisAuthorOtherNames% так будет отображаться на титульном листе или в тексте, где будет использоваться переменная
    }{%
        \thesisAuthorLastName, \thesisAuthorOtherNames% эта запись для свойств pdf-файла. В таком виде, если pdf будет обработан программами для сбора библиографических сведений, будет правильно представлена фамилия.
    }
}
\newcommand{\thesisAuthorShort}        % Диссертация, ФИО автора инициалами
{\thesisAuthorInitials~\thesisAuthorLastName}
%\newcommand{\thesisUdk}                % Диссертация, УДК
%{\fixme{xxx.xxx}}
\newcommand{\thesisTitle}              % Диссертация, название
{\fixme{Разработка метода тактильного очувствления для мобильного шагающего робота}}
\newcommand{\thesisSpecialtyNumber}    % Диссертация, специальность, номер
{\fixme{2.5.4.}}
\newcommand{\thesisSpecialtyTitle}     % Диссертация, специальность, название (название взято с сайта ВАК для примера)
{\fixme{Роботы, мехатроника и робототехнические системы}}
%% \newcommand{\thesisSpecialtyTwoNumber} % Диссертация, вторая специальность, номер
%% {\fixme{XX.XX.XX}}
%% \newcommand{\thesisSpecialtyTwoTitle}  % Диссертация, вторая специальность, название
%% {\fixme{Теория и~методика физического воспитания, спортивной тренировки,
%% оздоровительной и~адаптивной физической культуры}}
\newcommand{\thesisDegree}             % Диссертация, ученая степень
{\fixme{кандидата технических наук}}
\newcommand{\thesisDegreeShort}        % Диссертация, ученая степень, краткая запись
{\fixme{канд. тех. наук}}
\newcommand{\thesisCity}               % Диссертация, город написания диссертации
{\fixme{Волгоград}}
\newcommand{\thesisYear}               % Диссертация, год написания диссертации
{\the\year}
\newcommand{\thesisOrganization}       % Диссертация, организация
{\fixme{Автономная некоммерческая организация высшего образования \\ <<Университет Иннополис>>}}
\newcommand{\thesisOrganizationShort}  % Диссертация, краткое название организации для доклада
{\fixme{АНО ВО "Университет Иннополис"}}

\newcommand{\thesisInOrganization}     % Диссертация, организация в предложном падеже: Работа выполнена в ...
{\fixme{Университете Иннополис}}

%% \newcommand{\supervisorDead}{}           % Рисовать рамку вокруг фамилии
\newcommand{\supervisorFio}              % Научный руководитель, ФИО
{\fixme{Малолетов Александр Васильевич}}
\newcommand{\supervisorRegalia}          % Научный руководитель, регалии
{\fixme{Доктор физико-математических наук, доцент}}
\newcommand{\supervisorFioShort}         % Научный руководитель, ФИО
{\fixme{А.\,В.~Малолетов}}
\newcommand{\supervisorRegaliaShort}     % Научный руководитель, регалии
{\fixme{д.ф.-м.н.}}

%% \newcommand{\supervisorTwoDead}{}        % Рисовать рамку вокруг фамилии
%% \newcommand{\supervisorTwoFio}           % Второй научный руководитель, ФИО
%% {\fixme{Фамилия Имя Отчество}}
%% \newcommand{\supervisorTwoRegalia}       % Второй научный руководитель, регалии
%% {\fixme{уч. степень, уч. звание}}
%% \newcommand{\supervisorTwoFioShort}      % Второй научный руководитель, ФИО
%% {\fixme{И.\,О.~Фамилия}}
%% \newcommand{\supervisorTwoRegaliaShort}  % Второй научный руководитель, регалии
%% {\fixme{уч.~ст.,~уч.~зв.}}

\newcommand{\opponentOneFio}           % Оппонент 2, ФИО
{\fixme{Яцун Сергей Федорович}}
\newcommand{\opponentOneRegalia}       % Оппонент 2, регалии
{\fixme{доктор технических наук, профессор}}
\newcommand{\opponentOneJobPlace}      % Оппонент 2, место работы
{\fixme{Федеральное государственное бюджетное образовательное учреждение высшего образования <<Юго-Западный государственный университет>>}}
\newcommand{\opponentOneJobPost}       % Оппонент 2, должность
{\fixme{Заведующий кафедрой}}

\newcommand{\opponentTwoFio}           % Оппонент 1, ФИО
{\fixme{Воротников Сергей Анатольевич}}
\newcommand{\opponentTwoRegalia}       % Оппонент 1, регалии
{\fixme{кандидат технических наук, доцент}}
\newcommand{\opponentTwoJobPlace}      % Оппонент 1, место работы
{\fixme{Федеральное государственное бюджетное образовательное учреждение высшего образования <<Московский государственный технический университет имени Н. Э. Баумана>>}}
\newcommand{\opponentTwoJobPost}       % Оппонент 1, должность
{\fixme{Доцент}}



\newcommand{\leadingOrganizationTitle} % Ведущая организация, дополнительные строки. Удалить, чтобы не отображать в автореферате
{\fixme{Федеральнοе гοсударственнοе автономное οбразοвательнοе учреждение высшегο οбразοвания <<Южный федеральный университет>>, г. Ростов-на-Дону}}

\newcommand{\defenseDate}              % Защита, дата
{\fixme{22 декабря 2022~г.~в~10 часов}}
\newcommand{\defenseCouncilNumber}     % Защита, номер диссертационного совета
{\fixme{Д\,24.2.282.07}}
\newcommand{\defenseCouncilTitle}      % Защита, учреждение диссертационного совета
{\fixme{ФГБОУ  ВО  <<Волгоградский государственный технический университет>>}}
\newcommand{\defenseCouncilAddress}    % Защита, адрес учреждение диссертационного совета
{\fixme{400005, г. Волгоград, проспект им. В. И. Ленина, 28, ауд. 209}}
\newcommand{\defenseCouncilPhone}      % Телефон для справок
{\fixme{+7~(8442) 24-81-83}}

\newcommand{\defenseSecretaryFio}      % Секретарь диссертационного совета, ФИО
{\fixme{Попов Андрей Васильевич}}
\newcommand{\defenseSecretaryRegalia}  % Секретарь диссертационного совета, регалии
{\fixme{канд. техн. наук}}            % Для сокращений есть ГОСТы, например: ГОСТ Р 7.0.12-2011 + http://base.garant.ru/179724/#block_30000

\newcommand{\synopsisLibrary}          % Автореферат, название библиотеки
{\fixme{НФГБОУ  ВО  <<Волгоградский государственный технический университет>> и на сайте \url{www.vstu.ru} по ссылке \url{https://www.vstu.ru/upload/iblock/138/13831bd6fa27285e079e5019d5f1d275.pdf}}}
\newcommand{\synopsisDate}             % Автореферат, дата рассылки
{<< \rule{1cm}{0.15mm} >> \rule{3cm}{0.15mm} 2023~года}

% To avoid conflict with beamer class use \providecommand
\providecommand{\keywords}%            % Ключевые слова для метаданных PDF диссертации и автореферата
{}
           % Основные сведения
\input{common/fonts}          % Определение шрифтов (частичное)
\input{common/styles}         % Стили общие для диссертации и автореферата
\input{Synopsis/synstyles}    % Стили для автореферата
\input{Synopsis/userstyles}   % Стили для специфических пользовательских задач

%%% Библиография. Выбор движка для реализации %%%
\ifnumequal{\value{bibliosel}}{0}{%
    \input{biblio/predefined} % Встроенная реализация с загрузкой файла через движок bibtex8
}{
    \input{biblio/biblatex}   % Реализация пакетом biblatex через движок biber
}

\setcounter{draft}{0}

% Вывести информацию о выбранных опциях в лог сборки
\typeout{Selected options:}
\typeout{Draft mode: \arabic{draft}}
\typeout{Font: \arabic{fontfamily}}
\typeout{AltFont: \arabic{usealtfont}}
\typeout{Bibliography backend: \arabic{bibliosel}}
\typeout{Precompile images: \arabic{imgprecompile}}


% Вывести информацию о версиях используемых библиотек в лог сборки
% \listfiles

\begin{document}
% \includepdf[pages=-]{Titles_inno/synopsis_title.pdf}
% \includepdf[pages=-,fitpaper]{Titles_inno/synopsis_title.pdf}
% \input{Synopsis/title}       % Титульный лист
%\mainmatter                   % В том числе начинает нумерацию страниц арабскими цифрами с единицы
\mainmatter*                  % Нумерация страниц не изменится, но начнётся с новой страницы
\setcounter{page}{2}

\pdfbookmark{Общая характеристика работы}{characteristic}             % Закладка pdf
\section*{РЕФЕРАТ}
\addcontentsline{toc}{chapter}{ВВЕДЕНИЕ}
Отчет на 32 стр.\\
ШАГАЮЩИЙ РОБОТ, КАРСТОВЫЕ ПЕЩЕРЫ, ТАКТИЛЬНЫЕ СЕНСОРЫ, ОЧУВСТВЛЕНИЕ, ОПРЕДЕЛЕНИЕ ТИПА ТЕРРИТОРИИ, SLAM \\
\newcommand{\actuality}{\pdfbookmark[1]{Актуальность}{actuality}\underline{\textbf{\actualityTXT}}}
\newcommand{\progress}{\pdfbookmark[1]{Разработанность темы}{progress}\underline{\textbf{\progressTXT}}}
\newcommand{\aim}{\pdfbookmark[1]{Цели}{aim}\underline{{\textbf\aimTXT}}}
\newcommand{\tasks}{\pdfbookmark[1]{Задачи}{tasks}\underline{\textbf{\tasksTXT}}}
\newcommand{\aimtasks}{\pdfbookmark[1]{Цели и задачи}{aimtasks}\aimtasksTXT}
\newcommand{\novelty}{\pdfbookmark[1]{Научная новизна}{novelty}\underline{\textbf{\noveltyTXT}}}
\newcommand{\influence}{\pdfbookmark[1]{Практическая значимость}{influence}\underline{\textbf{\influenceTXT}}}
\newcommand{\methods}{\pdfbookmark[1]{Методология и методы исследования}{methods}\underline{\textbf{\methodsTXT}}}
\newcommand{\defpositions}{\pdfbookmark[1]{Положения, выносимые на защиту}{defpositions}\underline{\textbf{\defpositionsTXT}}}
\newcommand{\reliability}{\pdfbookmark[1]{Достоверность}{reliability}\underline{\textbf{\reliabilityTXT}}}
\newcommand{\probation}{\pdfbookmark[1]{Апробация}{probation}\underline{\textbf{\probationTXT}}}
\newcommand{\contribution}{\pdfbookmark[1]{Личный вклад}{contribution}\underline{\textbf{\contributionTXT}}}
\newcommand{\publications}{\pdfbookmark[1]{Публикации}{publications}\underline{\textbf{\publicationsTXT}}}

\newcommand{\struct}{\pdfbookmark[1]{Структура работы}{struct}\underline{\textbf{\structTXT}}}
\newcommand{\researchobj}{\pdfbookmark[1]{Объект исследования}{researchobj}\underline{\textbf{\researchobjTXT}}}
% \setcounter{page}{1}
{\actuality} Возможные направления применения мобильных роботов включают в себя использование их для исследовательских целей в труднодоступных условиях. Мобильные роботы могут проникать в места, недоступные и опасные для людей, например, в пещеры и шахты, перемещаться под завалами или внутри помещений во время стихийных бедствий, аварийных ситуаций и так далее. 
Одним из наиболее интересных и малоизученных направлений является разработка мобильных роботов, предназначенных для движения в условиях пещер естественного происхождения. 

Движение по пещере часто происходит по опасным и труднопроходимым участкам. Наиболее опасными являются сифоны \pic{fig:surface_types/syphon}, сталактиты, сталагмиты, обилие скользких грунтов \pic{fig:surface_types/ice, fig:surface_types/moss, fig:surface_types/clay}. В пещерах недостаток света, часто влажно. Встречаются участки, покрытые водой \pic{fig:surface_types/splash} и растительностью \pic{fig:surface_types/moss}.

\begin{figure}[H]
  \begin{subfigure}[b]{0.3\textwidth}
      \centering\includegraphics[height=2.8cm,width=1\textwidth,keepaspectratio]{surface_types/salt.jpg}\\
      \caption{Соляные отложения}
      \label{fig:surface_types/salt}
  \end{subfigure}
  \hfill
  \begin{subfigure}[b]{0.3\textwidth}
      \centering\includegraphics[height=2.8cm,width=1\textwidth,keepaspectratio]{surface_types/siphon.png}\\
      \caption{Сифон}
      \label{fig:surface_types/syphon}
  \end{subfigure}
  \hfill
  \begin{subfigure}[b]{0.3\textwidth}
      \centering\includegraphics[height=2.8cm,width=1\textwidth,keepaspectratio]{surface_types/ice.png}\\
      \caption{Ледяная пещера}
      \label{fig:surface_types/ice}
  \end{subfigure}

  \begin{subfigure}[b]{0.3\textwidth}
      \centering
      \begin{tikzpicture}
          % Include the image in a node
          \node [above right, inner sep=0] (image) at (0,0)
          {\centering\includegraphics[height=2.8cm,width=1\textwidth,keepaspectratio]{surface_types/clay.jpg}};
          % Create scope with normalized axes
          \begin{scope}[
                  x={($ 0.1*(image.south east)$)},
                  y={($ 0.1*(image.north west)$)}]
              % Grid and axes' labels
              % \draw[lightgray,step=1] (image.south west) grid (image.north east);
              % \foreach \x in {0,1,...,10} { \node [below] at (\x,0) {\x}; }
              % \foreach \y in {0,1,...,10} { \node [left] at (0,\y) {\y};}
              % Labels
              \draw[stealth-, very thick,green] (6,8) -- ++(1,1)
              node[rounded corners=3pt,right,black,fill=white]{\tiny Человек};
          \end{scope}
      \end{tikzpicture}
      \caption{Глина}
      \label{fig:surface_types/clay}
  \end{subfigure}
  \hfill
  \begin{subfigure}[b]{0.3\textwidth}
      \centering
      \begin{tikzpicture}
          % Include the image in a node
          \node [above right, inner sep=0] (image) at (0,0)
          {\centering\includegraphics[height=2.8cm,width=1\textwidth,keepaspectratio]{surface_types/splash.png}};
          % Create scope with normalized axes
          \begin{scope}[
                  x={($ 0.1*(image.south east)$)},
                  y={($ 0.1*(image.north west)$)}]
              % Grid and axes' labels
              % \draw[lightgray,step=1] (image.south west) grid (image.north east);
              % \foreach \x in {0,1,...,10} { \node [below] at (\x,0) {\x}; }
              % \foreach \y in {0,1,...,10} { \node [left] at (0,\y) {\y};}

              % Labels
              \draw[stealth-, very thick,green] (5,2) -- ++(-2,+1)
              node[rounded corners=3pt,left,black,fill=white]{\tiny Лужа};
          \end{scope}
      \end{tikzpicture}
      \caption{Пещера, заполненная водой по~колено}
      \label{fig:surface_types/splash}
  \end{subfigure}
  \hfill
  \begin{subfigure}[b]{0.3\textwidth}
      \centering\includegraphics[height=2.8cm,width=1\textwidth,keepaspectratio]{surface_types/moss.jpg}\\
      \caption{Мох}
      \label{fig:surface_types/moss}
  \end{subfigure}
  \caption{Препятствия, встречающиеся в пещерах}\label{fig:obstacles}
\end{figure}


Эти препятствия могут встретиться человеком при исследовании или инспекции пещеры. Одно из преимуществ роботов --- они могут работать в опасных средах без нахождения рядом человека. Таким образом использование роботов в пещерах нивелирует все опасности для человека.

Существуют различные типы движителей роботов. С препятствиями представленными выше лучше всего справляются многоногие шагающие роботы. Такие роботы могут проходить по сыпучим грунтам, каменистым грядам и преодолевать небольшие водные преграды.

Для полноценного функционирования любого мобильного робота необходимы сенсоры. Сенсоры мобильных роботов можно разделить на внешние и внутренние. Под внешними сенсорами подразумеваются устройства, элементы которых не могут быть защищены от воздействия окружающей среды. Примерами таких сенсоров являются камеры, лидары, сонары и тому подобные устройства.

Внутренние сенсоры включают в себя датчики усилий, акселерометры, магнитометры, амперметры и так далее. Такие устройства предполагают взаимодействие с внешней средой посредством гравитационных или магнитных полей, или механических элементов, и могут быть механически защищены от неблагоприятного воздействия окружающей среды.

Большую опасность для мобильных роботов представляет тот факт, что характерные для пещеры условия могут вывести из строя сенсоры. К примеру грязь \pic{fig:surface_types/clay} может закрыть обзор камере или лидару. Или водная гладь \pic{fig:surface_types/splash} будет отражать лучи лазера лидара и искажать данные \pic{fig:unsolvable_case}.

      \begin{figure}[ht!]
      \begin{subfigure}[t]{0.3\textwidth}
        \centering\includegraphics[height=3cm,width=1\textwidth,keepaspectratio]{terrain_wo_water.png}
        \caption{Территория без воды}
    \end{subfigure}
          \begin{subfigure}[t]{0.35\textwidth}
              \centering
              \begin{tikzpicture}
                  % Include the image in a node
                  \node [above right, inner sep=0] (image) at (0,0)
                  {\centering\includegraphics[height=3.5cm,width=1\textwidth,keepaspectratio]{terrain_w_water1.png}};
                  % Create scope with normalized axes
                  \begin{scope}[
                          x={($ 0.1*(image.south east)$)},
                          y={($ 0.1*(image.north west)$)}]
                      % Grid and axes' labels
                      \draw[stealth-, very thick,green] (6,8) -- ++(2,1)
                      node[rounded corners=3pt,right,black,fill=white]{\tiny Вода};

                      \draw[stealth-, very thick,green] (0.5,5.5) -- (3,2);
                      \draw[stealth-, very thick,green] (2.5,4.2) -- (3,2);
                      \draw[stealth-, very thick,green] (4.5,4) -- (3,2)
                      node[rounded corners=3pt,below,black,fill=white]{\tiny Данные с лидара};
                  \end{scope}
              \end{tikzpicture}
              \caption{Территория с водой}
              \label{fig:terrain_w_water1.png}
          \end{subfigure}
          \begin{subfigure}[t]{0.3\textwidth}
              \centering\includegraphics[height=3cm,width=1\textwidth,keepaspectratio]{terrain_w_water_camera.png}
              \caption{Изображение с камеры}
          \end{subfigure}
          \caption{Примеры ситуаций, где навигация, основанная на камере или лидаре построит неправильную карту}
          \label{fig:unsolvable_case}
      \end{figure}
  % \legend{Подрисуночный текст, описывающий обозначения, например. Согласно
  %     ГОСТ 2.105, пункт 4.3.1, располагается перед наименованием рисунка.}

% \ifsynopsis
% Этот абзац появляется только в~автореферате.
% Для формирования блоков, которые будут обрабатываться только в~автореферате,
% заведена проверка условия \verb!\!\verb!ifsynopsis!.
% Значение условия задаётся в~основном файле документа (\verb!synopsis.tex! для
% автореферата).
% \else
% Этот абзац появляется только в~диссертации.
% Через проверку условия \verb!\!\verb!ifsynopsis!, задаваемого в~основном файле
% документа (\verb!dissertation.tex! для диссертации), можно сделать новую
% команду, обеспечивающую появление цитаты в~диссертации, но~не~в~автореферате.
% \fi

{\aim} является разработка метода построения карты местности с определением геометрических и физико-механических свойств опорной поверхности роботом с шагающими движителями снабженными тактильными датчиками.

Предлагаемое решение подходит для первичного исследования замкнутых труднодоступных пространств, где отсутствует освещение, присутствует обилие грязи, пыли, а так же водных препятствий. Алгоритмы и концепты навигации данной системы могут быть использованы как резервная система навигации для других робототехнических систем, когда более точная --- оптическая вышла из строя.

Для~достижения поставленной цели решаются следующие {\tasks}:
\begin{enumerate}[beginpenalty=10000] % https://tex.stackexchange.com/a/476052/104425
    \item Определение профиля опорной поверхности, на основе информации о точках её касания ногами робота и внутренних датчиков, характеризующих механическое состояние аппарата.
    \item Определение физико-механических свойств опорной поверхности: жесткости, вязкости и пластичности, и выделение на их основе классов поверхностей на основе информации с датчиков силы, установленных на ногах и внутренних датчиков робота.
    \item Исследование влияния на точность измерения усилий площади пятна контакта при нажатии на сенсор.
    \item Изучение влияния геометрических параметров робота на точность и полноту физико-механических свойств опорной поверхности и профильную проходимость робота.
\end{enumerate}

{\researchobj}
Объектом исследования является класс многоногих шагающих роботов с цельным или сочленённым корпусом, и цикловыми движителями с одной степенью свободы, управляемые зависимо или независимо друг от друга.

\begin{figure}[H]
    \centerfloat{
        \hfill
        \subcaptionbox[List-of-Figures entry]{Первая итерация\label{fig:strirus_0}}{%
            \includegraphics[width=0.33\linewidth]{strirus_0.png}}
        \hfill
        \subcaptionbox[List-of-Figures entry]{Вторая итерация \label{fig:strirus_1}}{%
            \includegraphics[width=0.33\linewidth]{strirus_1.png}}
        \hfill
        \subcaptionbox{Третья итерация\label{fig:strirus_2}}{%
        \includegraphics[width=0.33\linewidth]{strirus_2.jpg}}
        \hfill
        \subcaptionbox{Третья итерация +\label{fig:strirus_3}}{%
        \includegraphics[width=0.35\linewidth]{strirus_3.JPG}}
        \hfill
        \subcaptionbox{Четвертая итерация\label{fig:strirus_4}}{%
        \includegraphics[width=0.35\linewidth]{strirus_4.png}}
    }
    \caption{Итерации разработанного робота СтриРус}\label{fig:striruses}
  \end{figure}

Итерации отличаются от друг друга количеством ног (от 54ех, до 6и), количества приводов (2 --- кол-во ног) , количеством сочленений сегментов корпуса (0 --- 1 с 2умя степенями свободы), высотой ног (54 --- 170 мм), возможностью изменять угол между корпусом робота и осью вала привода ноги (только 90 --- 90 - 45). 

Последняя итерация \pic{fig:strirus_4} обладает 10 независимыми ногами, 6 ног в первом сегменте, 4 во втором. 1 сочленение с 1 активной степенью свободы. Высота ноги --- 170 мм. Возможность изменять угол между корпусом и осью вала дискретно на 90 и 75 градусов. Данное конструктивное изменение позволило сделать перемещение робота всенаправленным, то есть робот может двигаться во все стороны без смены ориентации корпуса робота.


{\methods} За основу были взяты методологии из теории по разработке робототехнических систем, теоретической механики, механизмов и машин, теории оптимизации.

Для экспериментального исследования применялось численное и стендовое моделирования.

{\reliability} Правдивость результатов обеспечивается согласованностью с опубликованными результатами научных исследований других авторов, подтверждаются результатами компьютерного моделирования, натурными испытаниями. Результаты диссертационного исследования докладывались и обсуждались на российских и международных научных конференциях, и получили положительный отзыв научной общественности.


{\novelty} 
\begin{enumerate}
    \item Реализован метод построения карты местности, состоящий в определении геометрической формы поверхности с помощью тактильного очувствления, который позволяет решать задачу определения плана и профиля поверхности в условиях отсутствия видимости и при движении по поверхности, находящейся под водой. \textbf{Доказана} возможность построения карты местности с помощью тактильного очувствления, как в робототехническом симуляторе, так и с помощью натурного эксперимента.
    \item Реализован метод определения физико-механических свойств опорной поверхности на основе тактильного очувствления. \textbf{Показана} возможность различать материалы с упругими, жёсткими, пластичными свойствами.
    \item \textbf{Установлено} то, что датчик силы, на основе полимерного материала, обеспечивает погрешность определения силы не более 10\% при условии площади пятна контакта не менее 25\% от размера датчика, что позволяет применять датчик такого типа для тактильного очувствления мобильного робота. \textbf{Предложена} методика роботизированного исследования датчика силы.
    \item Предложен аддитивно-мультипликативный критерий оптимизации кинематической схемы многоногих шагающих роботов с цикловыми одностепенными движителями, включающий в себя показатели проходимости и покрытия опорной поверхности. На основании которого определено оптимальное количество ног для циклового движителя с одной степенью свободы.
\end{enumerate}

\textbf{Сделан вывод} об эффективности предложенных методов и методик, на основе результатов натурных испытаний.

{\defpositions}
\begin{enumerate}[beginpenalty=10000] % https://tex.stackexchange.com/a/476052/104425
    \item Метод построения карты местности, состоящий в определении геометрической формы поверхности с помощью тактильного очувствления, который позволяет решать задачу определения плана и профиля поверхности в условиях отсутствия видимости и при движении по поверхности, находящейся под водой.
    \item Метод определения физико-механических свойств опорной поверхности на основе тактильного очувствления, позволяющий различать материалы с упругими, жёсткими, пластичными свойствами.
    \item Зависимость погрешности датчика силы на основе полимерного материла от площади пятна контакта относительно размеров датчика, применяемого для тактильного очувствления мобильного робота. Методика роботизированного исследования датчика силы.
    \item Критерий оптимизации кинематической схемы многоногих шагающих роботов с цикловыми одностепенными движителями, включающий в себя показатели проходимости, покрытия опорной поверхности и её детализации. Определение на его основе габаритов и количества движителей шагающего робота.
\end{enumerate}

{\influence} Реализация полученных результатов позволит разрабатывать мобильных шагающих роботов, способных перемещаться без использования оптических сенсоров или в условиях невозможности их использования, обеспечивая построение карты местности с определением типа и свойств опорной поверхности за счёт очувствления механизмов шагания робота. 

Такие роботы могут быть востребованы для исследования естественных пещер, объектов антропогенного происхождения в условиях, когда локализация робота с помощь камер или лидаров невозможна из-за отсутствия света, наличия пыли, дыма или иных факторов делающих невозможным применение оптических сенсорных систем.

{\probation}
Основные положения диссертации доложены и обсуждены на конференциях:
\begin{itemize}
  \item ICINCO 2017 — 14th International Conference on Informatics in Control, Automation and Robotics (Мадрид, Испания, 26-28 июля 2017);
  \item IEEE International Conference on Robotics and Biomimetics, ROBIO 2017 (Макао, Китай, 5-8 декабря 2017);
  \item  Международная научно-практическая конференция «Прогресс транспортных средств и систем» (г. Волгоград, 9-11 октября 2018 г.);
  \item 23rd IEEE FRUCT Conference (Болонья, Италия, 13-16 ноября 2018);
  \item XXXI международная конференция молодых ученых и студентов МИКМУС-2019 (Москва, 4-6 декабря 2019 г.);
  \item Международная конференция «Зимняя Школа Робототехники в Сириусе — 2022» (Адлер, Россия, 25 января - 6 февраля 2022).
\end{itemize}

{\contribution} Все научные результаты диссертации, выдвигаемые для защиты, получены автором лично. Автор самостоятельно проводил анализ литературы по теме, участвовал в обсуждении постановки цели диссертации, лично планировал и проводил компьютерные эксперименты и физические эксперименты, спроектировал и собрал экспериментальные установки. Автор лично получил все представленные в работе численные данные. 

{%%% Реализация пакетом biblatex через движок biber
\begin{refsection}[bl-author, bl-registered]
    % Это refsection=1.
    % Процитированные здесь работы:
    %  * подсчитываются, для автоматического составления фразы "Основные результаты ..."
    %  * попадают в авторскую библиографию, при usefootcite==0 и стиле `\insertbiblioauthor` или `\insertbiblioauthorgrouped`
    %  * нумеруются там в зависимости от порядка команд `\printbibliography` в этом разделе.
    %  * при использовании `\insertbiblioauthorgrouped`, порядок команд `\printbibliography` в нём должен быть тем же (см. biblio/biblatex.tex)
    %
    % Невидимый библиографический список для подсчёта количества публикаций:

    \nocite{*}

    \printbibliography[heading=nobibheading, section=1, env=countauthorvak,          keyword=biblioauthorvak]%
    \printbibliography[heading=nobibheading, section=1, env=countauthorwos,          keyword=biblioauthorwos]%
    \printbibliography[heading=nobibheading, section=1, env=countauthorscopus,       keyword=biblioauthorscopus]%
    \printbibliography[heading=nobibheading, section=1, env=countauthorconf,         keyword=biblioauthorconf]%
    \printbibliography[heading=nobibheading, section=1, env=countauthorother,        keyword=biblioauthorother]%
    \printbibliography[heading=nobibheading, section=1, env=countregistered,         keyword=biblioregistered]%
    \printbibliography[heading=nobibheading, section=1, env=countauthorpatent,       keyword=biblioauthorpatent]%
    \printbibliography[heading=nobibheading, section=1, env=countauthorprogram,      keyword=biblioauthorprogram]%
    \printbibliography[heading=nobibheading, section=1, env=countauthor,             keyword=biblioauthor]%
    \printbibliography[heading=nobibheading, section=1, env=countauthorvakscopuswos, filter=vakscopuswos]%
    \printbibliography[heading=nobibheading, section=1, env=countauthorscopuswos,    filter=scopuswos]%
    %
    %
    %
    {\publications} Основные результаты по теме диссертации изложены в~\arabic{citeauthor}~печатных изданиях,
    \arabic{citeauthorvak} из которых изданы в журналах, рекомендованных ВАК\sloppy%
    \ifnum \value{citeauthorscopuswos}>0%
        , \arabic{citeauthorscopuswos} "--- в~периодических научных журналах, индексируемых Web of~Science и Scopus\sloppy%
    \fi%
    \ifnum \value{citeauthorconf}>0%
        , \arabic{citeauthorconf} "--- в~тезисах докладов.
    \else%
        .
    \fi%
    \ifnum \value{citeregistered}=1%
        \ifnum \value{citeauthorpatent}=1%
            Зарегистрирован \arabic{citeauthorpatent} патент.
        \fi%
        \ifnum \value{citeauthorprogram}=1%
            Зарегистрирована \arabic{citeauthorprogram} программа для ЭВМ.
        \fi%
    \fi%
    \ifnum \value{citeregistered}>1%
        , зарегистрированы\ %
        \ifnum \value{citeauthorpatent}>0%
        \formbytotal{citeauthorpatent}{патент}{}{а}{}\sloppy%
        \ifnum \value{citeauthorprogram}=0 . \else \ и~\fi%
        \fi%
        \ifnum \value{citeauthorprogram}>0%
        \formbytotal{citeauthorprogram}{программ}{а}{ы}{} для ЭВМ.
        \fi%
    \fi%
    % К публикациям, в которых излагаются основные научные результаты диссертации на соискание учёной
    % степени, в рецензируемых изданиях приравниваются патенты на изобретения, патенты (свидетельства) на
    % полезную модель, патенты на промышленный образец, патенты на селекционные достижения, свидетельства
    % на программу для электронных вычислительных машин, базу данных, топологию интегральных микросхем,
    % зарегистрированные в установленном порядке.(в ред. Постановления Правительства РФ от 21.04.2016 N 335)
\end{refsection}%
% \begin{refsection}[bl-author, bl-registered]
%     % Это refsection=2.
%     % Процитированные здесь работы:
%     %  * попадают в авторскую библиографию, при usefootcite==0 и стиле `\insertbiblioauthorimportant`.
%     %  * ни на что не влияют в противном случае
%     % \nocite{vakbib2}%vak
%     % \nocite{patbib1}%patent
%     % \nocite{progbib1}%program
%     % \nocite{bib1}%other
%     % \nocite{confbib1}%conf
% \end{refsection}%
    %
    % Всё, что вне этих двух refsection, это refsection=0,
    %  * для диссертации - это нормальные ссылки, попадающие в обычную библиографию
    %  * для автореферата:
    %     * при usefootcite==0, ссылка корректно сработает только для источника из `external.bib`. Для своих работ --- напечатает "[0]" (и даже Warning не вылезет).
    %     * при usefootcite==1, ссылка сработает нормально. В авторской библиографии будут только процитированные в refsection=0 работы.
}

Диссертационная работа была выполнена при поддержке грантов:
\begin{itemize}
    \item НТИ по поддержке Центра <<Технологий Компонентов Робототехники и Мехатроники>> на базе Университета Иннополис по теме <<Разработка роботизированных платформ для автономной подземной и наземной инспекции местности в условиях трудной проходимости и плохой видимости>>. 
    \item РФФИ № 20-38-90265 по теме <<Разработка метода очувствления мобильного шагающего робота, перемещающегося в закрытом пространстве естественного происхождения>>.
\end{itemize}

% Объем и структура находятся в файле introduction % Характеристика работы по структуре во введении и в автореферате не отличается (ГОСТ Р 7.0.11, пункты 5.3.1 и 9.2.1), потому её загружаем из одного и того же внешнего файла, предварительно задав форму выделения некоторым параметрам


\tableofcontents
\pdfbookmark{Содержание работы}{description}                          % Закладка pdf
\section*{Содержание работы}
Основанием для проведения работ является заключенный с Федеральным государственным бюджетным учреждением «Российский фонд фундаментальных исследований» договор №20-38-90265\textbackslash 20 от 28.08.2020 г. о предоставлении гранта победителю конкурса и реализации научного проекта «Разработка метода очувствления мобильного шагающего робота, перемещающегося в закрытом пространстве естественного происхождения».


\textbf{\underline{Во первой главе}} показан обзор существующих решений. Рассмотрены 3 глобальных темы: типы препятствий, которые могут встретиться; роботы, которые используются в исследованиях пещер; а так же методы построения карты местности.

Для решения поставленной цели необходимо понимать в каких условиях будет использоваться робот. Основные структуры поверхностей следующие \pic{fig:obstacles}:
\begin{itemize}
    \item твердые породы, прочные -- мрамор, кварц, базальт (магма);
    \item твердые породы, мягкие -- мел, гипс, соль, известняк;
    \item сыпучие грунты -- песок, глина, снег;
    \item водные преграды -- как и лужи (малый слой воды), так и целы залы, погруженные под воду. Часто встречаются сифоны;
    \item скользкие поверхности -- отложения мха и плесени, лед ;
    \item разрушаемые поверхности -- каменная гряда, паутина.
\end{itemize}



Так же были рассмотрены размеры пещер, чтобы понимать необходимый запас хода, размеры робототехнического комплекса.

В диссертации рассматривались роботы которые создавались специально для исследований пещер, в том числе и на Марсе. А так же те, которые потенциально могут быть использованы в условиях, определенных выше.

Как итог, их можно классифицировать следующим образом. Наземные роботы это шагающие, колесные, трековые и необычные. К необычным включены змеевидные, шарообразные и другие.

К летающим были отнесены защищенные дроны и дирижабли.

Продуктовым решением является робототехническая система, включающая в себя несколько роботов одного типа или комбинацию наземного и летающего роботов.

Были рассмотрены классические SLAM алгоритмы, основанные на использовании камеры, стереопары, с использованием лидара, GPS, IMU а так же их различные комбинации.

Были найдены способы получения облака точек объекта с помощью касания манипулятором данного объекта. Примерное местоположение объекта определялось камерой.

Определить тип поверхности можно так же с помощью различных сенсоров: визуально, IMU, с помощью снятия тока с моторов, момента с вала мотора, с помощью датчиков силы, установленных на конечность робота.

Были найдены следующие предложенные решения:
\begin{itemize}
    \item робототехнические системы для исследования свободных пещер;
    \item Построение карты с помощью лидаров и камер;
    \item Получение конечно элементной сетки с помощью тактильного очувствления манипулятором.
\end{itemize}

Поставленная задача является новой и не встречается в научных публикациях российских и зарубежных авторов.


\subsection{Решение задачи топологического
синтеза}


\subsection{Инженерная
часть}


\subsubsection{Итерация
1}

Про первого стремного стрируса. который не дожил даже до финалочки


\subsubsection{Итерация
2}

12и ного стрируса, котоырй был собран, но были найдены конструктивные
недостатки


\subsubsection{Итерация
3}

Про текущего, разработанного стрируса

Взять статьи, которые писал до этого


\subsection{Сенсор
Velostat}

Существует несколько типов датчиков, которые могут измерять контактные
силы и распределение давления. Это могут быть оптические,
пьезорезистивные, пьезоэлектрические, магнитные, емкостные, на основе
оптических волокон. Промышленные датчики силы и момента (F/T) широко
распространены на гуманоидах (Atlas, Fedor) или четвероногих (Spot,
AnyMal). Однако они слишком велики для небольших роботов, таких как
RHEX, WHEGS или StriRus. Та же проблема применима к оптическим и
магнитным датчикам. Емкостные датчики требуют высокой точности
изготовления. Кроме того, датчики перечисленных типов довольно дороги,
что делает их использование нецелесообразным в исследовательских
роботах, которые работают в опасных условиях и могут быть потеряны в
процессе исследования пещеры. Недорогой альтернативой являются
тензометрические датчики.

Самый популярный тип тензометрического датчика - тензорезистивный
датчик. Другой способ - использовать пьезорезистивные датчики на основе
проводящих волокон или полимеров. Они недорогие, очень гибкие и
компактные. Одним из основных недостатков является значительный
гистерезис. В представленной работе используется материал Velostat
(Linqstat) в качестве промежуточного слоя для резистивного датчика.

При исследовании преобразователя силы на основе Velostat, было замечено,
что площадь нажатия разительно влияет на показания преобразователя.
Площадь контакта может сильно повлиять на данные при перемещении роботом
по неровной твердой поверхности, к примеру по камням. Поэтому было
решено характеризовать материал для случаев, когда нагрузка меньше, чем
размер сенсора.


\subsubsection{Конструкция
преобразователя}

Датчик состоит из двух медных оболочек, разделенных слоем Velostat.
Давление на датчик приводит к изменению его сопротивления: чем выше
давление, тем ниже сопротивление. Сопротивление измеряется косвенным
методом. Измеренное сопротивление Velostat образует делитель напряжения
с постоянным резистором R1\ldots R8 (Рисунок 1).

Рисунок 1 --- Электрическая схема преобразователя


\subsection{Разработка экспериментального
стенда}


\subsubsection{Экспериментальный
стенд}

Для выполнения серии экспериментов был разработан роботизированный
экспериментальный стенд. Стенд предназначен для исследования
преобразователя силы. Среди требований к стенду можно отметить:
необходимость контролировать силу нажатия и повторяемость эксперимента
как по величине, так и по расположению и ориентации площадки контакта
инструмента и исследуемого преобразователя силы. Указанным требованиям
возможно удовлетворить, используя коллаборативный робот-манипулятор.
Использование коллаборативного робота позволяет также удовлетворить
требованиям безопасности и допустить работу робота в непосредственно
близости от экспериментатора. Разработанный стенд, представлен на
Рисунке 3.

Рисунок 3 --- Разработанный экспериментальный стенд

Для касания только части объекта исследования были разработаны различные
концевые инструменты. К примеру, на рисунке 5 представлен инструмент для
точечного нажатия. Диаметр площади контакта равен 2 мм

Рисунок 5 --- Инструмент (концевой эффектор) для нажатия объект
исследования с диаметром нажатия меньше, чем сам объект


\subsection{Экспериментальная
часть}

Про 3 эксперимента

\textbf{\underline{Четвертая глава}} раскрывает детали определения профиля опорной поверхности, на основе информации о точках её касания ногами робота и внутренних датчиков, характеризующих механическое состояние аппарата. Вторая часть главы показывает определение физико-математических свойств опорной поверхности:
жесткости, вязкости и пластичности, и выделение на их основе классов поверхностей на основе информации с датчиков силы, установленных на ногах и внутренних датчиков робота.

Традиционно, карта для навигации представляется в виде облака точек. Тогда, без предложенного алгоритма, будут получено очень разреженное облако точек, где точки будут являться точками касания лапок робота с поверхностью.

Сделав предположение, что расстояние между ногами робота мало относительно целой пещеры, мы можем предположить, что поверхность между ногами является плоскостью.

В рамках исследования предполагается, что робот движется по поверхности, у которой каждому набору координат $x,\ y$ соответствует одно и только одно значение координаты $z$.

Был реализован следующий алгоритм. Вначале необходимо очистить оригинальное облако точек от шумов и усреднить близлежащие точки с помощью Voxel grid. Потом из него генерируется полигональная сетка с помощью 2D Триангуляции Делоне \pic{fig:delone_idea.png} (вогнутая оболочка \pic{fig:exp_concave_hull}). На ее основе получается необходимое плотное облако точек \pic{fig:sampled_pcd.png}. 

\begin{figure}[H]
    \centering\includegraphics[height=2.5cm,width=1\textwidth,keepaspectratio]{delone_idea.png}
    \caption{2D Триангуляция Делоне (выпуклая оболочка)}
    \label{fig:delone_idea.png}
\end{figure}

Реализованный алгоритм проверялся, как в симуляции (Рис. \ref{fig:unsolvable_case}, \ref{fig:start_end_exp}), так и на реальном роботе \pic{fig:real_exp_map_creation}. Видео \quad \qrcode[height=1.5cm]{https://youtu.be/2dxHHTG4psQ}


\begin{figure}[H]
    \begin{subfigure}[t]{0.49\textwidth}
        \centering\includegraphics[height=2.5cm,width=1\textwidth,keepaspectratio]{terrain_w_water1.png}
        \caption{Начало движения}
    \end{subfigure}
    \begin{subfigure}[t]{0.49\textwidth}
        \centering\includegraphics[height=2.5cm,width=1\textwidth,keepaspectratio]{terrain_w_water_end.png}
        \caption{Конец движения}
    \end{subfigure}
    \caption{Эксперимент в симуляторе}
    \label{fig:start_end_exp}
\end{figure}

Ниже представлены полученные результаты \pic{fig:result_meshes_blah}. Для оценки точности полученных данных использовались метрики C2C \eqref{eqn:hauff} и C2M \pic{fig:metrics}.

\begin{equation}
    \label{eqn:hauff}
    d_{H}(X,\;Y)=\sup _{m\in M}\left\{\,|\mathrm {dist} _{X}(m)-\mathrm {dist} _{Y}(m)|\,\right\}    
\end{equation}
Где $X,\ Y$ непустые подмножества метрического пространства $M$; $\mathrm {dist} _{X}\colon M\to \mathbb {R}$ $\mathrm {dist} _{X}\colon M\to \mathbb {R}$ обозначает функцию расстояния до множества $X$.



\begin{figure}[h]
    \begin{center}
    \begin{subfigure}[t]{0.4\textwidth}
        \centering\includegraphics[height=3cm,width=1\textwidth,keepaspectratio]{mesh_rviz.png}
        \caption{Полигональная сетка, созданная 2D Триангуляцией Делоне (вогнутая оболочка)}
    \end{subfigure}
    \begin{subfigure}[t]{0.59\textwidth}
        \centering\includegraphics[height=3cm,width=1\textwidth,keepaspectratio]{mesh_comp.png}
        \caption{Наложенные полигональные сетки}
    \end{subfigure}

    \begin{subfigure}[t]{0.9\textwidth}
            \centering
             \begin{tikzpicture}
                % Include the image in a node
                \node [above right, inner sep=0] (image) at (0,0) 
                {\centering\includegraphics[height=3cm,width=1\textwidth,keepaspectratio]{sampled_pcd.png}};          
                % Create scope with normalized axes
                \begin{scope}[
                    x={($ 0.1*(image.south east)$)},
                    y={($ 0.1*(image.north west)$)}]
                    \draw[stealth-, very thick,green] (3,8) -- (2,8.5);
                    \draw[stealth-, very thick,green] (1,5.5) -- (2,8.5)
                    node[rounded corners=3pt,above,black,fill=white]{\tiny Ground Truth Point Cloud};
         
                    \draw[stealth-, very thick,green] (5.5,3) -- (5.5,8.5)
                    node[rounded corners=3pt,above,black,fill=white]{\tiny Generated Point Cloud};
                \end{scope}
            \end{tikzpicture}
            \caption{Наложенные облака точек}
            \label{fig:sampled_pcd.png}
    \end{subfigure}
    \caption{Результат эксперимента}
    \label{fig:result_meshes_blah}
\end{center}
\end{figure}

На рисунке \ref{fig:exp_concave_hull} проиллюстрирована  важность модификации триангуляции Делоне. Как можно заметить \pic{fig:conv_convex.png} алгоритм построил карту местности там, где робот не ходил и стоит стена. При использовании вогнутой оболочки \pic{fig:conv_concave.png} данная проблема не наблюдается.

\begin{figure}[h]
    \begin{subfigure}[t]{0.3\textwidth}
        \centering\includegraphics[height=4cm,width=1\textwidth,keepaspectratio]{convex_terr.png}
        \caption{Пример поля}
        \label{fig:convex_terr.png}
    \end{subfigure}
    \hfill
    \begin{subfigure}[t]{0.33\textwidth}
            \centering
             \begin{tikzpicture}
                % Include the image in a node
                \node [above right, inner sep=0] (image) at (0,0) 
                {\centering\includegraphics[height=4cm,width=1\textwidth,keepaspectratio]{conv_convex.png}};          
                % Create scope with normalized axes
                \begin{scope}[
                    x={($ 0.1*(image.south east)$)},
                    y={($ 0.1*(image.north west)$)}]
                    \draw[stealth-, very thick,green] (5.2,3.5) -- ++(1,-1)
                    node[rounded corners=3pt,right,black,fill=white]{\tiny Generated mesh};
                    
                    \draw[stealth-, very thick,green] (5.5,5.5) -- (7.4,4)
                    node[rounded corners=3pt,right,black,fill=white]{\tiny Lidar data};
                    \draw[stealth-, very thick,green] (3.4,0.8) -- (5,1);            
                    \draw[stealth-, very thick,green] (3.4,2.6) -- (5,1)
                    node[rounded corners=3pt,right,black,fill=white]{\tiny Cloud of contact points};
                \end{scope}
            \end{tikzpicture}
            \caption{Выпуклая оболочка}
            \label{fig:conv_convex.png}
    \end{subfigure}
    \hfill
    \begin{subfigure}[t]{0.30\textwidth}
        \centering\includegraphics[height=4cm,width=1\textwidth,keepaspectratio]{conv_concave.png}
        \caption{Вогнутая оболочка}
        \label{fig:conv_concave.png}
    \end{subfigure}
    \caption{Объяснение необходимости модификации алгоритма Делоне}
    \label{fig:exp_concave_hull}
\end{figure}

\begin{figure}[h]
    \begin{subfigure}[t]{0.49\textwidth}
        \centering\includegraphics[height=4cm,width=1\textwidth,keepaspectratio]{pcd_hist.png}
        \caption{Метрика C2C: гистограмма ошибок (абсолютное расстояние от точки до ближайшей реферальной точки)}
        \label{fig:metric_c2c}
    \end{subfigure}
    \begin{subfigure}[t]{0.49\textwidth}
        \centering\includegraphics[height=5cm,width=1\textwidth,keepaspectratio]{mesh_hist.png}
        \caption{Метрика C2M: Гистограмма ошибок (относительное расстояние от точки до ближайшей реферальной точки)}
        \label{fig:metric_c2m}
    \end{subfigure}
    \caption{Метрики оценки точности полученной карты}
    \label{fig:metrics}
\end{figure}


\begin{figure}[h]
    \begin{subfigure}[t]{0.49\textwidth}
            \centering\includegraphics[height=6cm,width=1\textwidth,keepaspectratio]{real_robot_mesh_video_preview.png}
        \caption{Робот проходит препятствие}
        \label{fig:real_robot_mesh_video_preview.png}
    \end{subfigure}
    \begin{subfigure}[t]{0.49\textwidth}
        \centering\includegraphics[height=6cm,width=1\textwidth,keepaspectratio]{real_mesh.jpg}
        \caption{Полученная полигональная сетка}
        \label{fig:real_mesh.jpg}
    \end{subfigure}
    \caption{Пример натурного эксперимента}
    \label{fig:real_exp_map_creation}
\end{figure}

Как итог, среднеквадратичная ошибка для C2C метрики была в среднем равна 5 см. А для C2M 1 см. В натурном эксперименте среднеквадратичная ошибка по метрике C2C получился 8 см.

Задачу определения типа поверхности можно определить следующим образом. Робот идет по поверхности, и собирает данные с датчиков силы, с момента на моторе и IMU. На основе предварительного обучения, данные обрабатываются и кластеризуются, на основе предварительно определенной базе знаний территорий.

Задачу обучения удобнее всего проводить в лабораторных условиях. Экспериментальная установка соответствует следующим требованиям: возможность установить новые поверхность и сменять их быстро. Это нужно для легкого создания базы знаний поверхностей. Бесконечное движение, для скорости обучения. Узел с ногой должен быть взят с робота, чтобы не пришлось решать похожую задачу на роботе.

Все это было достигнуто благодаря разборному экспериментальному столу и 2ух степенному механизму, который ходит по окружности \pic{fig:s_shape_leg/s_leg_setup.JPG}. Для бесконечного движения пришлось соединить две ноги робота в одну. На рисунке ниже \pic{fig:s_shape_leg/leg_design.png} показаны как установлены сенсоры на получившейся ноге.

\begin{figure}[H]
    \begin{subfigure}[t]{0.48\textwidth}
        \centering
        \begin{tikzpicture}
            % Include the image in a node
            \node [above right, inner sep=0] (image) at (0,0)
            {\centering\includegraphics[height=5cm,width=1\textwidth,keepaspectratio]{s_shape_leg/s_leg_setup.JPG}};
            % Create scope with normalized axes
            \begin{scope}[
                    x={($ 0.1*(image.south east)$)},
                    y={($ 0.1*(image.north west)$)}]
                \draw[stealth-, very thick,green] (3.5,2.5) -- (3,1.5)
                node[rounded corners=3pt,below,black,fill=white]{\tiny Table for surfaces};

                \draw[stealth-, very thick,green] (7.1,5.4) -- (7.4,7)
                node[rounded corners=3pt,above right,black,fill=white]{\tiny Self-made PCB};

                \draw[very thick,green] (6,6.1) rectangle (8.5,3.5)
                node[above left,black,fill=green]{\tiny S leg};
            \end{scope}
        \end{tikzpicture}
        \caption{Общий вид экспериментальной установки}
        \label{fig:s_shape_leg/s_leg_setup.JPG}
    \end{subfigure}
    \begin{subfigure}[t]{0.24\textwidth}
        \centering\includegraphics[height=5cm,width=1\textwidth,keepaspectratio]{s_shape_leg/leg_design.png}
        \caption{Пояснение по расположению сенсоров на ноге робота}
        \label{fig:s_shape_leg/leg_design.png}
    \end{subfigure}
    \begin{subfigure}[t]{0.24\textwidth}
        \centering\includegraphics[height=4cm,width=1\textwidth,keepaspectratio]{s_shape_leg/view.jpg}
        \caption{Каменистая поверхность}
        \label{fig:s_shape_leg/view.jpg}
    \end{subfigure}
    \caption{Экспериментальная установка для определения типа поверхности}
\end{figure}

Были взяты 2 сильно разных поверхности и изучены сырые данные. Резина \pic{fig:s_shape_leg/s_leg_setup.JPG}  \quad \qrcode[height=1.5cm]{https://gifyu.com/image/SxatY} \quad, камень \pic{fig:s_shape_leg/view.jpg} \quad \qrcode[height=1.5cm]{https://gifyu.com/image/Sxatt} 

% \begin{figure}[H]
%     \begin{subfigure}[t]{0.49\textwidth}
%         \centering\includegraphics[height=4cm,width=1\textwidth,keepaspectratio]{s_shape_leg/flat.jpg}
%         \caption{Резина}
%         \label{fig:s_shape_leg/flat.jpg}
%     \end{subfigure}
%     \hfill
%     \begin{subfigure}[t]{0.49\textwidth}
%         \centering\includegraphics[height=4cm,width=1\textwidth,keepaspectratio]{s_shape_leg/view.jpg}
%         \caption{Каменистая поверхность}
%         \label{fig:s_shape_leg/view.jpg}
%     \end{subfigure}
%     \caption{Типы определяемых поверхностей}
% \end{figure}

Ниже \pic{fig:data_from_legs} представлены сырые данные с лапок робота. Сырые данные легко различить, но можно заметить, что абсолютные значения у разных сегментов различно. Поэтому при обучении необходимо их нормализовать.

\begin{figure}[H]
    \begin{subfigure}[t]{0.49\textwidth}
        \centering\includegraphics[height=3.8cm,width=1\textwidth,keepaspectratio]{s_shape_leg/segment8_compare_front.png}
        \caption{Передняя часть ноги, 8ой сегмент}
    \end{subfigure}
    \begin{subfigure}[t]{0.49\textwidth}
        \centering\includegraphics[height=3.8cm,width=1\textwidth,keepaspectratio]{s_shape_leg/segment6_compare_front.png}
        \caption{Передняя часть ноги, 6ой сегмент}
    \end{subfigure}
    \caption{Сравнение сырых данных после эксперимента с разных сегментов ноги}
    \label{fig:data_from_legs}
\end{figure}

Карта местности может быть построена с помощью 2D триангуляции Делоне (вогнутая оболочка). Входными данными для алгоритма является разреженное облако точек касаний робота поверхности. Они получены с помощью преобразователя силы на основе Velostat.

Точность, полученная в симуляторе равна примерно 5 см, а во время натурного эксперимента -- 8 см, что является адекватным результатом для поставленной задачи.

С помощью разработанного преобразователя силы возможно различать 2 типа поверхности: резину и каменистую гряду.
\addcontentsline{toc}{chapter}{ЗАКЛЮЧЕНИЕ}
\section*{\underline{Заключение}}
%% Согласно ГОСТ Р 7.0.11-2011:
%% 5.3.3 В заключении диссертации излагают итоги выполненного исследования, рекомендации, перспективы дальнейшей разработки темы.
%% 9.2.3 В заключении автореферата диссертации излагают итоги данного исследования, рекомендации и перспективы дальнейшей разработки темы.

Основной  научный  результат  диссертации  заключается  в  решении  актуальной 
научной  задачи,  имеющей  важное  практическое  значение: разработка и исследование робототехнической системы построения карты местности и определения геометрических и физических свойств опорной поверхности на базе многоногого шагающего аппарата с тактильным очувствлением без использования оптических сенсоров.

Данное решение отлично подходит для первичного исследования замкнутых труднодоступных пространств, где отсутствует освещение, обилие грязи, пыли, а так же водных препятствий. Алгоритмы и концепты навигации данной системы могут быть использованы как резервная система навигации для других робототехнических систем, когда более точная --- оптическая вышла из строя.


Предложенное решение очень хорошо подходит для изначального изучения закрытых недоступных мест, где отсутствует освещение, много грязи, пыли, и водных преград. Методы и концепты навигации предоставленной системы имеют все шансы быть применены как запасная система навигации для иных робототехнических систем, когда больше точная --- оптическая вышла из строя.

При  проведении  исследований  и  разработок  в  диссертационной  работе  получены 
следующие результаты.
\begin{itemize}
  \item Был проведен обзор и анализ робототехнических систем и условия их применения. То есть была проведена классификация машин, использующих ноги в качестве дивжителя, упор был сделан на машины с циклическими движетилями. Рассмотрены роботы, которые могут быть использованы для исследования пещер. Была предложена их классификация.

  Более того, для понимания условий применений разрабатываемой робототехнической системы, было описаны параметры исследуемых пещер и их особенности.

  Для разработки системы, важной частью которой является сенсорные устройства, был проведен глубокий их обзор и классификая. Так же был проведен литературный обзор алгоритмической части работы с сенсорами, к примеру обзор алгоритмов по триангуляциию.

  Выводом обзора является описание применимости разработанной системы.
  \item Разработан метод оптимизации конструкции многоногих шагающих роботов с цикловыми движителями с одной степенью свободы по критериям проходимости (длина робота), детализации (количества ног), пройденного пути.

  Данный метод основан на применении генетического алгоритма OpenAI-ES, с самописными реализацией скрещивания и мутации. Была разработана математическая модель робота, которая была реализована в GazeboSim. 
  
  Для генерации семейства роботов было предложено геометрическое представление объекта. Так же пришлось разработать способ для генерирования местонсти, которую будет проходить экземляр робота.

  Помимо оптимизации конструкции по предложенным выше критериям, была разработан метод оптимизации конструкции робота для прохождения узких участков. Это важный концепт, так как по обзору пещер стало ясно, что пещеры имеют очень большую девиацию в ширину.
  \item Изучив существующие тактильные сенсоры было решено разработать и исследовать преобразователь силы на основе Velostat. Для этого пришлось физические создать преобразователь, адаптировать его под конкретное применение.

  В течение разработки сенсора были найдена особенность, что при одинаковой силе нажатия на сенсор, возникают различные результаты, в зависимости от места нажатия и площади нажатия. Для исследования данного феномена был разработан автоматизированный экспериментальный стенд. 
  
  По результатам поставленных экспериментов, характеристики преобразователя удовлетворяют требованиям к системе тактильного восприятия шагающего робота, когда ожидаемый размер площади контакта превышает 25 процентов площади преобразователя.
  \item Было разработан метод картографирования с помощью ощупывания поверхности. Для первичной проверки гипотез была разработана сцена в симуляторе CoppeliaSim. После череды экспериментов, было решено использовать и реализовать алгоритмы вогнутой Триангуляции Делоне с использованием альфа формы.

  Так же результат интеллектуальной деятельности проверялся на в натурном эксперименте и была получена точность в 8 см, что является приемлимой для данной задачи.
  \item Для получение максимально полной информации о проходимой поверхности, необходимо знать еще и тип поверхности, по которой проходит робот. Это было реализовано с помощью методов машинного обучения, а все данные были получены из натурных экспериментов. Как результат, точность определения примерно в 85 процентов (3 типа поверхности, земля, резина и камень), позволяет довольно эффективно использовать данные знания для реализации алгоритмов управления.
\end{itemize}

Подведя итог, данная работа имеет как практическую, так и теоретическую составляющую. Предложенные методы апробировались на роботе.

\pdfbookmark{Литература}{bibliography}                                % Закладка pdf
\insertbiblioauthorgrouped

\ifdefmacro{\microtypesetup}{\microtypesetup{protrusion=true}}{}
\urlstyle{tt}        


\end{document}
