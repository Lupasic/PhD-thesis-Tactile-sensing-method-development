\chapter{Разработка и исследование преобразователя силы на основе Velostat}\label{ch:ch3}

Третья глава посвящена разработке и исследованию самодельного преобразователя силы на основе Velostat.

Существует несколько типов датчиков, которые могут измерять контактные силы и распределение давления. Одним из способов его использования является область робототехники. В обнаружении формы поверхности. Существует класс роботов (RHEX, Strirus), которые обладают такими параметрами.

Наиболее плодотворным местом для использования ножных роботов является неровная местность, например, пещеры. Обычная пещера состоит из твердых и скользких поверхностей, ходовых грунтов. Для работы в таких местах робот должен получать информацию о физических свойствах местности. Эта информация оказывает существенное влияние на эффективность и стратегии локомоции. Например, на зернистой или травянистой местности взаимодействие между ногами и землей может привести к резкому рассеиванию энергии из-за трения. Это происходит из-за деформации поверхности ногами. Знание этой информации о взаимодействии ноги с землей может быть использовано для управления адаптацией.

Имея подробную информацию о взаимодействии ноги-земля, мы можем решить множество задач, таких как идентификация местности \cite{wu_integrated_2016, walas_terrain_2015, mrva_feature_2015, dallaire_learning_2015}, управление походкой на основе рельефа местности \cite{wu_tactile_2020, weingarten_automated_2004}, анализ устойчивости и SLAM \cite{odenthal_nonlinear_1999, peters_analysis_2006, saranli_design_2000}. Решение этих задач может значительно повысить проходимость и возможности исследования мобильных роботов.

Для решения задач упоминания могут быть использованы различные датчики. Это могут быть инерциальные измерительные приборы (IMU), ток от двигателей, датчики силы, звук, системы технического зрения \cite{libby_using_2012,ojeda_terrain_2006,peters_analysis_2006}. Ценные результаты были получены при использовании датчиков силы. Следовательно, применение их в нашей системе является правильным решением.

Существует несколько типов датчиков, которые могут измерять контактные силы и распределение давления. Это могут быть оптические, пьезорезистивные, пьезоэлектрические, магнитные, емкостные, на основе оптических волокон \cite{howe_dynamic_1993}. Промышленные датчики момента силы (F/T) широко распространены на гуманоидах (Atlas, Fedor) или четвероногих (Spot, AnyMal). Однако они слишком велики для небольших роботов, таких как RHEX, WHEGS или StriRus \cite{saranli_rhex_2001,schroer_comparing_2004, bulichev_concept_2018}. Та же проблема применима к оптическим и магнитным датчикам. Емкостные датчики требуют высокой точности изготовления. В итоге, пьезорезистивный датчик был выбран как наиболее подходящий.

Самый популярный тип пьезорезистивного датчика - тензодатчик. Он может быть установлен на ногах робота, но это решение требует наличия цепей формирования сигнала и создает трудности при прокладке проводов между постоянно вращающимися ногами и корпусом робота \cite{wu_tactile_2020}. Другой способ - использовать пьезорезистивные датчики на основе проводящих волокон или полимеров. Они недорогие, очень гибкие и компактные. К сожалению, одной из распространенных проблем является значительный гистерезис. Мы решили использовать Velostat (Linqstat)\cite{vehec_flexible_2020} в качестве промежуточного слоя для резистивного датчика.

Velostat - это проводящее волокно, которое обладает вязкоупругим поведением. Это резко влияет на отклик датчика. Этот материал обладает свойствами квантового туннелирования и предварительной локализации.

В результате мы решили разработать и изготовить пьезорезистивный датчик на основе Velostat. Такой датчик поможет нам решить проблемы классификации местности и создания карт на биомиметическом многоножном роботе StriRus. 

Для использования такого датчика необходимо оценить его поведение. Например, мы выяснили, что если прикоснуться к датчику два раза с одинаковой силой в разных местах, то результат на выходе будет отличаться в два раза. Чтобы понять, как с этим работать, нам нужно сформулировать и смоделировать сценарии использования.

В представленной работе используется материал Velostat (Linqstat) \pic{fig:velostat_sensor.jpg} в качестве промежуточного слоя для датчика \pic{fig:simplest_sensor.jpg}.

\begin{figure}[h]
    \begin{subfigure}[t]{0.9\textwidth}
        \centering\includegraphics[height=8cm,width=1\textwidth,keepaspectratio]{velostat_sensor.jpg}
        \caption{Материал Velostat}
        \label{fig:velostat_sensor.jpg}
    \end{subfigure}

    \begin{subfigure}[t]{0.95\textwidth}
        \centering\includegraphics[height=8cm,width=1\textwidth,keepaspectratio]{simplest_sensor.jpg}
        \caption{Простейший преобразователь силы на основе Velostat}
        \label{fig:simplest_sensor.jpg}
    \end{subfigure}
    \caption{Примеры использования Velostat}
\end{figure}

При исследовании преобразователя силы на основе Velostat, было замечено, что площадь нажатия влияет на показания преобразователя. Поэтому было решено характеризовать материал для случаев, когда нагрузка меньше, чем размер сенсора.

\section{Физическая реализация преобразователя силы на основе Velostat}

Датчик состоит из двух медных оболочек, разделенных слоем велостата. Велостат - это упаковочный материал, изготовленный из полимерной пленки (полиолефины), пропитанной сажей для придания ей электропроводности. Он используется для защиты предметов или устройств, которые могут быть повреждены электростатическим разрядом. Свойство изменять свое сопротивление при изгибе или давлении делает его популярным решением для изготовления недорогих датчиков давления.

Датчик состоит из двух медных оболочек, разделенных слоем Velostat. Давление на датчик приводит к изменению его сопротивления: чем выше давление, тем ниже сопротивление. Сопротивление измеряется косвенным методом. Измеренное сопротивление Velostat образует делитель напряжения с постоянным резистором R1...R8 \pic{fig:el_scheme}.


\begin{figure}[H]
\centering\includegraphics[width=0.8\textwidth]{electric_scheme.jpg}\\
\caption{Электрическая схема преобразователя силы}
\label{fig:el_scheme}
\end{figure}

На одну из пластин датчика подается напряжение 3,3 вольта. Таким образом, когда давление на датчик отсутствует (в идеальном случае сопротивление стремится к бесконечности), напряжение на выходе делителя стремится к нулю. По мере увеличения давления сопротивление будет уменьшаться, и напряжение на делителе будет приближаться к напряжению питания.


Созданный преобразователь состоит из двух медных оболочек, разделенных слоем Velostat. Давление на датчик приводит к изменению его сопротивления: чем выше давление, тем ниже сопротивление. На \pic{fig:velostat_pressure_resistance.jpg} показана рабочая область сенсора, основанная на весе, который может быть приложен на одну ногу робота.
\begin{figure}[h]
    \centering
    \begin{tikzpicture}
        % Include the image in a node
        \node [above right, inner sep=0] (image) at (0,0)
        {\centering\includegraphics[height=10cm,width=1\textwidth,keepaspectratio]{velostat_pressure_resistance.jpg}};
        % Create scope with normalized axes
        \begin{scope}[
                x={($ 0.1*(image.south east)$)},
                y={($ 0.1*(image.north west)$)}]
            \draw[stealth-, very thick,green] (4.21,2.75) -- (6.5,5);
            \draw[stealth-, very thick,green] (8.75,2.15) -- (6.5,5)
            node[rounded corners=3pt,above,black,fill=white]{\small Рабочая область};
        \end{scope}
    \end{tikzpicture}
    \caption{График зависимости прикладываемого веса от сопротивления}
    \label{fig:velostat_pressure_resistance.jpg}
\end{figure}

\section{Разработка экспериментального стенда}

Исследования преобразователя Velostat, для случаев которых площадь нагрузки меньше, чем размер преобразователя, были проведены с помощью разработанного для этой цели исследовательского стенда. Среди требований к стенду можно отметить: необходимость контролировать силу нажатия и повторяемость эксперимента как по величине, так и по расположению площадки контакта инструмента и исследуемого преобразователя силы. Указанным требованиям возможно удовлетворить, используя коллаборативный робот-манипулятор, который будет управляться с помощью импедансного управления.

Использование коллаборативного робота позволяет также удовлетворить требованиям безопасности и допустить работу робота в непосредственно близости от экспериментатора. Разработанный стенд, представлен на рисунке \ref{fig:exp_standd}. Видео работы стенда \quad \qrcode[height=1.5cm]{https://youtu.be/Gw4wVZ-ESuE}

\begin{figure}[h]
    \begin{center}
        \begin{subfigure}{0.8\textwidth}
            \begin{tikzpicture}
                % Include the image in a node
                \node [
                    above right,
                    inner sep=0] (image) at (0,0) {\centering\includegraphics[height=10cm,width=1\textwidth,keepaspectratio]{exp_stand1}};

                % Create scope with normalized axes
                \begin{scope}[
                        x={($0.1*(image.south east)$)},
                        y={($0.1*(image.north west)$)}]
                    \draw[latex-, very thick,green] (3.5,2.2) -- (2.5,1)
                    node[rounded corners=3pt,below left,black,fill=white]{\small Velostat сенсор};

                    \draw[stealth-, very thick,green] (3.5,2.6) -- ++(-0.7,+0.5)
                    node[rounded corners=3pt,left,black,fill=white]{\small Датчик силы};

                    \draw[stealth-, very thick,green] (6.5,3) -- (7,6)
                    node[rounded corners=3pt,above right,black,fill=white]{\small Self-made PCB};

                    \draw[stealth-, very thick,green] (7.2,1.5) -- (8,5)
                    node[rounded corners=3pt,above right,black,fill=white]{\small Ардуино};

                    \draw[stealth-, very thick,green] (2.5,9.5) -- (4,9.5)
                    node[rounded corners=3pt,right,black,fill=white]{\small Камера};

                    \draw[very thick,green] (0.5,2.5) rectangle (4.2,9)
                    node[below left,black,fill=green]{\small UR10e};

                    \draw[latex-, very thick,green] (4.5,7.2) edge (5.5,7.5)
                    (4.8,5.3) -- (5.5,7.5)
                    node[rounded corners=3pt,above,black,fill=white]{\small Aruco маркеры};
                \end{scope}
            \end{tikzpicture}
            \caption{Общий вид экспериментального стенда}
            \label{fig:exp_standd}
        \end{subfigure}

        \begin{subfigure}{0.5\textwidth}
            \centering\includegraphics[height=10cm,width=1\textwidth,keepaspectratio]{exp_stand2}
            \caption{Способ нивелировать ошибку по углу с помощью Aruco маркеров}
            \label{fig:exp_stand2}
        \end{subfigure}
        \caption{Разработанный экспериментальный стенд}
    \end{center}
\end{figure}

Для касания только части объекта исследования были разработаны различные насадки. Такие размеры были выбраны из-за размеров преобразователя. Минимальный размер препятствия, которое может коснуться было взято за 2 мм. А длина ребра датчика -- 15 мм. Поэтому 15 мм насадка является максимальной\pic{fig:all_end_effectors.png}.

\begin{figure}[h]
    \begin{subfigure}[t]{0.6\textwidth}
        \centering
        \begin{tikzpicture}
            % Include the image in a node
            \node [above right, inner sep=0] (image) at (0,0)
            {\centering\includegraphics[height=5cm,width=1\textwidth,keepaspectratio]{all_end_effectors.png}};
            % Create scope with normalized axes
            \begin{scope}[
                    x={($ 0.1*(image.south east)$)},
                    y={($ 0.1*(image.north west)$)}]
                \node[rounded corners=3pt,black,fill=white] at (1.1,7.4){\tiny 2 mm };
                \node[rounded corners=3pt,black,fill=white] at (3.1,7.9){\tiny 6 mm };
                \node[rounded corners=3pt,black,fill=white] at (4.9,8.1){\tiny 8 mm };
                \node[rounded corners=3pt,black,fill=white] at (6.7,7.9){\tiny 12 mm };
                \node[rounded corners=3pt,black,fill=white] at (8.6,7.9){\tiny 15 mm };
            \end{scope}
        \end{tikzpicture}
        \caption{Насадка для нажатия объект
            исследования с диаметром нажатия меньше, чем сам объект}
        \label{fig:all_end_effectors.png}
    \end{subfigure}
    \begin{subfigure}[t]{0.38\textwidth}
        \centering
        \begin{tikzpicture}

            % Include the image in a node
            \node [
                above right,
                inner sep=0] (image) at (0,0) {\centering\includegraphics[height=5cm,width=1\textwidth,keepaspectratio]{sensors_grid.png}};

            % Create scope with normalized axes
            \begin{scope}[
                    x={($0.1*(image.south east)$)},
                    y={($0.1*(image.north west)$)}]
                \draw [green, very thick,
                    decorate,
                    decoration = {brace,
                            raise=5pt,
                            amplitude=5pt,
                            aspect=0.5}] (6,3.7) --  (3,3.7)
                node[pos=0.5,below=10pt,green]{$15\ mm$};

                \draw [green, very thick,
                    decorate,
                    decoration = {brace, mirror,
                            raise=5pt,
                            amplitude=5pt,
                            aspect=0.5}] (6,3.6) --  (6,6.4)
                node[pos=0.5,right=10pt,green]{$15\ mm$};

                \draw[green,step=1,xshift=34, yshift=43]  (0.5,0.5) grid +(3,3);

                \node[circle,fill=green,scale=0.4] at (3.3,6.27){\small 1};
                \node[circle,fill=green,scale=0.4] at (5.92,3.7){\small 16};
            \end{scope}

        \end{tikzpicture}
        \caption{Сенсор представлен \\ как $4\times4$ сетка}
        \label{fig:sensor_grid}
    \end{subfigure}
    \caption{Представление места нажатия инструментом сенсора и сам инструмент}
\end{figure}

Импедансное управление состоит из двух блоков -- модификация траектории для оси $z$, начиная с\eqref{eq:traj_mod}, и управление по скорости -- с \eqref{eq:vel_control}.

\begin{eqnarray}
    \label{eq:traj_mod}
    X_s^0 = 0, \dot{X}_s^0 =0,  X_g^k, \dot{X}_g^k \text{ -- goal state}, X_s = X_g - X_d \\
    X_g = X_g^0 + \frac{F_d}{\eta } \\
    \dot{X}_s + \eta  X_s = F^k \\
    X_s^k = odeint(X_s^{k-1},t,F^k), t = [0,dT] \\
    X_s^{k-1} = X_s^k;  \dot{X}_s = f(X_s,t,F^k) \\
    X_d = X_g - X_s; \dot{X}_d = \dot{X}_g - \dot{X}_s
\end{eqnarray}

\begin{eqnarray}
    \label{eq:vel_control}
    X_d = \begin{bmatrix}
        x_g \\ y_g \\ z_d
    \end{bmatrix} \\
    U = \dot{X}_d + K(X_d - X), \\ \text{ where } X=get\_state(); \\ 
    set\_speed(U)
\end{eqnarray}

На рисунке ниже \pic{fig:force_data_pos.png} представлен результат работы импедансного управления на частоте 450 $Hz$. Необходимая сила нажатия --- $17\ H$.
\begin{figure}[h]
        \centering
         \begin{tikzpicture}
            % Include the image in a node
            \node [above right, inner sep=0] (image) at (0,0) 
            {\centering\includegraphics[height=10cm,width=1\textwidth,keepaspectratio]{force_data_pos.png}};          
            % Create scope with normalized axes
            \begin{scope}[
                x={($ 0.1*(image.south east)$)},
                y={($ 0.1*(image.north west)$)}]
                \draw[thick,green, dashed] (4.2,1) -- (4.2,8)
                node[above right,black,fill=white]{\tiny Касание с поверхностью};
            \end{scope}
        \end{tikzpicture}
        \caption{Графики зависимости силы и позиции по $z$ от времени во время эксперимента по исследованию Velostat}
        \label{fig:force_data_pos.png}
    \end{figure}

\section{Экспериментальная часть}

В исследовании были проведены:
\begin{enumerate}
    \item \textbf{статический эксперимент}. Цель — определить коэффициенты для математической модели преобразователя. Для этого на сенсор кладется известная нагрузка на 60 секунд (за это время можно явно наблюдать гистерезис) и собираются данные с преобразователя;
          \item\textbf{динамический эксперимент}. Цель — определить влияние показаний сенсора в зависимости от положения площадки контакта. Для этого преобразователь представлен в виде матрицы $4 \times 4$. Размер преобразователя в эксперименте 15 на 15 мм. Манипулятор нажимает на преобразователь с одинаковым давлением на протяжении всех экспериментов в различные позиции на преобразователе, используя пять различных насадок (диаметр окружности от 2 мм до 15 мм) \pic{fig:sensor_grid}.
\end{enumerate}

Статическим экспериментом проверялась формула \eqref{eq:velostat_eqn}. Из-за гистерезиса необходимо учитывать время нажатия на объект. При прикладывании на сенсор константной нагрузки показания сенсора будут меняться.
\begin{eqnarray}
    \label{eq:velostat_eqn}
    V_{out} = V_0 + p[k_p + k_e(1-e^\frac{-(t-t_0)}{\tau_{res}})](1-e^{-\frac{A}{p}}) \\
    k_p = A_1e^{-A_2p}; \tau_{res} = B_0 + B_1e^{-\frac{p}{B_2}}
\end{eqnarray}
где,  \nom{$V_0$}{начальное напряжение}, \nom{$p,\ A_i,\ B_i,\ \tau_{res},\ k_i$}{настраиваемые константы}, \nom{$t$}{текущее время}, \nom{$t_0$}{время начала нажатия}.
Для решения задачи регрессии использовался робастный нелинейный алгоритм наименьших квадратов. Результат представлен ниже \pic{fig:least_square_model.png}.

\begin{figure}[H]
    \centering\includegraphics[height=10cm,width=1\textwidth,keepaspectratio]{least_square_model.png}
    \caption{Результаты статического эксперимента}
    \label{fig:least_square_model.png}
\end{figure}

Ниже \pic{fig:dynamics_exp} представлены некоторые результаты распределения ошибок по площади сенсора при взаимодействии с насадками разных размеров. Ошибки определялись как разница между показаниями калиброванного сенсора силы Futek и исследуемого преобразователя на базе Velostat. На рисунке \ref{fig:sens1_pike1} показаны ошибки для насадки диаметром 2 мм, а на рисунке \ref{fig:sens1_pike3} — для насадки диаметром 8 мм.

\begin{figure}[H]
    \centering\includegraphics[width=0.99\textwidth]{sensor_sensor.png}\\
    \caption{Проверка чувствительности датчика. Слева - идеальные данные, справа - результат, полученный с помощью созданного датчика.}
    \label{fig:sensor_sensor}
    \end{figure}

Можно заметить, что в \ref{fig:sens1_pike3} максимальная разница между Futek и Velostat не более 0.2 единиц в одном месте. Остальные элементы сетки не превышают 10\%. Такая же тенденция продолжается как и при увеличении размера насадки, так и на других сенсорах.


\begin{figure}[H]
    \begin{subfigure}{0.49\textwidth}
        \centering\includegraphics[height=10cm,width=1\textwidth,keepaspectratio]{sens1_pike1.png}
        \caption{диаметр насадки равный 2 мм }
        \label{fig:sens1_pike1}
    \end{subfigure}
    \begin{subfigure}{0.49\textwidth}
        \centering\includegraphics[height=10cm,width=1\textwidth,keepaspectratio]{sens1_pike3.png}
        \caption{Диаметр насадки равный 8 мм }
        \label{fig:sens1_pike3}
    \end{subfigure}
    \caption{Динамический эксперимент}
    \label{fig:dynamics_exp}
\end{figure}

По результатам исследований показано, что характеристики преобразователя удовлетворяют требованиям к системе тактильного восприятия шагающего робота, когда ожидаемый размер площади контакта превышает 25 процентов площади преобразователя.