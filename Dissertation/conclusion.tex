\chapter*{Заключение}                       % Заголовок
\addcontentsline{toc}{chapter}{Заключение}  % Добавляем его в оглавление

%% Согласно ГОСТ Р 7.0.11-2011:
%% 5.3.3 В заключении диссертации излагают итоги выполненного исследования, рекомендации, перспективы дальнейшей разработки темы.
%% 9.2.3 В заключении автореферата диссертации излагают итоги данного исследования, рекомендации и перспективы дальнейшей разработки темы.

Основной  научный  результат  диссертации  заключается  в  решении  актуальной 
научной  задачи,  имеющей  важное  практическое  значение: разработка метода тактильного очувствления для мобильного шагающего робота в закрытых пространствах естественного или искусственного происхождения. 

При  проведении  исследований  и  разработок  в  диссертационной  работе  получены 
следующие результаты, обладающие научной новизной:
\begin{enumerate}
  \item метод оптимизации конструкции многоногих роботов;
  \item разработанная методика исследования датчика силы, когда площадь нажатия на сенсор меньше самого сенсора;
  \item реализация программно-алгоритмического обеспечения, позволяющего определять тип поверхности;
  \item методика построения карты местности с помощью датчиков силы, установленных на ногах робота.
\end{enumerate}

\textbf{Доказана} возможность построения карты местности и определения типа поверхности с помощью тактильного очувствления как в робототехническом симуляторе, так с помощью натурного эксперимента.

\textbf{Показано}, что оптимальное количество ног для циклового движителя с одной степенью свободы в ноге находится в диапазоне от 8 до 14 ног. 

\textbf{Предложено} использовать преобразователь силы на основе полимерного материала Velostat. \textbf{Установлено}, что данный преобразователь можно рассматривать как единое тело, при площади нажатия больше 50\% площади сенсора. 

\textbf{Сделан вывод} об эффективности предложенных методик, на основе результатов натурных испытаний.
