\clearpage                                  % В том числе гарантирует, что список литературы в оглавлении будет с правильным номером страницы

\nocite{a27442152,a36488151,a41451571,a42488020,a42953245,a44730151,a47144307,a47478445,a47571510,a47783599,a47783681,a48268534,bulichevConceptDevelopmentBiomimetic2018,bulichevOptimizationCentipedeRobot2017,sokolov2017analysis,bulichevProgrammaObucheniyaRobota2019,bulichevProgrammaPlanirovaniyaTraektorii2020,bulichevProgrammaPodboraKinematicheskih2019,bulichevProgrammaPodboraKinematicheskih2019,bulichevProgrammaPozicionirovaniyaStabilizacii2019,bulichevProgrammaUpravleniyaZvenyami2020}

%\hypersetup{ urlcolor=black }               % Ссылки делаем чёрными
%\providecommand*{\BibDash}{}                % В стилях ugost2008 отключаем использование тире как разделителя
\urlstyle{rm}                               % ссылки URL обычным шрифтом
\ifdefmacro{\microtypesetup}{\microtypesetup{protrusion=false}}{} % не рекомендуется применять пакет микротипографики к автоматически генерируемому списку литературы
% \insertbibliofull                           % Подключаем Bib-базы: все статьи единым списком
% Режим с подсписками
\insertbiblioexternal                      % Подключаем Bib-базы: статьи, не являющиеся статьями автора по теме диссертации
% Для вывода выберите и расскомментируйте одно из двух
%\insertbiblioauthor                        % Подключаем Bib-базы: работы автора единым списком 
% \nocite{*}
\insertbiblioauthorgrouped                 % Подключаем Bib-базы: работы автора сгруппированные (ВАК, WoS, Scopus и т.д.)
\ifdefmacro{\microtypesetup}{\microtypesetup{protrusion=true}}{}
\urlstyle{tt}                               % возвращаем установки шрифта ссылок URL
%\hypersetup{ urlcolor={urlcolor} }          % Восстанавливаем цвет ссылок

