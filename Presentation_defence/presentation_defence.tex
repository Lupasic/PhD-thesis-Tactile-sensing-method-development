% \documentclass[aspectratio=169,notes]{beamer}
\documentclass[aspectratio=169,xcolor=table]{beamer}
\usetheme[faculty=phil]{fibeamer}
\usepackage{polyglossia}
\setmainlanguage{english} %% main locale instead of `english`, you
%% can typeset the presentation in either Czech or Slovak,
%% respectively.
\setotherlanguages{russian} %% The additional keys allow
%%
%%   \begin{otherlanguage}{czech}   ... \end{otherlanguage}
%%   \begin{otherlanguage}{slovak}  ... \end{otherlanguage}
%%
%% These macros specify information about the presentation
\title[]{Разработка метода тактильного очувствления для мобильного шагающего робота} %% that will be typeset on the
\subtitle{Соискатель: Олег Буличев \\ Руководитель: Александр Малолетов \\ \ } %% title page.
\author{Олег Буличев}
%% These additional packages are used within the document:
\usepackage{ragged2e}  % `\justifying` text
\usepackage{booktabs}  % Tables
\usepackage{tabularx}
\usepackage{tikz}      % Diagrams
\usetikzlibrary{calc, shapes, backgrounds}
\usepackage{amsmath, amssymb}
\usepackage{url}       % `\url`s
\usepackage{listings}  % Code listings
\usepackage{floatrow}
\usepackage{mathtools}
\usepackage{todonotes}
\usepackage{fontspec}
\usepackage{multicol}
\usepackage{pdfpages}
\usepackage{wrapfig}
\usepackage{animate}
\usepackage{booktabs}
\usepackage{multirow}
\usepackage{multimedia}
\usepackage{makecell}
\usepackage{colortbl}
\usepackage{hhline}
\usepackage{rotating}
\usepackage{amsmath}

\usepackage[font={large}, labelfont=it,textfont={it},justification=centering, skip=2pt]{caption}
% will apply to all subcaptions
\usepackage[font={large},skip=2pt]{subcaption}

\graphicspath{{../images/}}
\frenchspacing


\usetikzlibrary{decorations.pathreplacing,calligraphy,calc,graphs}

\setbeamertemplate{caption}[numbered]

% \usepackage[backend=biber,style=ieee,autocite=footnote]{biblatex}
% \addbibresource{biblio.bib}
% \DefineBibliographyStrings{english}{%
%   bibliography = {References},}

\newcommand{\oleg}[2][] {\todo[color=red, #1] {OLEG:\\ #2}}
\newcommand{\fbckg}[1]{\usebackgroundtemplate{\includegraphics[width=\paperwidth]{#1}}}%frame background

\usepackage[framemethod=TikZ]{mdframed}
\newcommand{\dbox}[1]{
\begin{mdframed}[roundcorner=3pt, backgroundcolor=yellow, linewidth=0]
\vspace{1mm}
{#1}
\vspace{1mm}
\end{mdframed}
}

\begin{document}
\setlength{\abovedisplayskip}{0pt}
\setlength{\belowdisplayskip}{0pt}
\setlength{\abovedisplayshortskip}{0pt}
\setlength{\belowdisplayshortskip}{0pt}

\fbckg{fibeamer/figs/title_page.png}
\frame[c]{\setcounter{framenumber}{0}
    \usebeamerfont{title}%
    \usebeamercolor[fg]{title}%
    \begin{minipage}[b][7.5\baselineskip][b]{\textwidth}%
        \textcolor{black}{\raggedright\inserttitle}
    \end{minipage}
    % \vskip-1.5\baselineskip

    \usebeamerfont{subtitle}%
    \usebeamercolor[fg]{framesubtitle}%
    \begin{minipage}[b][3\baselineskip][b]{\textwidth}
        \raggedright%
        \insertsubtitle%
    \end{minipage}
    \vskip.25\baselineskip
}
%   \frame[c]{\maketitle}

\fbckg{fibeamer/figs/common.png}

\begin{frame}[t]{О себе}
    \framesubtitle{}
    \vspace{-0.2cm}
    \begin{exampleblock}{Образование}
        \vspace{-0.2cm}
        \begin{itemize}
            \item Бакалавриат --- МГТУ им. Н.Э. Баумана, РК6 (красный диплом) \\ \textbf{Тема:} Разработка системы управления наведением МРК <<Пластун>>
            \item Магистратура --- Университет Иннополис, Робототехника \\ \textbf{Тема:} Development of biomimetic centipede robot <<StriRus>>
            \item Аспирантура --- Университет Иннополис, Робототехника \\ \textbf{Тема:} Tactile perception method development for a mobile walking~robot
        \end{itemize}
    \end{exampleblock}
    \begin{alertblock}{Текущие должности}
        \vspace{-0.2cm}
        \begin{itemize}
            \item Старший преподаватель (ЛинАл, Механизмы и Машины, ТеорМех)
            \item Тренер - преподаватель (Туризм, Историческое фехтование, Народные игры)
        \end{itemize}
    \end{alertblock}
\end{frame}

\begin{frame}[t]{Необходимость исследования пещер роботами}
    % \framesubtitle{Cave dangerous obstacles}
    \vspace{-0.85cm}
    \begin{figure}[H]
        \begin{subfigure}[b]{0.3\textwidth}
            \centering\includegraphics[height=2.8cm,width=1\textwidth,keepaspectratio]{surface_types/salt.jpg}\\
            \caption*{Соляные отложения}
            \label{fig:surface_types/salt}
        \end{subfigure}
        \hfill
        \begin{subfigure}[b]{0.3\textwidth}
            \centering\includegraphics[height=2.8cm,width=1\textwidth,keepaspectratio]{surface_types/siphon.png}\\
            \caption*{Сифоны}
            \label{fig:surface_types/siphon}
        \end{subfigure}
        \hfill
        \begin{subfigure}[b]{0.3\textwidth}
            \centering\includegraphics[height=2.8cm,width=1\textwidth,keepaspectratio]{surface_types/ice.png}\\
            \caption*{Ледяные пещеры}
            \label{fig:surface_types/ice}
        \end{subfigure}

        \begin{subfigure}[b]{0.3\textwidth}
            \centering
            \begin{tikzpicture}
                % Include the image in a node
                \node [above right, inner sep=0] (image) at (0,0)
                {\centering\includegraphics[height=2.8cm,width=1\textwidth,keepaspectratio]{surface_types/clay.jpg}};
                % Create scope with normalized axes
                \begin{scope}[
                        x={($ 0.1*(image.south east)$)},
                        y={($ 0.1*(image.north west)$)}]
                    % Grid and axes' labels
                    % \draw[lightgray,step=1] (image.south west) grid (image.north east);
                    % \foreach \x in {0,1,...,10} { \node [below] at (\x,0) {\x}; }
                    % \foreach \y in {0,1,...,10} { \node [left] at (0,\y) {\y};}
                    % Labels
                    \draw[stealth-, very thick,green] (6,8) -- ++(1,1)
                    node[rounded corners=3pt,right,black,fill=white]{\tiny Человек};
                \end{scope}
            \end{tikzpicture}
            \caption*{Глина}
            \label{fig:surface_types/clay.jpg}
        \end{subfigure}
        \hfill
        \begin{subfigure}[b]{0.3\textwidth}
            \centering
            \begin{tikzpicture}
                % Include the image in a node
                \node [above right, inner sep=0] (image) at (0,0)
                {\centering\includegraphics[height=2.8cm,width=1\textwidth,keepaspectratio]{surface_types/splash.png}};
                % Create scope with normalized axes
                \begin{scope}[
                        x={($ 0.1*(image.south east)$)},
                        y={($ 0.1*(image.north west)$)}]
                    % Grid and axes' labels
                    % \draw[lightgray,step=1] (image.south west) grid (image.north east);
                    % \foreach \x in {0,1,...,10} { \node [below] at (\x,0) {\x}; }
                    % \foreach \y in {0,1,...,10} { \node [left] at (0,\y) {\y};}

                    % Labels
                    \draw[stealth-, very thick,green] (5,2) -- ++(-2,+1)
                    node[rounded corners=3pt,left,black,fill=white]{\tiny Лужа};
                \end{scope}
            \end{tikzpicture}
            \caption*{Лужа}
            \label{fig:surface_types/splash.png}
        \end{subfigure}
        \hfill
        \begin{subfigure}[b]{0.3\textwidth}
            \centering\includegraphics[height=2.8cm,width=1\textwidth,keepaspectratio]{surface_types/moss.jpg}\\
            \caption*{Мох}
            \label{fig:surface_types/moss}
        \end{subfigure}
    \end{figure}
\end{frame}

\begin{frame}[t]{Проблематика}
    \framesubtitle{}
    \vspace{-0.3cm}
    \begin{block}{Проблема}
        \textit{Нет технологий} для исследования \textbf{узких пещер} естественного происхождения
    \end{block}
    \begin{columns}[T,onlytextwidth]
        \begin{column}{0.59\textwidth}
            Стандартное решение для автономной навигации не будет работать по следующим причинам:
            \begin{itemize}
                \item оптические сенсоры (лидары, камеры) могут покрыться грязью;
                \item камеры выдают некачественные данные при недостатке освещения;
                \item спутниковая навигация (GPS) не работает в замкнутых пространствах под землей.
            \end{itemize}
        \end{column}
        \begin{column}{0.39\textwidth}
            \begin{figure}[H]
                \centering\includegraphics[height=3cm,width=1\textwidth,keepaspectratio]{open_cave.jpg}
                \caption*{DARPA Subterranean Challenge, свободная пещера}
                \label{fig:open_cave.jpg}
            \end{figure}
        \end{column}
    \end{columns}
\end{frame}

\begin{frame}[t]{Нерешаемая задача с помощью камеры или лидара}
    \framesubtitle{Вопрос: Как картографировать поверхность под лужей?}
    \vspace{-1cm}
    \begin{columns}[T,onlytextwidth]
        \begin{column}{0.55\textwidth}
            \begin{figure}[H]
                \centering\includegraphics[height=6cm,width=1\textwidth,keepaspectratio]{terrain_wo_water.png}
                \caption*{Поверхность без воды}
            \end{figure}
        \end{column}
        \begin{column}{0.44\textwidth}
            \begin{figure}[H]
                \begin{subfigure}[b]{0.9\textwidth}
                    \centering
                    \begin{tikzpicture}
                        % Include the image in a node
                        \node [above right, inner sep=0] (image) at (0,0)
                        {\centering\includegraphics[height=3.5cm,width=1\textwidth,keepaspectratio]{terrain_w_water1.png}};
                        % Create scope with normalized axes
                        \begin{scope}[
                                x={($ 0.1*(image.south east)$)},
                                y={($ 0.1*(image.north west)$)}]
                            \draw[stealth-, very thick,green] (6,8) -- ++(2,1)
                            node[rounded corners=3pt,right,black,fill=white]{\tiny Вода};

                            \draw[stealth-, very thick,green] (0.5,5.5) -- (3,2);
                            \draw[stealth-, very thick,green] (2.5,4.2) -- (3,2);
                            \draw[stealth-, very thick,green] (4.5,4) -- (3,2)
                            node[rounded corners=3pt,below,black,fill=white]{\tiny Лазер Лидара};
                        \end{scope}
                    \end{tikzpicture}
                    % \caption*{}
                    \label{fig:terrain_w_water1.png}
                \end{subfigure}
                \vspace{-0.5cm}

                \begin{subfigure}{0.8\textwidth}
                    \centering\includegraphics[height=2cm,width=1\textwidth,keepaspectratio]{terrain_w_water_camera.png}
                    \caption*{Вид с камеры}
                \end{subfigure}
            \end{figure}
        \end{column}
    \end{columns}

\end{frame}

\begin{frame}[t]{Цель работы}
    \framesubtitle{}
    \vspace{-0.4cm}
    Разработать метод для определения \underline{геометрических} и \underline{физических} свойств пройденной \textbf{поверхности} с помощью многоногого шагающего робота с цикловыми движителемями, используя \textit{тактильное очувствление}, без использования оптических сенсоров.
    \begin{figure}[H]
        \begin{subfigure}{0.49\textwidth}
            \centering\includegraphics[height=3.5cm,width=1\textwidth,keepaspectratio]{conv_concave.png}
            \caption*{Определение геометрических свойств}
        \end{subfigure}
        \begin{subfigure}{0.49\textwidth}
            \centering\includegraphics[height=3.5cm,width=1\textwidth,keepaspectratio]{s_shape_leg/view.jpg}
            \caption*{Определение физических свойств}
            \label{fig:s_shape_leg/view.jpg}
        \end{subfigure}
    \end{figure}
\end{frame}

\begin{frame}[t]{Объект исследования}
    \framesubtitle{}
    Объектом исследования является \textbf{класс многоногих шагающих роботов} с цельным или сочленённым корпусом, и цикловыми движителями с одной степенью свободы, управляемые зависимо или независимо друг от друга.

    \begin{figure}[H]
        \begin{subfigure}{0.32\textwidth}
            \centering\includegraphics[height=3cm,width=1\textwidth,keepaspectratio]{from_master/gakken.jpg}
            \label{fig:from_master/gakken.jpg}
        \end{subfigure}
        \begin{subfigure}{0.32\textwidth}
            \centering\includegraphics[height=3cm,width=1\textwidth,keepaspectratio]{from_master/rhex.jpg}
            \label{fig:from_master/rhex.jpg}
        \end{subfigure}
        \begin{subfigure}{0.32\textwidth}
            \centering\includegraphics[height=3cm,width=1\textwidth,keepaspectratio]{from_master/whegs2.jpg}
            \label{fig:from_master/whegs2.jpg}
        \end{subfigure}
    \end{figure}
\end{frame}

\begin{frame}[t]{Основные научные задачи исследования}
    \framesubtitle{}
    \begin{enumerate}
        \item Разработка метода \textbf{оптимизации конструкции многоногих шагающих роботов} с цикловыми движителями с одной степенью свободы критериям проходимости, покрытия опорной поверхности и её детализации, длины пройденного пути.
        \item Создание методики \textbf{исследования датчика силы}, когда площадь контакта нажатия на сенсор меньше чувствительной области самого сенсора.
        \item  Разработка метода \textbf{построения карты местности и определения геометрических свойств поверхности} с помощью тактильного очувствления.
        \item Реализация алгоритма, позволяющего \textbf{определять физические свойства} опорной поверхности.
    \end{enumerate}
\end{frame}


\usebackgroundtemplate{}
\setbeamercolor{background canvas}{bg=}
\begin{frame}[t]{}
    \framesubtitle{}
    \vspace{-0.4cm}
    \begin{figure}[H]
        \centering\includegraphics[height=12cm,width=1\textwidth,keepaspectratio]{main_diag_hor.png}
        \label{fig:main_diag_hor.png}
    \end{figure}
\end{frame}
\fbckg{fibeamer/figs/common.png}

\begin{frame}[c]{}
    \framesubtitle{}
    \centering\LARGE Разработка робота
\end{frame}

% \begin{frame}[t]{Литературный обзор}
%     \framesubtitle{}
%     \vspace{-0.65cm}
%     \begin{itemize}
%         \item Пещеры: препятствия, размеры. \\ \alert{-- \textbf{Классификация} пещер и препятствий \\ -- \textbf{Оценка сложности} территории}
%         \item Роботы для исследования пещер: от дирижаблей, до шагающих. \\ \alert{-- \textbf{Робототехнические системы} для исследования \textbf{свободных пещер}}
%         \item Способы определения силы реакции опоры. \\ \alert{-- Неявные и явные \textbf{способы}. \textbf{Классификая типов датчиков} силы}
%         \item Методы распознования типа поверхности. \\ \alert{-- С помощью \textbf{машинного обучения}, используя набор датчиков}
%         \item Методы построения карты: оптические и тактильные. \\ \alert{-- Построение поверхности \textbf{с помощью датчика силы на манипуляторе} \\ -- Построение карты с помощью \textbf{лидаров и камер}}
%     \end{itemize}

% \end{frame}



\begin{frame}[t]{Разработка робота}
    \framesubtitle{Задачи}
    \large
    \vspace{-0.5cm}
    \begin{columns}[T,onlytextwidth]
        \begin{column}{0.49\textwidth}
            \begin{itemize}
                \item Смоделировать робота
                \item Разработать критерий оптимизации конструкции
                \item Решить задачу оптимизации
                \item Спроектировать и собрать прототип
            \end{itemize}
        \end{column}
        \begin{column}{0.49\textwidth}
            \vspace{-1.1cm}
            \begin{figure}[H]
                \centering\includegraphics[height=5cm,width=1\textwidth,keepaspectratio]{strirus_4.png}
                \caption*{Шагающий цикловой движитель с 1 степенью свободы в ноге \\ \textbf{СтриРус}, 4-ая итерация}
                \label{fig:strirus_4.png}
            \end{figure}
        \end{column}
    \end{columns}
\end{frame}

\begin{frame}[t]{Разработка робота}
    \framesubtitle{Математическая модель: Описание механической системы}
    \vspace{-0.8cm}
    \begin{align*}
        M \dot{\vec{u}} = \vec{g}                                \\
        M = \begin{bmatrix}
                M_1    & \cdots & 0      \\
                \vdots & \ddots & \vdots \\
                0      & \cdots & M_n
            \end{bmatrix},\ M_i = \begin{bmatrix}
                                      m_i E_{3\times 3} & 0   \\
                                      0                 & I_i
                                  \end{bmatrix}        \\
        \vec{u}_i^{\ T} = \begin{bmatrix}
                              \vec{v}_i^{\ T} & \vec{\omega}_i^{\ T}
                          \end{bmatrix} \\
        \vec{g}^{\ T} = \begin{bmatrix}
                            \cdots \  \vec{F}_i^{\ T}, & (\vec{\tau}_i - \vec{\omega}_i \times I_i \vec{\omega}_i)^T\  \cdots
                        \end{bmatrix}
    \end{align*}
    где, $M_i$~---~матрицы, содержащие массово-инерционные характеристики; $m_i$~---~масса тела; $I_i$~---~тензор инерции; $\vec{u_i}$~---~вектор обобщённых скоростей; $E$~---~единичная матрица; $\vec{g}$~---~вектор обобщённых сил; $\vec{v_i}$~---~вектор линейной скорости; $\vec{\omega_i}$~---~вектор угловой скорости; $\vec{F_i}$, $\vec{\tau_i}$~---~силы и моменты сил взаимодействия.
\end{frame}

\begin{frame}[t]{Разработка робота}
    \framesubtitle{Геометрические связи}
    \vspace{-0.5cm}
    Тела соединены цилиндрическими шарнирами:
    \begin{align*}
        \label{eq:kin_constr}
        \phi(q_{j_1},\ u_{j_1},\ \cdots,\ q_{j_k},\ u_{j_k},\ t) \geqslant  0 \\
        \vec{q_i}^{\ T} = \begin{bmatrix}
                              \vec{x}_i^{\ T} & \vec{Q}_i^{\ T}
                          \end{bmatrix}                   \\
        \dot{\vec{q_i}} = \begin{bmatrix}
                              E_{3\times3} & 0            \\
                              0            & G(\vec{q}_i)
                          \end{bmatrix}\vec{u}_i
    \end{align*}

    \begin{align*}
        \vec{g}_i = \tau_i^T \vec{z}_{i-1} -k_i \dot{\vec{q_i}}
    \end{align*}
    где через $\phi$ обозначена функция связи; $t$~---~время; $\vec{q}_{i}$~---~вектор обобщенных координат, включающий в себя координаты центра масс $\vec{x_i}$ и кватернион $\vec{Q_i}$, описывающий ориентацию тела в пространстве; через $G(\vec{q}_i)$ обозначена матрица, вид которой зависит от выбранной системы координат; $k$~---~ коэффициент вязкого трения в шарнире.
\end{frame}

\begin{frame}[t]{Разработка робота}
    \framesubtitle{Взаимодействие опорной поверхности и ноги робота}
    \vspace{-0.6cm}
    \begin{columns}[T,onlytextwidth]
        \begin{column}{0.69\textwidth}
            \begin{align*}
                \phi_u(\vec{q}\ ) \geqslant 0                                                     \\
                \phi_u(\vec{q}\ ) = (\vec{x}_1 + \vec{s}_1 - \vec{x}_2 - \vec{s}_2) \cdot \vec{n} \\
                \frac{d }{d t}\phi_u(\vec{q}\ ) \approx \begin{bmatrix}
                                                            \vec{n}^{\ T} & (\vec{s}_1 \times \vec{n})^T & -\vec{n}^{\ T} & (-\vec{s}_2 \times \vec{n})^T
                                                        \end{bmatrix} \begin{bmatrix}
                                                                          \vec{v}_1      \\
                                                                          \vec{\omega}_1 \\
                                                                          \vec{v}_2      \\
                                                                          \vec{\omega}_2 \\
                                                                      \end{bmatrix}
            \end{align*}
            где, $\phi_u(\vec{q})$~---~функция связи; $ \mu $~---~ коэффициент трения между ногой и опорной поверхностью;  радиус-векторы $\vec{x}_{1,2},\ \vec{s}_{1,2}$ и орты координатных осей $\vec{t}_{1,2}, \vec{n}$ показаны на рисунке; $ f_{1,2} $~---~значения сил трения вдоль осей $t_{1,2}$.
        \end{column}
        \begin{column}{0.29\textwidth}
            \vspace{-0.4cm}
            \begin{figure}[H]
                \centering\includegraphics[height=6cm,width=1\textwidth,keepaspectratio]{contact_interaction.png}
                \label{fig:contact_interaction.png}
            \end{figure}
            \vspace{-1cm}
            \begin{align*}
                \left\{\begin{matrix*}[l]
                           \mu f_n \geqslant \sqrt{f_1^2 + f_2^2}\\
                           \left\lVert \vec{v_t}\right\rVert (\mu f_n - \sqrt{f_1^2 + f_2^2}) = 0\\
                           \dfrac{\vec{f_t}}{\left\lVert \vec{f_t}\right\rVert } = - \dfrac{\vec{v_t}}{\left\lVert \vec{v_t}\right\rVert }
                       \end{matrix*}\right.
            \end{align*}
        \end{column}
    \end{columns}
\end{frame}

\begin{frame}[t]{Разработка робота}
    \framesubtitle{Структурный синтез}
    {\large\begin{block}{Вопрос}
            Какое оптимальное количество ног должен иметь такой движитель?
        \end{block}}
    {\large\begin{alertblock}{Ответ}
            \centering Решив задачу структурного синтеза,\\ результатом которого является движитель с \textbf{8---14 ногами}
        \end{alertblock}}
\end{frame}

\begin{frame}[c]{Разработка робота}
    \framesubtitle{Используемые технологии}
    \vspace{-0.7cm}
    \begin{figure}[H]
        \begin{subfigure}[t]{0.45\textwidth}
            \centering\includegraphics[height=4.5cm,width=1\textwidth,keepaspectratio]{c1_paper.png}
            \caption*{\small Генерация поверхности \\ (Параметризованная \\ \textbf{искусственная территория})}
        \end{subfigure}
        \begin{subfigure}[t]{0.45\textwidth}
            \centering\includegraphics[height=4.5cm,width=1\textwidth,keepaspectratio]{gen_algo.jpg}
            \caption*{ \small Генетический алгоритм}
        \end{subfigure}
        \hfill
    \end{figure}
\end{frame}

\begin{frame}[t]{Разработка робота}
    \framesubtitle{Предлагаемое решение}
    \begin{columns}[T,onlytextwidth]
        \begin{column}{0.48\textwidth}
            \begin{figure}[H]
                \centering\includegraphics[height=3cm,width=1\textwidth,keepaspectratio]{optimization_idea.png}
                \caption*{\textbf{Идея}: Минимизировать кол-во ног без потери проходимости}
                \label{fig{optimization_idea.png}}
            \end{figure}
        \end{column}
        \begin{column}{0.50\textwidth}
            \vspace{-2cm}
            \begin{figure}[H]
                \centering
                \begin{tikzpicture}
                    % Include the image in a node
                    \node [above right, inner sep=0] (image) at (0,0)
                    {\centering\includegraphics[height=2.5cm,width=1\textwidth,keepaspectratio]{best_gen_robot.jpg}};
                    % Create scope with normalized axes
                    \begin{scope}[
                            x={($ 0.1*(image.south east)$)},
                            y={($ 0.1*(image.north west)$)}]
                        % Grid and axes' labels
                        % \draw[lightgray,step=1] (image.south west) grid (image.north east);
                        % \draw[lightgray,step=0.5] (image.south west) grid (image.north east);
                        % \foreach \x in {0,1,...,10} { \node [below] at (\x,0) {\x}; }
                        % \foreach \y in {0,1,...,10} { \node [left] at (0,\y) {\y};}

                        % Labels
                        \draw [green, very thick,
                            decorate,
                            decoration = {brace,
                                    raise=5pt,
                                    amplitude=5pt,
                                    aspect=0.5}] (1.4,3.6) --  (8.1,6.8)
                        node[rounded corners=3pt, pos=0.5,above left =14pt,black,fill=white]{\tiny $(\gamma - 1) h_{\text{leg}}sin(\alpha)$};

                        \draw[stealth-, very thick,green] (9.5,7.8) -- (7.8,1.94);
                        \draw[stealth-, very thick,green] (1.5,2.8) -- (7,1)
                        node[rounded corners=3pt,right,black,fill=white]{\tiny $\gamma = 6$};

                        \draw[thin,green] (6.7,4) -- (5.75,9);
                        \draw[thin,green] (4.85,3.5) -- (5.75,9);
                        \draw[thin,green,stealth-stealth] (6.32,6) arc (-79.2:-99.2:3) node [rounded corners=3pt,below = 2pt,black,fill=white, midway] {\tiny $\alpha$};
                    \end{scope}
                \end{tikzpicture}
                % \caption*{}
                \label{fig:best_gen_robot.jpg}
            \end{figure}
            \vspace{-1cm}
            {\footnotesize
                \begin{eqnarray*}
                    % \resizebox{0.9\hsize}{!}{
                    F \rightarrow max = \beta \left( {\omega}_{1} \cdot \overbrace{\delta}^{\text{Дистанция}} + {\omega}_{2} \cdot \overbrace{\frac{1}{(\gamma - 1) h_{\text{leg}}sin(\alpha)}}^{\text{Упр. длина корпуса}}\right) +\\ \nonumber + (1 - \beta) {\delta}^{{\omega}_{1}} {\left( \frac{1}{(\gamma - 1)h_{\text{leg}}sin(\alpha)}\right)}^{{\omega}_{2}}
                    % }
                \end{eqnarray*}
            }
            % \vspace{1pt}

            $\beta$ -- адаптивный параметр, \\ ${\omega}_{1,2} \in  [ 0..1 ] $ -- весовые коэффициенты.
        \end{column}
    \end{columns}
\end{frame}

\begin{frame}[t]{Разработка робота}
    \framesubtitle{Видео: История одного сгенерированного робота}
    \vspace{-0.6cm}
    \begin{figure}[H]
        % \href{run:./videos/pass_rand_terr.mp4}{
        \href{https://youtu.be/DcovvkTZgsg}{
            \centering\includegraphics[height=6cm,width=1\textwidth,keepaspectratio]{genetic_video_preview.jpg}}
        % \caption{Click on a picture for a video}
    \end{figure}
\end{frame}

\begin{frame}[t]{Разработка робота}
    \framesubtitle{Конкретные результаты: $\omega_1 = 0.6$, $\omega_2 = 0.4$}
    \vspace{-0.6cm}

    \begin{table}[H]
        \centering
        \begin{tabular}{c|c|c|c|c}
                                              & \textbf{\begin{tabular}[c]{@{}c@{}}Тип\\ территории\end{tabular}}                                                              & \textbf{Кол-во ног}        & \textbf{\begin{tabular}[c]{@{}c@{}}Угол между\\ соседними ногами\end{tabular}} & \textbf{Кол-во индивидов} \\
            \hline
            \rule{0cm}{0.5cm}
            \textbf{Этап 1}                   &                                                                                                                                & \cellcolor[HTML]{DAE8FC}12 & 73                                                                             & 200                       \\ \cline{1-1} \cline{3-5}
                                              & \multirow{-2}{*}{\begin{minipage}{2.5cm}\includegraphics[height=3cm,width=2.5cm,keepaspectratio]{terrain_1.jpg}\end{minipage}} & \cellcolor[HTML]{DAE8FC}12 & 72                                                                             &                           \\ [0.5cm] \cline{3-4}
                                              & \begin{minipage}{2.5cm}\includegraphics[height=3cm,width=2.5cm,keepaspectratio]{terrain_2.jpg}\end{minipage}                   & \cellcolor[HTML]{DAE8FC}10 & 68                                                                             &                           \\ [0.5cm] \cline{3-4}
            \multirow{-3}{*}{\textbf{Этап 2}} & \begin{minipage}{2.5cm}\includegraphics[height=3cm,width=2.5cm,keepaspectratio]{terrain_3.jpg}\end{minipage}                   & \cellcolor[HTML]{DAE8FC}12 & 77                                                                             & \multirow{-3}{*}{55}
        \end{tabular}
        % \caption*{\large\centering\textbf{Summary}: created robot should have 10-12 legs in total}
    \end{table}

\end{frame}

\begin{frame}[t]{Разработка робота}
    \framesubtitle{Закономерность}
    \begin{columns}[T,onlytextwidth]
        \begin{column}{0.49\textwidth}
            Лучшие роботы в экспериментах начинались с 8 до 14 ног для различных значений $\omega$.

            Это объясняется критерием статического равновесия. В таком случае минимум 4 ноги всегда касаются поверхности.
        \end{column}
        \begin{column}{0.49\textwidth}
            \vspace{-1.8cm}
            \begin{figure}[H]
                \centering\includegraphics[height=5cm,width=1\textwidth,keepaspectratio]{box_plot_structural_synthesis.png}
                \caption*{Зависимость между кол-вом ног и пройденной дистанцией}
                \label{fig:box_plot_structural_synthesis.png}
            \end{figure}
        \end{column}
    \end{columns}
\end{frame}

\begin{frame}[t]{Разработка робота}
    \framesubtitle{Видео}
    \vspace{-0.6cm}
    \begin{figure}[H]
        % \href{run:./videos/sidestep_segments.mp4}{
        \href{https://youtu.be/EQ6oGZVDpoc}{
            \centering\includegraphics[height=6cm,width=1\textwidth,keepaspectratio]{sidestep_segment_video_preview.png}}
        % \caption{Click on a picture for a video}
    \end{figure}
\end{frame}

\begin{frame}[c]{}
    \framesubtitle{}
    \centering\LARGE Разработка и исследование преобразователя силы
\end{frame}

\begin{frame}[t]{Разработка преобразователя силы}
    \framesubtitle{}
    {\large\begin{block}{Вопрос}
            Как получить силу реакции опоры?
        \end{block}}
    {\large\begin{alertblock}{Ответ}
            \vspace{-0.2cm}

            \begin{itemize}
                \color{lightgray}
                \item Измерив ток/напряжение на моторе
                \item Установив датчик момента на вал мотора
                \item {\color{black} Установив датчик силы на ногу робота \\  \alert{\textbf{Пьезорезистивный датчик основанный на \underline{Velostat}}: дешевый и надежный, но имеет проблемы с гистерезисом}}
            \end{itemize}
        \end{alertblock}}
\end{frame}



\begin{frame}[t]{Разработка преобразователя силы}
    \framesubtitle{Velostat}
    \vspace{-15pt}
    \begin{columns}[T,onlytextwidth]
        \begin{column}{0.6\textwidth}
            Представляет собой полимерный материал, наполненный техническим углеродом.\\
            \textbf{Встреченные проблемы}:
            \begin{itemize}
                \item \underline{Гистерезис} -- зависимость от текущего и предыдущих состояний
                \item \underline{Нелинейность материала}
                \item \underline{Малая точность} при весе от 300 грамм
                \item \underline{Разность значений при одинаковом давлении}, когда площадь нажатия меньше датчика $\rightarrow$ \alert{Научная задача -- охарактеризовать материал для таких случаев}
            \end{itemize}
        \end{column}
        \begin{column}{0.38\textwidth}
            \vspace{-1.1cm}
            \begin{figure}[H]
                \begin{subfigure}{0.9\textwidth}
                    \centering\includegraphics[height=3cm,width=1\textwidth,keepaspectratio]{velostat_sensor.jpg}
                    % \caption*{Velostat material}
                    \label{fig:velostat_sensor.jpg}
                \end{subfigure}

                \begin{subfigure}{0.9\textwidth}
                    \centering\includegraphics[height=2.9cm,width=1\textwidth,keepaspectratio]{simplest_sensor.jpg}
                    \caption*{Простейший преобразователь силы}
                    \label{fig:simplest_sensor.jpg}
                \end{subfigure}
            \end{figure}
        \end{column}
    \end{columns}

\end{frame}

\begin{frame}[t]{Разработка преобразователя силы}
    \framesubtitle{Эксперименты}
    \vspace{-15pt}
    \begin{columns}[T,onlytextwidth]
        \begin{column}{0.6\textwidth}
            {\large
                \begin{enumerate}
                    \item \textbf{Статический}. Прикладывается статический груз с размером в сенсор
                          \item\textbf{Динамический}.
                          \begin{itemize}
                              \large
                              \item Чувствительная область представляется в виде сетки $4\times4$. Мы касаемся с одинаковым давлением, используя все 5 насадок
                              \item Используются насадки только 2 и 15 мм. Происходит нажатие с силой 5, 10, 20, 30, 40 H
                          \end{itemize}
                \end{enumerate}
            }
        \end{column}
        \begin{column}{0.39\textwidth}
            \vspace{-0.5cm}
            \begin{figure}[H]
                \centering
                \begin{tikzpicture}

                    % Include the image in a node
                    \node [
                        above right,
                        inner sep=0] (image) at (0,0) {\centering\includegraphics[height=5cm,width=1\textwidth,keepaspectratio]{sensors_grid.png}};

                    % Create scope with normalized axes
                    \begin{scope}[
                            x={($0.1*(image.south east)$)},
                            y={($0.1*(image.north west)$)}]

                        % Grid
                        % \draw[lightgray,step=1] (image.south west) grid (image.north east);

                        % % Axes' labels
                        % \foreach \x in {0,1,...,10} { \node [below] at (\x,0) {\x}; }
                        % \foreach \y in {0,1,...,10} { \node [left] at (0,\y) {\y};}

                        % Labels
                        % Simple brace
                        \draw [green, very thick,
                            decorate,
                            decoration = {brace,
                                    raise=5pt,
                                    amplitude=5pt,
                                    aspect=0.5}] (6,3.7) --  (3,3.7)
                        node[pos=0.5,below=10pt,green]{$15\ \text{мм}$};

                        \draw [green, very thick,
                            decorate,
                            decoration = {brace, mirror,
                                    raise=5pt,
                                    amplitude=5pt,
                                    aspect=0.5}] (6,3.6) --  (6,6.4)
                        node[pos=0.5,right=10pt,green]{$15\ \text{мм}$};

                        \draw[green,step=1,xshift=34, yshift=43]  (0.5,0.5) grid +(3,3);

                        \node[circle,fill=green,scale=0.4] at (3.3,6.27){\small 1};
                        \node[circle,fill=green,scale=0.4] at (5.92,3.7){\small 16};
                    \end{scope}

                \end{tikzpicture}
                \caption*{Поверхность \\ как $4\times4$ сетка}
                \label{fig:file_name}
            \end{figure}
        \end{column}
    \end{columns}
\end{frame}

\begin{frame}[t]{Разработка преобразователя силы}
    \framesubtitle{Результаты: Статический эксперимент}
    \vspace{-0.5cm}
    \begin{columns}[T,onlytextwidth]
        \begin{column}{0.52\textwidth}
            \begin{eqnarray*}
                V_{out} = V_0 + p[k_p + k_e(1-e^\frac{-(t-t_0)}{\tau_{res}})](1-e^{-\frac{A}{p}}) \\
                k_p = A_1e^{-A_2p}; \tau_{res} = B_0 + B_1e^{-\frac{p}{B_2}}
            \end{eqnarray*}
            Где $V_0$ -- начальное напряжение, \\ $p$ -- приложенное давление, \\ $A_i,\ B_i,\ \tau_{res},\ k_i$ искомые параметры, \\  $t$ -- текущее время, $t_0$ -- время начала нажатия.
            \\ \alert{Апробированна модель для калибровки датчика}
        \end{column}
        \begin{column}{0.45\textwidth}
            \vspace{-15pt}
            \begin{figure}[H]
                \begin{subfigure}{0.99\textwidth}
                    \centering\includegraphics[height=2.8cm,width=1\textwidth,keepaspectratio]{least_square_model.png}
                    \label{fig:least_square_model.png}
                \end{subfigure}
                \vspace{-1cm}

                \begin{subfigure}{0.99\textwidth}
                    \centering
                    \begin{tikzpicture}
                        % Include the image in a node
                        \node [above right, inner sep=0] (image) at (0,0)
                        {\centering\includegraphics[height=3.4cm,width=1\textwidth,keepaspectratio]{static_load_meh.JPG}};
                        % Create scope with normalized axes
                        \begin{scope}[
                                x={($ 0.1*(image.south east)$)},
                                y={($ 0.1*(image.north west)$)}]
                            % Grid and axes' labels
                            % \draw[lightgray,step=1] (image.south west) grid (image.north east);
                            % \foreach \x in {0,1,...,10} { \node [below] at (\x,0) {\x}; }
                            % \foreach \y in {0,1,...,10} { \node [left] at (0,\y) {\y};}
                            % Labels
                            \draw[latex-, very thick,green] (4.3,2.3) -- (5,1.6)
                            node[rounded corners=3pt,below right,black,fill=white]{\tiny Исследуемый датчик};

                            \draw[latex-, very thick,green] (4.3,3.5) -- (5.5,2.45)
                            node[rounded corners=3pt,right,black,fill=white]{\tiny \O \ 15 мм насадка};

                            \draw[latex-, very thick,green] (6,6) -- (6.4,4.9)
                            node[rounded corners=3pt,below right,black,fill=white]{\tiny Известная нагрузка};
                        \end{scope}
                    \end{tikzpicture}
                    % \caption*{}
                \end{subfigure}
            \end{figure}
        \end{column}
    \end{columns}
\end{frame}

\begin{frame}[t]{Разработка преобразователя силы}
    \framesubtitle{Требования к установке}
    \vspace{-0.5cm}
    \begin{columns}[T,onlytextwidth]
        \begin{column}{0.39\textwidth}
            \begin{itemize}
                \item Управление силой нажатия {\\ \alert{Импедансное управления}}
                \item Повторяемость эксперимента по силе и позиции{\\ \alert{Добавив манипулятор и камеру}}
                \item Возможность нажимать только на часть сенсора{\\ \alert{Насадки для манипулятора}}
            \end{itemize}
        \end{column}
        \begin{column}{0.8\textwidth}
            \scalebox{0.81}{
                \begin{tikzpicture}

                    % Include the image in a node
                    \node [
                        above right,
                        inner sep=0] (image) at (0,0) {\centering\includegraphics[height=6cm,width=1\textwidth,keepaspectratio]{exp_stand1}};

                    % Create scope with normalized axes
                    \begin{scope}[
                            x={($0.1*(image.south east)$)},
                            y={($0.1*(image.north west)$)}]

                        % Grid
                        % \draw[lightgray,step=1] (image.south west) grid (image.north east);

                        % % Axes' labels
                        % \foreach \x in {0,1,...,10} { \node [below] at (\x,0) {\x}; }
                        % \foreach \y in {0,1,...,10} { \node [left] at (0,\y) {\y};}

                        % Labels
                        % \node[circle,fill=green] at (7.25,6.75){\small 2};

                        \draw[latex-, very thick,green] (3.5,2.2) -- (3.5,1)
                        node[rounded corners=3pt,below left,black,fill=white]{\small Исследуемые датчики};

                        \draw[stealth-, very thick,green] (3.5,2.6) -- ++(-0.7,+0.5)
                        node[rounded corners=3pt,left,black,fill=white]{\small Датчик силы};

                        \draw[stealth-, very thick,green] (6.5,3) -- (7,6)
                        node[rounded corners=3pt,above right,black,fill=white]{\small Печатная плата};

                        \draw[stealth-, very thick,green] (7.2,1.5) -- (8,5)
                        node[rounded corners=3pt,above right,black,fill=white]{\small Контроллер};

                        \draw[stealth-, very thick,green] (2.5,9.5) -- (4,9.5)
                        node[rounded corners=3pt,right,black,fill=white]{\small Камера};

                        \draw[very thick,green] (0.5,2.5) rectangle (4.2,9)
                        node[below left,black,fill=green]{\small UR10e};

                        \draw[latex-, very thick,green] (4.5,7.2) edge (5.5,7.5)
                        (4.8,5.3) -- (5.5,7.5)
                        node[rounded corners=3pt,above,black,fill=white]{\small Маркеры Аруко};
                    \end{scope}

                \end{tikzpicture}
            }
        \end{column}
    \end{columns}

\end{frame}

\begin{frame}[t]{Разработка преобразователя силы}
    \framesubtitle{Установка: Насадки}
    \vspace{-0.9cm}
    \begin{columns}[T,onlytextwidth]
        \begin{column}{0.6\textwidth}
            \begin{figure}[H]
                \centering
                \begin{tikzpicture}
                    % Include the image in a node
                    \node [above right, inner sep=0] (image) at (0,0)
                    {\centering\includegraphics[height=5cm,width=1\textwidth,keepaspectratio]{all_end_effectors.png}};
                    % Create scope with normalized axes
                    \begin{scope}[
                            x={($ 0.1*(image.south east)$)},
                            y={($ 0.1*(image.north west)$)}]
                        % Grid and axes' labels
                        % \draw[lightgray,step=1] (image.south west) grid (image.north east);
                        % \foreach \x in {0,1,...,10} { \node [below] at (\x,0) {\x}; }
                        % \foreach \y in {0,1,...,10} { \node [left] at (0,\y) {\y};}

                        % Labels
                        \node[rounded corners=3pt,black,fill=white] at (1.1,7.4){\tiny 2 мм };
                        \node[rounded corners=3pt,black,fill=white] at (3.1,7.9){\tiny 6 мм };
                        \node[rounded corners=3pt,black,fill=white] at (4.9,8.1){\tiny 8 мм };
                        \node[rounded corners=3pt,black,fill=white] at (6.7,7.9){\tiny 12 мм };
                        \node[rounded corners=3pt,black,fill=white] at (8.6,7.9){\tiny 15 мм };
                    \end{scope}
                \end{tikzpicture}
                \caption*{Все насадки}
                \label{fig:all_end_effectors.png}
            \end{figure}
        \end{column}
        \begin{column}{0.39\textwidth}
            \vspace{-1.3cm}

            \begin{figure}[H]
                \begin{subfigure}[t]{0.6\textwidth}
                    \centering\includegraphics[width=0.99\textwidth]{LTH350-DONUT-LOAD-CELL-1.png}\\
                    \caption*{\normalsize Промышленный \\ датчик силы}
                    \label{fig:futek}
                \end{subfigure}
                \vspace{-0.2cm}

                \begin{subfigure}[t]{0.6\textwidth}
                    \centering\includegraphics[width=0.99\textwidth]{point_load.JPG}\\
                    \caption*{\normalsize Насадка в сборке}
                    \label{fig:point_load}
                \end{subfigure}
            \end{figure}

        \end{column}
    \end{columns}
\end{frame}

\begin{frame}[t]{Разработка преобразователя силы}
    \framesubtitle{Установка: Видео}
    \vspace{-15pt}
    \begin{figure}[H]
        % \href{run:./videos/exp_stand_video.mp4}{
        \href{https://youtu.be/Gw4wVZ-ESuE}{
            \centering\includegraphics[height=6cm,width=1\textwidth,keepaspectratio]{exp_stand_video_preview.jpg}}
        % \caption{caption_name}
    \end{figure}
\end{frame}

\begin{frame}[t]{Разработка преобразователя силы}
    \framesubtitle{Импедансное управление}
    \vspace{-18pt}
    \begin{columns}[T,onlytextwidth]
        \begin{column}{0.55\textwidth}
            \begin{exampleblock}{Модификация траектории (Только ось $z$)}
                \vspace{-14pt}
                \begin{eqnarray*}
                    X_s^0 = 0, \dot{X}_s^0 =0,  X_g^k, \dot{X}_g^k \text{ -- целевое состояние}\\, X_s = X_g - X_d \\
                    X_g = X_g^0 + \frac{F_d}{\eta } \\
                    \dot{X}_s + \eta  X_s = F^k \\
                    X_s^k = odeint(X_s^{k-1},t,F^k), t = [0,dT] \\
                    X_s^{k-1} = X_s^k;  \dot{X}_s = f(X_s,t,F^k) \\
                    X_d = X_g - X_s; \dot{X}_d = \dot{X}_g - \dot{X}_s
                \end{eqnarray*}
            \end{exampleblock}
        \end{column}
        \begin{column}{0.40\textwidth}
            \begin{exampleblock}{Управление по скорости}
                \vspace{-14pt}
                \begin{eqnarray*}
                    X_d = \begin{bmatrix}
                        x_g \\ y_g \\ z_d
                    \end{bmatrix} \\
                    U = \dot{X}_d + K(X_d - X), \\ \text{ где } X=\text{принять\_состояние}(); \\
                    \text{установить\_скорость}(U)
                \end{eqnarray*}
            \end{exampleblock}
        \end{column}
    \end{columns}
\end{frame}

\begin{frame}[t]{Разработка преобразователя силы}
    \framesubtitle{Импедансное управление: пример результата}
    \vspace{-15pt}
    \begin{figure}[H]
        \centering\includegraphics[height=6.1cm,width=1\textwidth,keepaspectratio]{force_data_pos.png}
        % \caption{caption_name}
        \label{fig:impedance}
    \end{figure}
\end{frame}


\begin{frame}[t]{Разработка преобразователя силы}
    \framesubtitle{Результаты: Ошибки показаний датчика в динамическом эксперименте}
    \vspace{-15pt}
    \begin{figure}[H]
        \begin{subfigure}{0.64\textwidth}
            \centering\includegraphics[height=5cm,width=1\textwidth,keepaspectratio]{sens1_pike1_mod.png}
            \caption*{2 мм диаметр насадки}
            \label{fig:sens1_pike1}
        \end{subfigure}
        \begin{subfigure}{0.34\textwidth}
            \centering\includegraphics[height=5cm,width=1\textwidth,keepaspectratio]{sens1_pike3.png}
            \caption*{8 мм диаметр насадки}
            \label{fig:sens1_pike3}
        \end{subfigure}
    \end{figure}
    \vspace{-0.8cm}
    \alert{Одинаковые данные, когда площадь нажатия превышает 25\% от площади датчика}
\end{frame}

\begin{frame}[c]{}
    \framesubtitle{}
    \centering\LARGE Определение физических свойств поверхности
\end{frame}

\begin{frame}[t]{Определение физических свойств поверхности}
    \framesubtitle{}
    \vspace{-0.3cm}
    {\large\begin{block}{Вопрос}
            Как определить физические свойства местности во время движения по ней?
        \end{block}}
    {\large\begin{alertblock}{Ответ}
            1. Подготовить для экспериментов различные поверхности.

            2. Собрать датасет, состоящий из угловой скорости мотора и показаний датчиков с ног робота.

            3. Представить их в виде вектора признаков.

            4. Решить задачу классификации данных с помощью SVM, используя метрику 10-fold cross validation.

            5. Протестировать модель на собранных данных.
        \end{alertblock}}
\end{frame}

\begin{frame}[t]{Определение физических свойств поверхности}
    \framesubtitle{Требования к установке}
    \vspace{-0.5cm}
    \begin{columns}[T,onlytextwidth]
        \begin{column}{0.57\textwidth}
            \begin{itemize}
                \item Иметь возможность быстро менять используемые поверхности {\\ \alert{Быстроразборный стол}}
                \item Бесконечное движение робота {\\ \alert{2-ух степенной механизм и нога S-образной формы}}
                \item Узел движителя должен быть такой же как на СтриРусе {\\ \alert{Создано крепление для узла ноги робота}}
            \end{itemize}
        \end{column}
        \begin{column}{0.42\textwidth}
            \vspace{-0.8cm}
            \begin{figure}[H]
                \centering
                \begin{tikzpicture}
                    % Include the image in a node
                    \node [above right, inner sep=0] (image) at (0,0)
                    {\centering\includegraphics[height=6cm,width=1\textwidth,keepaspectratio]{s_shape_leg/s_leg_setup.JPG}};
                    % Create scope with normalized axes
                    \begin{scope}[
                            x={($ 0.1*(image.south east)$)},
                            y={($ 0.1*(image.north west)$)}]
                        \draw[stealth-, very thick,green] (3.5,2.5) -- (3,1.5)
                        node[rounded corners=3pt,below,black,fill=white]{\tiny Стол для поверхностей};

                        \draw[stealth-, very thick,green] (7.1,5.4) -- (7.4,7)
                        node[rounded corners=3pt,above right,black,fill=white]{\tiny Контроллер};

                        \draw[very thick,green] (6,6.1) rectangle (8.5,3.5)
                        node[above left,black,fill=green]{\tiny S leg};
                    \end{scope}
                \end{tikzpicture}
                % \caption*{}
                \label{fig:s_shape_leg/s_leg_setup.JPG}
            \end{figure}
        \end{column}
    \end{columns}
\end{frame}

\begin{frame}[t]{Определение физических свойств поверхности}
    \framesubtitle{Установка сенсоров на ногу}
    \vspace{-0.7cm}
    \begin{figure}[H]
        \begin{subfigure}{0.49\textwidth}
            \centering\includegraphics[height=6cm,width=1\textwidth,keepaspectratio]{s_shape_leg/socks.jpg}
            % \caption{capture1}
            \label{fig:s_shape_leg/socks.jpg}
        \end{subfigure}
        \begin{subfigure}{0.49\textwidth}
            \centering\includegraphics[height=6cm,width=1\textwidth,keepaspectratio]{s_shape_leg/leg_design.png}
            % \caption{capture2}
            \label{fig:s_shape_leg/leg_design.png}
        \end{subfigure}
    \end{figure}
\end{frame}


\begin{frame}[t]{Определение физических свойств поверхности}
    \framesubtitle{Установка: Типы поверхности, видео}
    \vspace{-15pt}
    \begin{figure}[H]
        \begin{subfigure}{0.22\textwidth}
            % \href{run:./videos/flat.gif}
            \href{https://gifyu.com/image/SxatY}
            {\centering\includegraphics[height=6cm,width=1\textwidth,keepaspectratio]{s_shape_leg/flat.jpg}}
            \caption*{Упругая}
        \end{subfigure}
        \hfill
        \begin{subfigure}{0.26\textwidth}
            % \href{run:./videos/rock.gif}
            \href{https://gifyu.com/image/Sxatt}
            {\centering\includegraphics[height=6cm,width=1\textwidth,keepaspectratio]{s_shape_leg/view.jpg}}
            \caption*{Твердая}
        \end{subfigure}
        \begin{subfigure}{0.5\textwidth}
            {\centering\includegraphics[height=6cm,width=1\textwidth,keepaspectratio]{s_shape_leg/mould.jpg}}
            \caption*{Пластичная}
        \end{subfigure}
    \end{figure}
\end{frame}


\begin{frame}[t]{Определение физических свойств поверхности}
    \framesubtitle{Данные с одного эксперимента}
    \begin{columns}[T,onlytextwidth]
        \begin{column}{0.60\textwidth}
            \begin{figure}[H]
                \centering\includegraphics[height=5cm,width=1\textwidth,keepaspectratio]{s_shape_leg/TaxelIndForce.png}
            \end{figure}
        \end{column}
        \begin{column}{0.35\textwidth}
            \vspace{-1.4cm}
            \begin{figure}[H]
                \centering\includegraphics[height=3.8cm,width=1\textwidth,keepaspectratio]{s_shape_leg/avg_lin_vel_rev_min.png}
            \end{figure}

            \resizebox{\linewidth}{!}{%
                \begin{tabular}{|c|c|c|c|c|}
                    \cline{3-5}
                    \multicolumn{1}{l}{}                           & \multicolumn{1}{l|}{} & \multicolumn{3}{c|}{\textbf{Predicted Class}}                                                                                           \\
                    \cline{3-5}
                    \multicolumn{1}{l}{}                           &                       & Rubber                                        & Rock                                       & Ground                                     \\
                    \hline
                    \multirow{3}{*}{\rotcell{\textbf{True class}}} & Rubber                & {\cellcolor[rgb]{0.741,0.843,0.929}}84.0\%    & 2.56\%                                     & 13.44\%                                    \\
                    \hhline{|~----|}
                                                                   & Rock                  & 20.1\%                                        & {\cellcolor[rgb]{0.741,0.843,0.929}}67.8\% & 12.1\%                                     \\
                    \hhline{|~----|}
                                                                   & Ground                & 1.0\%                                         & 18.9\%                                     & {\cellcolor[rgb]{0.741,0.843,0.929}}80.1\% \\
                    \hline
                \end{tabular}
            }

        \end{column}
    \end{columns}
\end{frame}

% \begin{frame}[t]{Определение физических свойств поверхности}
%     \framesubtitle{Итог}
%     \large
%     \begin{itemize}
%         \item Возможность различать резиновую и каменистую поверхность.
%         \item Выбраны параметры классификации рельефа для машинного обучения:
%               \begin{itemize}
%                 \large
%                   \item Число оборотов в минуту
%                   \item Крутящий момент двигателя
%                   \item Ускорение от IMU
%                   \item Данные о силе, которые представлены как значение датчика\/сегмент, пиковая амплитуда, средняя амплитуда
%               \end{itemize}
%         \item Velostat датчик силы доказал свою работоспособность.
%     \end{itemize}
% \end{frame}

\begin{frame}[c]{}
    \framesubtitle{}
    \centering\LARGE Определение геометрических свойств поверхности
\end{frame}

\begin{frame}[t]{Определение геометрических свойств поверхности}
    \framesubtitle{}
    {\large\begin{block}{Вопрос}
            Как создать плотное облако точек, используя следовую дорожку?
        \end{block}}
    {\large\begin{alertblock}{Ответ}
            1. \textit{Создать полигональную сетку}, используя 2D триангуляцию Делоне (вогнутая оболочка) с использованием разреженных данных

            2. \textit{Сгенерировать новые точки} из полигональной сетки

            3. Вернуть плотное облако точек навигации навигации
        \end{alertblock}}
\end{frame}

\begin{frame}[t]{Определение геометрических свойств поверхности}
    \framesubtitle{Места проведения экспериментов}
    \vspace{-15pt}
    \begin{figure}[H]
        \begin{subfigure}[t]{0.49\textwidth}
            \centering\includegraphics[height=5cm,width=1\textwidth,keepaspectratio]{coppelia_sim.png}
            \caption*{CoppeliaSim симулятор,\\ \textbf{4th gen} СтриРус}
        \end{subfigure}
        \begin{subfigure}[t]{0.49\textwidth}
            \centering\includegraphics[height=5cm,width=1\textwidth,keepaspectratio]{rl_sim.JPG}
            \caption*{Натурные испытания,\\ \textbf{3th+ gen} СтриРус}
        \end{subfigure}
    \end{figure}
\end{frame}

\begin{frame}[t]{}
    \framesubtitle{}
    \vspace{-0.6cm}
    \begin{figure}[H]
        \centering
        \begin{tikzpicture}
            % Include the image in a node
            \node [above right, inner sep=0] (image) at (0,0)
            {\centering\includegraphics[height=9cm,width=1\textwidth,keepaspectratio]{StriRus_10_legs_15_angle_v4.png}};
            % Create scope with normalized axes
            \begin{scope}[
                    x={($ 0.1*(image.south east)$)},
                    y={($ 0.1*(image.north west)$)}]
                % Grid and axes' labels
                % \draw[lightgray,step=1] (image.south west) grid (image.north east);
                % \foreach \x in {0,1,...,10} { \node [below] at (\x,0) {\x}; }
                % \foreach \y in {0,1,...,10} { \node [left] at (0,\y) {\y};}

                % Labels

                \coordinate (Xc) at (0.4415/2,-0.2347/2);
                \coordinate (Yc) at (-0.4512/2,-0.2156/2);
                \coordinate (Zc) at (0,0.5/2);
                % Labels
                \tikzstyle{origin} = [rounded corners=2pt, black, fill=gray!40, fill opacity=0.75, text opacity=1, scale=0.8,inner sep=1pt]
                \tikzstyle{transform_text} = [rounded corners=2pt, black, fill=white!85!gray, fill opacity=0.75, text opacity=1, scale=0.8,inner sep=1pt]
                \tikzstyle{transform_arrow} = [thick, green]

                % \coordinate (o_g) at (1,9);
                \node[circle,fill=green,scale=0.25] (o_g) at (1,9){};
                \draw[-stealth, very thick,blue] (o_g) -- ++(Xc);
                \draw[-stealth, very thick,green!70!black] (o_g) -- ++(Yc);
                \draw[-stealth, very thick,red] (o_g) -- ++(Zc);
                \node[origin,above right=3pt] at (o_g){\tiny $\mathbf{O_{glob}}$};

                % \coordinate (o_b) at (2.9,7.05);
                \node[circle,fill=green,scale=0.25] (o_b) at (2.9,7.05){};
                \draw[-stealth, very thick,blue] (o_b) -- ++(Xc);
                \draw[-stealth, very thick,green!70!black] (o_b) -- ++(Yc);
                \draw[-stealth, very thick,red] (o_b) -- ++(Zc);
                \node[origin,above right=2pt] at (o_b){\tiny $\mathbf{O_{base}}$};

                % \coordinate (o_1) at (2.9,6.6);
                \node[circle,fill=green,scale=0.25] (o_1) at (2.9,6.6){};
                \draw[-stealth, very thick,blue] (o_1) -- ++(Xc);
                \draw[-stealth, very thick,green!70!black] (o_1) -- ++(Yc);
                \draw[-stealth, very thick,red] (o_1) -- ++(Zc);
                \node[origin,above right=2pt] at (o_1){\tiny $\mathbf{O_{1}}$};

                % \coordinate (o_2) at (4,6);
                \node[circle,fill=green,scale=0.25] (o_2) at (4,6){};
                \draw[-stealth, very thick,blue] (o_2) -- ++(Xc);
                \draw[-stealth, very thick,green!70!black] (o_2) -- ++(Yc)
                node[origin,below=2pt]{\tiny $\mathbf{\alpha_3}$};
                \draw[-stealth, very thick,red] (o_2) -- ++(Zc);
                \node[origin,above right=2pt] at (o_2){\tiny $\mathbf{O_{2}=O_{3}}$};

                % \coordinate (o_4) at (7.0,4.55);
                \node[circle,fill=green,scale=0.25] (o_4) at (7.0,4.55){};
                \draw[-stealth, very thick,blue] (o_4) -- ++(Xc);
                \draw[-stealth, very thick,green!70!black] (o_4) -- ++(Yc);
                \draw[-stealth, very thick,red] (o_4) -- ++(Zc);
                \node[origin,above right=2pt] at (o_4){\tiny $\mathbf{O_{4}}$};

                % \coordinate (o_5) at (6.7,4.45);
                \node[circle,fill=green,scale=0.25] (o_5) at (6.7,4.45){};
                \draw[-stealth, very thick,blue] (o_5) -- ++(Xc);
                \draw[-stealth, very thick,green!70!black] (o_5) -- ++(Yc);
                \draw[-stealth, very thick,red] (o_5) -- ++(Zc);
                \node[origin,above left=3pt] at (o_5){\tiny $\mathbf{O_{5}=O_{6}}$};

                \coordinate (Xcr) at (0.49/2,0.07/2);
                % \coordinate (Ycr) at (-0.38/2,-0.32/2);
                \coordinate (Ycr) at (-0.24/2,-0.43/2);

                \draw[-stealth, very thick,blue] (o_5) -- ++(Xcr);
                \draw[-stealth, very thick,green!70!black] (o_5) -- ++(Ycr);


                % \coordinate (o_7) at (6.36,3.68);
                \node[circle,fill=green,scale=0.25] (o_7) at (6.36,3.68){};
                \draw[-stealth, very thick, blue] (o_7) -- ++(Xcr);
                \draw[-stealth, very thick, green!70!black] (o_7) -- ++(Ycr)
                node[origin,above left=2pt]{\tiny $\mathbf{\alpha_8}$};
                \draw[-stealth, very thick, red] (o_7) -- ++(Zc);
                \node[origin,below right=3pt] at (o_7){\tiny $\mathbf{O_{7}=O_{8}}$};

                \node[circle, draw ,fill=green,scale=0.4] (s_1) at (6.6,1.3){1};
                \node[circle,draw, fill=green,scale=0.4] (s_3) at (5.85,1.7){3};
                \node[circle,draw, fill=green,scale=0.4] (s_5) at (5.55,2.9){5};

                \draw[-stealth, transform_arrow] (o_g) -- (o_b)
                node[midway,below left=2pt, transform_text]{\tiny $\mathbf{H_{base}^{glob}}$};

                \draw[-stealth, transform_arrow] (o_b) -- (o_1)
                node[midway,left=3pt, transform_text]{\tiny $\mathbf{H_{1}^{base}}$};

                \draw[-stealth, transform_arrow] (o_1) -- (o_2)
                node[midway,below=2pt, transform_text]{\tiny $\mathbf{H_{2}^{1}}$};

                \draw[-stealth, transform_arrow] (o_2) -- (o_4)
                node[midway,below=2pt, transform_text]{\tiny $\mathbf{H_{4}^{3}}$};

                \draw[-stealth, transform_arrow] (o_4) -- (o_5)
                node[midway,below right=2pt, transform_text]{\tiny $\mathbf{H_{5}^{4}}$};

                \draw[-stealth, transform_arrow] (o_5) -- (o_7)
                node[midway,left=3pt, transform_text]{\tiny $\mathbf{H_{7}^{6}}$};

                \draw[-stealth, transform_arrow] (o_7) -- (s_1);
                \draw[-stealth, transform_arrow] (o_7) -- (s_3);
                \draw[-stealth, transform_arrow] (o_7) -- (s_5);
            \end{scope}
        \end{tikzpicture}
    \end{figure}
\end{frame}

\begin{frame}[t]{Определение геометрических свойств поверхности}
    \framesubtitle{Триангуляция Делоне}
    \vspace{-0.2cm}
    \begin{figure}[H]
        \centering\includegraphics[height=6cm,width=1\textwidth,keepaspectratio]{delone_idea.png}
        \caption*{2D триангуляция Делоне (Выпуклая оболочка) \\ \textbf{От облака точек к полигональной сетке}}
        \label{fig:delone_idea.png}
    \end{figure}
\end{frame}

\begin{frame}[t]{Определение геометрических свойств поверхности}
    \framesubtitle{Почему важно использовать вогнутую оболочку (модификация Делоне)}
    \vspace{-15pt}
    \begin{figure}[H]
        \begin{subfigure}[t]{0.3\textwidth}
            \centering\includegraphics[height=5cm,width=1\textwidth,keepaspectratio]{convex_terr.png}
            \caption*{Пример поверхности}
            \label{fig:convex_terr.png}
        \end{subfigure}
        \hfill
        \begin{subfigure}[t]{0.33\textwidth}
            \centering
            \begin{tikzpicture}
                % Include the image in a node
                \node [above right, inner sep=0] (image) at (0,0)
                {\centering\includegraphics[height=6cm,width=1\textwidth,keepaspectratio]{conv_convex.png}};
                % Create scope with normalized axes
                \begin{scope}[
                        x={($ 0.1*(image.south east)$)},
                        y={($ 0.1*(image.north west)$)}]
                    % Grid and axes' labels
                    % \draw[lightgray,step=1] (image.south west) grid (image.north east);
                    % \foreach \x in {0,1,...,10} { \node [below] at (\x,0) {\x}; }
                    % \foreach \y in {0,1,...,10} { \node [left] at (0,\y) {\y};}

                    % Labels
                    \draw[stealth-, very thick,green] (5.2,3.5) -- ++(0,-1)
                    node[rounded corners=3pt,right,black,fill=white]{\tiny Полученная сетка};

                    \draw[stealth-, very thick,green] (5.5,5.5) -- (6.4,4)
                    node[rounded corners=3pt,right,black,fill=white]{\tiny Данные лидара};


                    \draw[stealth-, very thick,green] (3.4,0.8) -- (5,1);
                    \draw[stealth-, very thick,green] (3.4,2.6) -- (5,1)
                    node[rounded corners=3pt,right,black,fill=white]{\tiny Следовая дорожка};
                \end{scope}
            \end{tikzpicture}
            \caption*{Выпуклая оболочка}
            \label{fig:conv_convex.png}
        \end{subfigure}
        \hfill
        \begin{subfigure}[t]{0.33\textwidth}
            \centering\includegraphics[height=6cm,width=1\textwidth,keepaspectratio]{conv_concave.png}
            \caption*{Вогнутая оболочка}
            \label{fig:conv_concave.png}
        \end{subfigure}

    \end{figure}
\end{frame}

\begin{frame}[t]{Определение геометрических свойств поверхности}
    \framesubtitle{Результат: Маршрут, полигональная сетка}
    \vspace{-15pt}
    \begin{figure}[H]
        \begin{subfigure}[t]{0.36\textwidth}
            \centering\includegraphics[height=5cm,width=1\textwidth,keepaspectratio]{terrain_wo_water.png}
            \caption*{Начало маршрута}
        \end{subfigure}
        \begin{subfigure}[t]{0.36\textwidth}
            \centering\includegraphics[height=5cm,width=1\textwidth,keepaspectratio]{terrain_w_water_end.png}
            \caption*{Конец маршрута}
        \end{subfigure}
        \begin{subfigure}[t]{0.26\textwidth}
            \centering\includegraphics[height=5cm,width=1\textwidth,keepaspectratio]{mesh_rviz.png}
            \caption*{Созданная сетка}
        \end{subfigure}
    \end{figure}


\end{frame}

\begin{frame}[t]{Определение геометрических свойств поверхности}
    \framesubtitle{Результаты Cloud2Cloud и Cloud2Mesh}
    \vspace{-15pt}
    \begin{figure}[H]
        \begin{subfigure}[t]{0.49\textwidth}
            \centering
            \begin{tikzpicture}
                % Include the image in a node
                \node [above right, inner sep=0] (image) at (0,0)
                {\centering\includegraphics[height=2.8cm,width=1\textwidth,keepaspectratio]{sampled_pcd.png}};
                % Create scope with normalized axes
                \begin{scope}[
                        x={($ 0.1*(image.south east)$)},
                        y={($ 0.1*(image.north west)$)}]
                    % Grid and axes' labels
                    % \draw[lightgray,step=1] (image.south west) grid (image.north east);
                    % \foreach \x in {0,1,...,10} { \node [below] at (\x,0) {\x}; }
                    % \foreach \y in {0,1,...,10} { \node [left] at (0,\y) {\y};}

                    % Labels
                    \draw[stealth-, very thick,green] (3,8) -- (2,8.5);
                    \draw[stealth-, very thick,green] (1,5.5) -- (2,8.5)
                    node[rounded corners=3pt,above,black,fill=white]{\tiny Ground Truth Point Cloud};

                    \draw[stealth-, very thick,green] (5.5,3) -- (5.5,8.5)
                    node[rounded corners=3pt,above,black,fill=white]{\tiny Generated Point Cloud};
                \end{scope}
            \end{tikzpicture}
            % \caption*{Наложенные облака точек}
            \label{fig:sampled_pcd.png}
        \end{subfigure}
        \begin{subfigure}[t]{0.49\textwidth}
            \centering\includegraphics[height=2.8cm,width=1\textwidth,keepaspectratio]{mesh_comp.png}
            % \caption*{Наложенные сетки}
        \end{subfigure}

        \begin{subfigure}[t]{0.49\textwidth}
            \centering\includegraphics[height=2.6cm,width=1\textwidth,keepaspectratio]{pcd_hist.png}
            \caption*{Гистограмма ошибок C2C}
        \end{subfigure}
        \begin{subfigure}[t]{0.49\textwidth}
            \centering\includegraphics[height=2.6cm,width=1\textwidth,keepaspectratio]{mesh_hist.png}
            \caption*{Гистограмма ошибок C2M}
        \end{subfigure}
    \end{figure}
\end{frame}

\begin{frame}[t]{Определение геометрических свойств поверхности}
    \framesubtitle{Результат: Натурные испытания, Видео}
    \vspace{-0.5cm}
    \begin{figure}[H]
        \begin{subfigure}[t]{0.49\textwidth}
            % \href{run:./videos/big_angle2.mp4}{
            \href{https://youtu.be/2dxHHTG4psQ}{
                \centering\includegraphics[height=6cm,width=1\textwidth,keepaspectratio]{real_robot_mesh_video_preview.png}}
            \caption*{Робот проходит препятствие}
        \end{subfigure}
        \begin{subfigure}[t]{0.49\textwidth}
            \centering\includegraphics[height=6cm,width=1\textwidth,keepaspectratio]{real_mesh.jpg}
            \caption*{Полигональная сетка, полученная с помощью ног}
        \end{subfigure}
    \end{figure}
\end{frame}

% \begin{frame}[t]{Определение геометрических свойств поверхности}
%     \framesubtitle{Итог}
%     \large
%     \begin{itemize}
%         \item Карта может быть построена с помощью \textit{вогнутой оболочки 2D триангуляции Делоне}, где входными данными являются \textit{точки касания, определенные датчиком силы}.
%         \item \textit{Симулятор} (Среднее значение среднеквадратичной ошибки): \begin{itemize}
%             \large
%                   \item При сравнении облаков точек составляет около 5 см.
%                   \item При сравнении сеткок составляет около 1 см.
%               \end{itemize}
%         \item \textit{Натурный эксперимент} (---//---): \begin{itemize}
%             \large
%                   \item При сравнении облаков точек составляет около 8 см.
%                         % \item Среднее значение. RMSE при сравнении сетки составляет около 1 см.
%               \end{itemize}
%               \textit{Это приемлимая точность для такой задачи.}
%     \end{itemize}
% \end{frame}


\begin{frame}[t]{Результаты решения задач}
    \framesubtitle{}
    \vspace{-0.7cm}
    \begin{columns}[T,onlytextwidth]
        \begin{column}{0.48\textwidth}
            \begin{block}{Научных задач (научная новизна)}
                1. Метод \textbf{подбора количества ног для шагающих цикловых движителей}.

                2. Методика \textbf{характеризации датчика}, когда площадь касания нагрузки меньше, чем размеры датчика.

                3. Алгоритмы \textbf{калибровки} и \textbf{определения физических свойств поверхности}.

                4. Метод определения \textbf{геометрических свойств местности}.

            \end{block}
        \end{column}
        \begin{column}{0.48\textwidth}
            \begin{alertblock}{Экспериментальных разработок}
                1. Спроектированы и собраны 2 прототипа с \textbf{Шагающим цикловым движителем} с одной степенью свободы в ноге.

                2. Разработана и создана \textbf{экспериментальная установка} для \textbf{автоматизированного исследования датчика силы}.

                3. Разработана и создана \textbf{экспериментальная установка} для \textbf{определения типа поверхности}.

            \end{alertblock}
        \end{column}
    \end{columns}
\end{frame}

\begin{frame}[t]{Апробация работы}
    \framesubtitle{Статьи в периодических изданиях по перечню ВАК РФ}
    \begin{enumerate}
        \item Буличев О. В., Полёткин К. В., Малолетов А. В. Исследование характеристик датчика силы на основе материала «Velostat» для мобильного шагающего робота // Известия Волгоградского государственного технического университета. 2022. № 4. C. 6–12.
        \item Буличев О. В., Малолетов А. В. Метод оптимизации количества ног шагающего робота на основе эволюционного алгоритма // Известия Волгоградского государственного технического университета. 2022. № 9. C. 12–19.
    \end{enumerate}
\end{frame}

\begin{frame}[c]{}
    \framesubtitle{}
    \centering\LARGE Приложения
\end{frame}

\begin{frame}[t]{Апробация работы}
    \framesubtitle{В изданиях из списка Scopus}
    \begin{enumerate}
        \item Bulichev O., Klimchik A. Concept Development Of Biomimetic Centipede Robot StriRus // 2018 23rd Conference of Open Innovations Association (FRUCT). 2018. C. 85–90.
        \item Bulichev O., Klimchik A., Mavridis N. Optimization of Centipede Robot Body Designs through Evolutionary Algorithms and Multiple Rough Terrains Simulation // 2017 IEEE International Conference on Robotics and Biomimetics (ROBIO). 2017. C. 290–295.
    \end{enumerate}
\end{frame}

\begin{frame}[t]{Результаты интеллектуальной деятельности}
    \framesubtitle{}
    \large
    \begin{itemize}
        \item \textit{Количество публикаций}
              \begin{itemize}
                  \large
                  \item \textbf{2} --- журналы, рекомендованных ВАК
                  \item \textbf{3} --- журналы, индексируемые в Scopus (2 работы Q2)
                  \item \textbf{9} --- РИНЦ
                  \item \textbf{2} --- готовятся к публикации в Scopus
              \end{itemize}
        \item \textbf{8} --- Зарегистрированных программ для ЭВМ
        \item \textbf{3} --- Выигранных гранта (Умник, ЦНТИ, РФФИ)
    \end{itemize}
\end{frame}

\begin{frame}[t]{Соответствие паспорту специальности}
    \framesubtitle{2.5.4 Робототехника, Роботы, мехатроника и
        робототехнические системы}
    1. Развитие теоретических основ и методов анализа, структурного и параметрического синтеза и автоматизированного проектирования роботов и робототехнических систем. \\
    7. Методы экспериментального исследования, создания прототипов и
    экспериментальных стендов и модульных платформ для разработки роботов, робототехнических и мехатронных систем. \\
    9. Методы расчета и проектирования мехатронных сервоприводов,
    исполнительных, сенсорных и управляющих компонентов роботов,
    робототехнических и мехатронных систем.
\end{frame}

\begin{frame}[t]{Определение геометрических свойств поверхности}
    \framesubtitle{Метрики Cloud2Cloud и Cloud2Mesh}


    \begin{columns}[T,onlytextwidth]
        \begin{column}{0.49\textwidth}
            Метрики Cloud2Cloud и Cloud2Mesh основаны на метрике Хаусдорфа.
            \begin{equation*}
                d_{H}(X,\;Y)=\sup _{m\in M}\left\{\,|\mathrm {dist} _{X}(m)-\mathrm {dist} _{Y}(m)|\,\right\}
            \end{equation*}
            Где $X,\ Y$ --- непустые подмножества метрического пространства $M$; $\mathrm {dist} _{X}\colon M\to \mathbb {R}$ обозначает функцию расстояния до множества $X$.
        \end{column}
        \begin{column}{0.49\textwidth}
            \vspace{-0.8cm}
            \begin{figure}[H]
                \centering\includegraphics[height=6cm,width=1\textwidth,keepaspectratio]{Hausdorff_distance_sample.svg.png}
                % \caption{caption_name}
                \label{fig:Hausdorff_distance_sample.svg.png}
            \end{figure}
        \end{column}
    \end{columns}
\end{frame}

% \fbckg{fibeamer/figs/last_page.png}
% \frame[plain]{}
% \fbckg{fibeamer/figs/common.png}

\end{document}