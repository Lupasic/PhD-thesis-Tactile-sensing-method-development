
\textbf{{Во первой главе}} показан обзор существующих решений. Рассмотрены 3 глобальных темы: типы препятствий, которые могут встретиться; роботы, которые используются в исследованиях пещер; а так же методы построения карты местности.

Для решения поставленной цели необходимо понимать в каких условиях будет использоваться робот. Основные структуры поверхностей следующие \pic{fig:obstacles}:
\begin{itemize}
    \item твердые породы, прочные -- мрамор, кварц, базальт (магма);
    \item твердые породы, мягкие -- мел, гипс, соль, известняк;
    \item сыпучие грунты -- песок, глина, снег;
    \item водные преграды -- как и лужи (малый слой воды), так и целы залы, погруженные под воду. Часто встречаются сифоны;
    \item скользкие поверхности -- отложения мха и плесени, лед ;
    \item разрушаемые поверхности -- каменная гряда, паутина.
\end{itemize}



Так же были рассмотрены размеры пещер, чтобы понимать необходимый запас хода, размеры робототехнического комплекса.

В диссертации рассматривались роботы которые создавались специально для исследований пещер, в том числе и на Марсе. А так же те, которые потенциально могут быть использованы в условиях, определенных выше.

Как итог, их можно классифицировать следующим образом. Наземные роботы это шагающие, колесные, трековые и необычные. К необычным включены змеевидные, шарообразные и другие.

К летающим были отнесены защищенные дроны и дирижабли.

Продуктовым решением является робототехническая система, включающая в себя несколько роботов одного типа или комбинацию наземного и летающего роботов.

Были рассмотрены классические SLAM алгоритмы, основанные на использовании камеры, стереопары, с использованием лидара, GPS, IMU а так же их различные комбинации.

Были найдены способы получения облака точек объекта с помощью касания манипулятором данного объекта. Примерное местоположение объекта определялось камерой.

Определить тип поверхности можно так же с помощью различных сенсоров: визуально, IMU, с помощью снятия тока с моторов, момента с вала мотора, с помощью датчиков силы, установленных на конечность робота.

Были найдены следующие предложенные решения:
\begin{itemize}
    \item робототехнические системы для исследования свободных пещер;
    \item Построение карты с помощью лидаров и камер;
    \item Получение конечно элементной сетки с помощью тактильного очувствления манипулятором.
\end{itemize}

Поставленная задача является новой и не встречается в научных публикациях российских и зарубежных авторов.