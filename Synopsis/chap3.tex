
Взять статьи, которые писал до этого


\subsection{Сенсор
Velostat}

Существует несколько типов датчиков, которые могут измерять контактные
силы и распределение давления. Это могут быть оптические,
пьезорезистивные, пьезоэлектрические, магнитные, емкостные, на основе
оптических волокон. Промышленные датчики силы и момента (F/T) широко
распространены на гуманоидах (Atlas, Fedor) или четвероногих (Spot,
AnyMal). Однако они слишком велики для небольших роботов, таких как
RHEX, WHEGS или StriRus. Та же проблема применима к оптическим и
магнитным датчикам. Емкостные датчики требуют высокой точности
изготовления. Кроме того, датчики перечисленных типов довольно дороги,
что делает их использование нецелесообразным в исследовательских
роботах, которые работают в опасных условиях и могут быть потеряны в
процессе исследования пещеры. Недорогой альтернативой являются
тензометрические датчики.

Самый популярный тип тензометрического датчика - тензорезистивный
датчик. Другой способ - использовать пьезорезистивные датчики на основе
проводящих волокон или полимеров. Они недорогие, очень гибкие и
компактные. Одним из основных недостатков является значительный
гистерезис. В представленной работе используется материал Velostat
(Linqstat) в качестве промежуточного слоя для резистивного датчика.

При исследовании преобразователя силы на основе Velostat, было замечено,
что площадь нажатия разительно влияет на показания преобразователя.
Площадь контакта может сильно повлиять на данные при перемещении роботом
по неровной твердой поверхности, к примеру по камням. Поэтому было
решено характеризовать материал для случаев, когда нагрузка меньше, чем
размер сенсора.


\subsubsection{Конструкция
преобразователя}

Датчик состоит из двух медных оболочек, разделенных слоем Velostat.
Давление на датчик приводит к изменению его сопротивления: чем выше
давление, тем ниже сопротивление. Сопротивление измеряется косвенным
методом. Измеренное сопротивление Velostat образует делитель напряжения
с постоянным резистором R1\ldots R8 (Рисунок 1).

Рисунок 1 --- Электрическая схема преобразователя


\subsection{Разработка экспериментального
стенда}


\subsubsection{Экспериментальный
стенд}

Для выполнения серии экспериментов был разработан роботизированный
экспериментальный стенд. Стенд предназначен для исследования
преобразователя силы. Среди требований к стенду можно отметить:
необходимость контролировать силу нажатия и повторяемость эксперимента
как по величине, так и по расположению и ориентации площадки контакта
инструмента и исследуемого преобразователя силы. Указанным требованиям
возможно удовлетворить, используя коллаборативный робот-манипулятор.
Использование коллаборативного робота позволяет также удовлетворить
требованиям безопасности и допустить работу робота в непосредственно
близости от экспериментатора. Разработанный стенд, представлен на
Рисунке 3.

Рисунок 3 --- Разработанный экспериментальный стенд

Для касания только части объекта исследования были разработаны различные
концевые инструменты. К примеру, на рисунке 5 представлен инструмент для
точечного нажатия. Диаметр площади контакта равен 2 мм

Рисунок 5 --- Инструмент (концевой эффектор) для нажатия объект
исследования с диаметром нажатия меньше, чем сам объект


\subsection{Экспериментальная
часть}

Про 3 эксперимента