
\textbf{\underline{Вторая глава}} покрывает разработку объекта исследования, а именно решение задачи структурного синтеза и инженерную разработку прототипа.

Зная область применения робототехнической системы возможно оптимизировать ее механическую часть. Были выставлены следующие требования.
\begin{enumerate}
    \item иметь малые габариты, чтобы иметь возможность пролезать через щели в скальной породе и не застревать среди камней;
    \item обладать достаточной проходимостью по сыпучим грунтам;
    \item иметь возможность преодолевать малые водные преграды;
    \item мог взбираться на большие каменные уступы.
\end{enumerate}
Изучая данные требования возможно заметить, что часть из них коррелируют друг с другом, а часть - антогонируют. Чем больше количество полученных точек на пройденной поверхности, тем выше будет детализация карты. Одним из способов увеличения детализации это увеличение количества ног у робота. С другой стороны, это увеличивает длину робота, а следовательно робот хуже сможет проходить узкие участки с обилием поворотов. Чем большее расстояние робот сможет пройти за одно и то же время, тем быстрее будет построена карта и робот меньше повлияет на окружающую среду при прочих равных условиях. 

Как итог возникает задача, которая не имеет одного лучшего решения. Следовательно, это мультикритериальная задача оптимизации.

Было решено, что цикловой движитель с одной степенью свободы в ноге лучше всего подходит для решения подобных задач.

Для цикловых движителей с одной степенью свободы в ноге вопрос о количестве ног не имеет однозначного решения. Поэтому необходимо провести структурный синтез, чтобы определить их количество. Данная задача решалась с помощью генетического алгоритма.

Генетический алгоритм это эвристический алгоритм поиска, используемый для решения задач оптимизации и моделирования путём случайного подбора, комбинирования и вариации искомых параметров с использованием механизмов, аналогичных естественному отбору в природе. Для решения задачи использовалась библиотека Deap.

Математическая модель робота представлена следующим образом. Исследуется механическая система, состоящая из твёрдых тел \eqref{eq:newton_euler}, движение которых описывается дифференциальными уравнениями вида:

\begin{align}
    \label{eq:newton_euler}
    M \dot{\vec{u}} = \vec{g} \\
    M = \begin{bmatrix}
    M_1 & \cdots  & 0 \\
    \vdots  & \ddots  & \vdots  \\ 
    0 & \cdots   & M_n 
    \end{bmatrix},\ M_i = \begin{bmatrix}
    m_i E_{3\times 3} & 0 \\ 
    0 & I_i 
    \end{bmatrix} \\
    \vec{u}_i^{\ T} = \begin{bmatrix}
        \vec{v}_i^{\ T} & \vec{\omega}_i^{\ T}
    \end{bmatrix} \\ 
    \vec{g}^{\ T} = \begin{bmatrix}
        \cdots \  \vec{F}_i^{\ T}, & (\vec{\tau}_i - \vec{\omega}_i \times I_i \vec{\omega}_i)^T\  \cdots 
    \end{bmatrix}
\end{align}
где, $M_i$~---~матрицы, содержащие массово-инерционные характеристики; $m_i$~---~масса тела; $I_i$~---~тензор инерции; $\vec{u_i}$~---~вектор обобщённых скоростей; $E$~---~единичная матрица; $\vec{g}$~---~вектор обобщённых сил; $\vec{v_i}$~---~вектор линейной скорости; $\vec{\omega_i}$~---~вектор угловой скорости; $\vec{F_i}$, $\vec{\tau_i}$~---~силы и моменты сил взаимодействия.

Тела, входящие в систему соединены между собой цилиндрическими шарнирами, которые описываются следующими связями и динамическими ограничениями:
\begin{align}
    \label{eq:kin_constr}
    \phi(q_{j_1},\ u_{j_1},\ \cdots,\ q_{j_k},\ u_{j_k},\ t) \geqslant  0 \\
    \vec{q}_i^{\ T} = \begin{bmatrix}
        \vec{x}_i^{\ T} & \vec{Q}_i^{\ T}
    \end{bmatrix} \\
    \dot{\vec{q}_i} = \begin{bmatrix}
    E_{3\times3} & 0\\ 
    0 & G(\vec{q}_i) 
    \end{bmatrix}\vec{u}_i  
\end{align}

\begin{align}
    \label{eq:phys_constr}
    \vec{g}_i = \tau_i^T \vec{z}_{i-1} -k\vec{v}_i \dot{\vec{q}_i} 
\end{align}
где через $\phi$ обозначена функция связи; $t$~---~время; $q_{j}$~---~вектор обобщенных координат, включающий в себя координаты центра масс $\vec{x_i}$ и кватернион $\vec{Q_i}$, описывающий ориентацию тела в пространстве; через $G(\vec{q}_i)$ обозначена матрица, вид которой зависит от выбранной системы координат и способа задания ориентации тела; $k$~---~ коэффициент вязкого трения в шарнире.

Контакт ног робота с опорной поверхностью \pic{fig:contact_interaction.png} описывается на базе модели сухого трения и выражатеся следующими уравнениями:

\begin{figure}[H]
    \centering\includegraphics[height=6cm,width=1\textwidth,keepaspectratio]{images/contact_interaction.png}
    \caption{Описание переменных для модели взаимодействия опорной поверхности и ноги робота}
    \label{fig:contact_interaction.png}
\end{figure}

\begin{align}
    \label{eq:contact_inter}
    \phi_u(\vec{q}\ ) = g(\vec{q}\ ) \geqslant 0 \\ 
    g(\vec{q}\ ) = (\vec{x}_1 + \vec{s}_1 - \vec{x}_2 - \vec{s}_2) \cdot \vec{n} \\
    \frac{d }{d t}\phi_u(\vec{q}\ ) \approx \begin{bmatrix}
        \vec{n}^{\ T} & (\vec{s}_1 \times \vec{n})^T & -\vec{n}^{\ T} & (-\vec{s}_2 \times \vec{n})^T
    \end{bmatrix} \begin{bmatrix}
        \vec{v}_1\\ 
    \vec{\omega}_1\\ 
    \vec{v}_2\\
    \vec{\omega}_2\\
    \end{bmatrix}
\end{align} 

\begin{align}
    \label{eq:ground_inter}
\left\{\begin{matrix*}[l]
\mu f_n \geqslant \sqrt{f_1^2 + f_2^2}\\ 
\left\lVert \vec{v_t}\right\rVert (\mu f_n - \sqrt{f_1^2 + f_2^2}) = 0\\
\dfrac{\vec{f_t}}{\left\lVert \vec{f_t}\right\rVert } = - \dfrac{\vec{v_t}}{\left\lVert \vec{v_t}\right\rVert }
\end{matrix*}\right.
\end{align}
где, $\phi_u(\vec{q})$~---~функция связи; $ \mu $~---~ коэффициент трения между ногой и опорной поверхностью;  радиус-векторы $\vec{x}_{1,2},\ \vec{s}_{1,2}$ и орты координатных осей $\vec{t}_{1,2}, \vec{n}$ показаны на рисунке \pic{fig:contact_interaction.png}; $ f_{1,2} $~---~значения сил трения вдоль осей $t_{1,2}$ соответственно.

Геометрическая модель робота представлена в виде трехмерного параллелепипеда. Количество движителей по каждому из бортов обозначается через $\gamma$. Разность фаз между соседними движителями обозначается через  $\alpha$ \pic{fig:best_gen_robot.jpg}.

\begin{figure}[H]
    \centering
    \begin{tikzpicture}
        % Include the image in a node
        \node [above right, inner sep=0] (image) at (0,0)
        {\centering\includegraphics[height=2.7cm,width=1\textwidth,keepaspectratio]{best_gen_robot.jpg}};
        % Create scope with normalized axes
        \begin{scope}[
                x={($ 0.1*(image.south east)$)},
                y={($ 0.1*(image.north west)$)}]
            % Labels
            \draw [green, very thick,
                decorate,
                decoration = {brace,
                        raise=5pt,
                        amplitude=5pt,
                        aspect=0.5}] (1.4,3.6) --  (8.1,6.8)
            node[rounded corners=3pt, pos=0.5,above left =14pt,black,fill=white]{\tiny $(\gamma - 1) h_{\text{leg}}sin(\alpha)$};

            \draw[stealth-, very thick,green] (9.5,7.8) -- (7.8,1.94);
            \draw[stealth-, very thick,green] (1.5,2.8) -- (7,1)
            node[rounded corners=3pt,right,black,fill=white]{\tiny $\gamma = 6$};

            \draw[thin,green] (6.7,4) -- (5.75,9);
            \draw[thin,green] (4.85,3.5) -- (5.75,9);
            \draw[thin,green,stealth-stealth] (6.32,6) arc (-79.2:-99.2:3) node [rounded corners=3pt,below = 2pt,black,fill=white, midway] {\tiny $\alpha$};
        \end{scope}
    \end{tikzpicture}
    \caption{Схема модели робота для генетического алгоритма}
    \label{fig:best_gen_robot.jpg}
\end{figure}

Эту задачу можно сформулировать как мультикритериальную задачу оптимизации, где необходимо максимизировать дистанцию, пройденную за фиксированное время, и минимизировать длину робота \eqref{eq:second}. Параметрами индивида являлись $\gamma$ и $\alpha$.

\begin{eqnarray}
    \label{eq:second}
    F \rightarrow max = \beta \left( {\omega}_{1} \cdot \overbrace{\delta}^{\text{Distance}} + {\omega}_{2} \cdot \overbrace{\frac{1}{(\gamma - 1) h_{\text{leg}}sin(\alpha)}}^{\text{Simplified body length}}\right) + \\ \nonumber + (1 - \beta) {\delta}^{{\omega}_{1}} {\left( \frac{1}{(\gamma - 1)h_{\text{leg}}sin(\alpha)}\right)}^{{\omega}_{2}}
\end{eqnarray}
где $\delta$ дистанция, $\beta$ адаптивный параметр, ${\omega}_{1,2} \in  [ 0..1 ] $ весовые коэффициенты.


Весовые коэффициенты настраивались в зависимости от выбора приоритета. Невзирая на выбранные коэффициенты, оптимальным набор ног начинался с 8 и заканчивался 14. Это объясняется критерием статического равновесия, который, как оказалось, увеличивает проходимость механизма. В данном случае 4 ноги всегда будут касаться пола. 

Было проведено два испытания. На первом испытании мы стремились найти только одного лучшего робота, только для местности T1 \pic{fig:terrain_1}. На втором этапе мы хотели видеть зависимость от разных типов ландшафтов при меньшем количестве индивидуальностей.

Первый этап: каждый робот проходил 10 разных ландшафтов по 9 секунд каждую. Вторая фаза: она имеет те же параметры, что и первая фаза, но с измененным размером популяции. 

В соответствии с таблицей \ref{tabular:Table2} (весовые коэффициенты равны 0.6 и 0.4 соответственно) видно, что мы имеем сходимость в параметрах. Видео прохождения препятствия лучшим индивидом \quad
\qrcode[height=1.5cm]{https://youtu.be/DcovvkTZgsg}

\begin{figure}[h]
    \begin{subfigure}{0.33\textwidth}
    \centering\includegraphics[width=0.8\textwidth]{terrain_1} 
    \caption{T1: 3D-боксы с равномерным распределением высоты}
    \label{fig:terrain_1}
    \end{subfigure}
    \begin{subfigure}{0.33\textwidth}
    \centering\includegraphics[width=0.8\textwidth]{terrain_2} 
    \caption{T2: 2D-полосы с гауссовой функциональной высотой}
    \label{fig:terrain_2}
    \end{subfigure}
    \begin{subfigure}{0.33\textwidth}
    \centering\includegraphics[width=0.8\textwidth]{terrain_3}
    \caption{T3: 2D-полосы с распределением высоты по гауссовской функции)}
    \label{fig:terrain_3}
    \end{subfigure}
     
    \caption{Примеры сгенерированных территорий}
    \label{fig:terrains}
\end{figure}
\vspace{-0.5cm}

\begin{table}[H]
\caption{Зависимость между статистикой значения пригодности и типами ландшафта}
\label{tabular:Table2}
\begin{center}
\begin{tabular}{c|c|c|c}

\textbf{\makecell{Территория, популяция}} & \textbf{\makecell{Параметры}} & \textbf{\makecell{Среднее \\значение }} & \textbf{\makecell{Std \\целевая функция}}\\
\hline
\textbf{\makecell{T1 \pic{fig:terrain_1}, 110}} & \makecell{(6, 72)} & \makecell{2.38} & \makecell{0.34}
\\
\textbf{\makecell{T2 \pic{fig:terrain_2}, 55}}& \makecell{(5, 68)} & \makecell{1.95} & \makecell{0.35} 
\\
\textbf{\makecell{T3 \pic{fig:terrain_3}, 55}} & \makecell{(6, 77)} &  \makecell{2.08} & \makecell{0.33} \\
\hline
\end{tabular}
\end{center}
\end{table}

В первом пункте требований к движителю (начало главы) стоит требование, чтобы робот не застревал при поворотах. Проблема застревания решается с помощью изменения угла между ногой и корпусом робота.

\begin{figure}[H]
    \centering\includegraphics[height=3cm,width=1\textwidth,keepaspectratio]{omni_rot.png}
    \caption{Векторное представление сил в классическом и всенаправленном состоянии}
    \label{fig:omnidirection}
\end{figure}

На рисунке \ref{fig:omnidirection} представлена иллюстрация данной концепции: для того, чтобы робот двигался во всех направлениях, необходимо разбить ноги на группы, чтобы получилось 4 группы A-D.

Если сравнивать с классической компоновкой роботов (угол между корпусом робота и осью вала привода ноги равен 90 градусов), то вектор внешних сил будет таким, как на левой части рис. \ref{fig:omnidirection}. Стрелка в центре робота — суперпозиция всех сил. Если изменить угол оси привода ноги в соответствии с предлагаемой концепцией, то возможно получить значения суперпозиции сил, представленные на рис. \ref{fig:omnidirection} в центре. То есть, чтобы переместить корпус робота направо, группы А и D должны вращать ноги в одну сторону, а группы C и B — в противоположную. Правая часть рисунка иллюстрирует расположение групп ног на исследуемом роботе. 

В рамках исследования было разработано четыре концепции робота СтриРус. В таблице \ref{tabular:robot_comparison} в строке недостатки объясняются основные причины перехода из одной итерации к другой. Концептуально было замечено, что высота ноги и наличие сегмента разительно влияет на проходимость конструкции. \quad \qrcode[height=1.5cm]{https://youtu.be/EQ6oGZVDpoc}

\begin{figure}[H]
    \centerfloat{
        \hfill
        \subcaptionbox[List-of-Figures entry]{Первая итерация\label{fig:strirus_0}}{%
            \includegraphics[width=0.33\linewidth]{strirus_0.png}}
        \hfill
        \subcaptionbox[List-of-Figures entry]{Вторая итерация \label{fig:strirus_1}}{%
            \includegraphics[width=0.33\linewidth]{strirus_1.png}}
        \hfill
        \subcaptionbox{Третья итерация\label{fig:strirus_2}}{%
        \includegraphics[width=0.33\linewidth]{strirus_2.jpg}}
        \hfill
        \subcaptionbox{Третья итерация, улучшенная\label{fig:strirus_3}}{%
        \includegraphics[width=0.35\linewidth]{strirus_3.JPG}}
        \hfill
        \subcaptionbox{Четвертая итерация\label{fig:strirus_4}}{%
        \includegraphics[width=0.35\linewidth]{strirus_4.png}}
    }
    % \legend{Подрисуночный текст, описывающий обозначения, например. Согласно
    %     ГОСТ 2.105, пункт 4.3.1, располагается перед наименованием рисунка.}
    \caption{Итерации робота СтриРуса}\label{fig:striruses}
  \end{figure}

\vspace{-1cm}

\begin{table}[H]
    \caption{Сравнение итераций робота}
    \label{tabular:robot_comparison}
    % \begin{center}
    \begin{footnotesize}
    \begin{tabular}{p{1.6cm}|p{1.6cm}|p{1.5cm}|p{1.7cm}|p{1.4cm}|p{1.4cm}}
    \toprule
    \toprule
    % \rowcolor{Gray}
     Итерация & 1 \pic{fig:strirus_0}  & 2 \pic{fig:strirus_1} &  3 \pic{fig:strirus_2} & 3+ \pic{fig:strirus_3} & 4 \pic{fig:strirus_4} \\
     \hline
     Кол-во ног & 54 & 12 & 12 & 6 & 10 \\ 
    %   \rowcolor{lightgray}
     \makecell[l]{Кол-во \\ сегментов} & 1 & 2 & 2 & 1 & 2 \\
     \makecell[l]{Тип \\ соединения} & --- & Тангаж & \makecell[l]{Тангаж,\\ рыскание} & --- & Тангаж \\
    %  \rowcolor{lightgray}
     Отн. угол телом -- нога, градусы & 0 & 0--45 & 0, 15, 30, 45 & 0 & 0, 15 \\
     \makecell[l]{Высота \\ ноги, мм} & 54 & 60 & 60 & 90 & 170 \\
     \hline
     Особенности & Волноход & Механизм, который позволяет непрерывно изменять отн. угол & Двухстепенной узел, соединяющий сегменты & Большие ноги & Гигантские ноги  \\
    %  \rowcolor{lightgray}
    \hline
     Недостатки & Невозможно установить сенсоры на ноги. Много подвижных частей & Слишком сложный механизм, изменяющий отн. угол & Мал. ноги. Избыточная вторая степень свободы в соединительном узле & 1 сегмент. Маленькие ноги & --- \\
    \bottomrule
    \bottomrule
    \end{tabular}
    \end{footnotesize}
    % \end{center}
    \end{table}

Как итог, был разработан 10 ногий двух сегментный робот СтриРус. 10 ног было выбрано на основе результатов, полученных во время решения мультикритериальной задачи оптимизации с помощью генетического алгоритма.

Конструкция робота соответствует всем требованиям, поставленным вначале. А именно, возможность проходить сквозь узкие пространства, иметь возможность преодолевать большие каменные гряды и возможность эффективно перемещаться по сыпучим грунтам.


