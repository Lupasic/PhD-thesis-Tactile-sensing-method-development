\section{Выводы}

\begin{frame}[t]{Результаты решения задач}
    \framesubtitle{}
    \begin{columns}[T,onlytextwidth]
        \begin{column}{0.48\textwidth}
            \begin{block}{Научных задач (научная новизна)}
                1. Метод \textbf{подбора количества ног для шагающих цикловых движителей}.

                2. Методика \textbf{характеризации датчика}, когда площадь касания нагрузки меньше, чем размеры датчика.

                3. Алгоритмы \textbf{калибровки} и \textbf{определения физических свойств поверхности}.

                4. Метод определения \textbf{геометрических свойств местности}.

            \end{block}
        \end{column}
        \begin{column}{0.48\textwidth}
            \begin{alertblock}{Экспериментальных разработок}
                1. Спроектированы и собраны 2 прототипа с \textbf{Шагающим цикловым движителем} с одной степенью свободы в ноге.

                2. Разработана и создана \textbf{экспериментальная установка} для \textbf{автоматизированного исследования датчика силы}.

                3. Разработана и создана \textbf{экспериментальная установка} для \textbf{определения типа поверхности}.

            \end{alertblock}
        \end{column}
    \end{columns}
\end{frame}

\begin{frame}[t]{Результаты интеллектуальной деятельности}
    \framesubtitle{}
    \begin{itemize}
        \item \textit{Количество публикаций}
              \begin{itemize}
                  \item \textbf{2} --- журналы, рекомендованных ВАК
                  \item \textbf{3} --- журналы, индексируемые в Scopus
                  \item \textbf{5} --- РИНЦ
              \end{itemize}
        \item \textbf{8} --- Зарегистрированных программ для ЭВМ
        \item \textbf{3} --- Выигранных гранта (Умник, ЦНТИ, РФФИ)
    \end{itemize}
\end{frame}



\begin{frame}[t]{Соответствие паспорту специальности}
    \framesubtitle{2.5.4 Робототехника, Роботы, мехатроника и
    робототехнические системы}
    1. Развитие теоретических основ и методов анализа, структурного и параметрического синтеза и автоматизированного проектирования роботов и робототехнических систем. \\
    7. Методы экспериментального исследования, создания прототипов и
    экспериментальных стендов и модульных платформ для разработки роботов, робототехнических и мехатронных систем. \\
    9. Методы расчета и проектирования мехатронных сервоприводов,
    исполнительных, сенсорных и управляющих компонентов роботов,
    робототехнических и мехатронных систем.
\end{frame}

% \begin{frame}[t]{Апробация работы}
%     \framesubtitle{Статьи в периодических изданиях по перечню ВАК РФ}
%     \begin{enumerate}
%         \item Буличев О. В., Полёткин К. В., Малолетов А. В. Исследование характеристик датчика силы на основе материала «Velostat» для мобильного шагающего робота // Известия Волгоградского государственного технического университета. 2022. № 4. C. 6–12.
%         \item Буличев О. В., Малолетов А. В. Метод оптимизации количества ног шагающего робота на основе эволюционного алгоритма // Известия Волгоградского государственного технического университета. 2022. № 9. C. 12–19.
%     \end{enumerate}
% \end{frame}