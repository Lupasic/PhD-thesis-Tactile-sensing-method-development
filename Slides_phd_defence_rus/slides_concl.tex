\section{Выводы}

\begin{frame}[t]{Результаты}
    \framesubtitle{}
    \begin{columns}[T,onlytextwidth]
        \begin{column}{0.48\textwidth}
            \begin{block}{Научные задачи (научная новизна)}
                1. Метод \textbf{подбора количества ног для шагающих цикловых движителей}.

                2. Методика \textbf{характеризации датчика}, когда площадь касания нагрузки меньше, чем размеры датчика.

                3. Алгоритмы \textbf{калибровки} и \textbf{определения физических свойств поверхности}.

                4. Метод определения \textbf{геометрических свойств местности}.

            \end{block}
        \end{column}
        \begin{column}{0.48\textwidth}
            \begin{alertblock}{Экспериментальные разработки}
                1. Спроектированы и собраны 2 прототипа с \textbf{Шагающим цикловым движителем} с одной степенью свободы в ноге.

                2. Разработана и создана \textbf{экспериментальная установка} для \textbf{автоматизированного исследования датчика силы}.

                3. Разработана и создана \textbf{экспериментальная установка} для \textbf{определения типа поверхности}.

            \end{alertblock}
        \end{column}
    \end{columns}
\end{frame}

\note{\small \setlength{\parindent}{20pt}

Подведя итог, было решено 4 научных задачи. Было разработано 2 метода --- оптимизация кинематической схемы робота, а также определения геометрических свойств опорной поверхности. Это задачи 1 и 3.

Был разработана методика для определения характеристик преобразователя силы на основе Велостат, что являлось задачей 4.

А также разработан алгоритм калибровки и определения физико-механических свойств поверхности. Задача 2 и подзадача 4й задачи.

Для решения этих научных задач были собраны 2 прототипа робота, созданы 2 экспериментальных установки.}

\begin{frame}[t]{Публикации}
    \framesubtitle{}
    \begin{itemize}
        \item \textit{Количество публикаций}
              \begin{itemize}
                  \item \textbf{2} --- журналы, рекомендованных ВАК
                  \item \textbf{3} --- статьи, индексируемые в Scopus
                  \item \textbf{5} --- РИНЦ
              \end{itemize}
        \item \textbf{8} --- Зарегистрированных программ для ЭВМ
        \item \textbf{3} --- Работа сделана при грантовой поддержке ФСИ, ЦНТИ, РФФИ.
    \end{itemize}
\end{frame}

\note{\small \setlength{\parindent}{20pt}

Мной были опубликованы работы в 2ух ваковских журналах, 3 статьи в Скопус и еще 5 работ в РИНЦ.

На мое имя зарегистрированы 8 программ для ЭВМ.

Работа делалась при поддержке 3ех фондов ФСИ, ЦНТИ и РФФИ}

\begin{frame}[t]{Соответствие паспорту специальности}
    \framesubtitle{2.5.4 Робототехника, Роботы, мехатроника и
    робототехнические системы}
    1. Развитие теоретических основ и методов анализа, структурного и параметрического синтеза и автоматизированного проектирования роботов и робототехнических систем. \\
    7. Методы экспериментального исследования, создания прототипов и
    экспериментальных стендов и модульных платформ для разработки роботов, робототехнических и мехатронных систем. \\
    9. Методы расчета и проектирования мехатронных сервоприводов,
    исполнительных, сенсорных и управляющих компонентов роботов,
    робототехнических и мехатронных систем.
\end{frame}

\note{\small \setlength{\parindent}{20pt}

Моя работа соответствует пунктам 1, 7 и 9 паспорта специальности}